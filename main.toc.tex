\boolfalse {citerequest}\boolfalse {citetracker}\boolfalse {pagetracker}\boolfalse {backtracker}\relax 
\babel@toc {ngerman}{}
\defcounter {refsection}{0}\relax 
\contentsline {chapter}{\nonumberline Vorbemerkungen zur ersten Auf"|lage}{xiii}{chapter*.1}
\defcounter {refsection}{0}\relax 
\contentsline {chapter}{\nonumberline Vorbemerkungen zur zweiten Auf"|lage}{xvii}{chapter*.5}
\defcounter {refsection}{0}\relax 
\contentsline {chapter}{\nonumberline Vorbemerkungen zur dritten Auf"|lage}{xix}{chapter*.8}
\defcounter {refsection}{0}\relax 
\contentsline {part}{\numberline {I}Grundlagen}{1}{part.1}
\defcounter {refsection}{0}\relax 
\contentsline {chapter}{\numberline {1}Grammatik}{3}{chapter.1}
\defcounter {refsection}{0}\relax 
\contentsline {section}{\numberline {1.1}Sprache und Grammatik}{3}{section.1.1}
\defcounter {refsection}{0}\relax 
\contentsline {subsection}{\numberline {1.1.1}Sprache}{3}{subsection.1.1.1}
\defcounter {refsection}{0}\relax 
\contentsline {subsection}{\numberline {1.1.2}Grammatik als System}{5}{subsection.1.1.2}
\defcounter {refsection}{0}\relax 
\contentsline {subsection}{\numberline {1.1.3}Akzeptabilität und Grammatikalität}{6}{subsection.1.1.3}
\defcounter {refsection}{0}\relax 
\contentsline {subsection}{\numberline {1.1.4}Ebenen der Grammatik}{8}{subsection.1.1.4}
\defcounter {refsection}{0}\relax 
\contentsline {subsection}{\numberline {1.1.5}Kern und Peripherie}{9}{subsection.1.1.5}
\defcounter {refsection}{0}\relax 
\contentsline {section}{\numberline {1.2}Deskriptive und präskriptive Grammatik}{13}{section.1.2}
\defcounter {refsection}{0}\relax 
\contentsline {subsection}{\numberline {1.2.1}Beschreibung und Vorschrift}{13}{subsection.1.2.1}
\defcounter {refsection}{0}\relax 
\contentsline {subsection}{\numberline {1.2.2}Regel, Regularität und Generalisierung}{15}{subsection.1.2.2}
\defcounter {refsection}{0}\relax 
\contentsline {subsection}{\numberline {1.2.3}Norm als Beschreibung}{18}{subsection.1.2.3}
\defcounter {refsection}{0}\relax 
\contentsline {subsection}{\numberline {1.2.4}Empirie}{19}{subsection.1.2.4}
\defcounter {refsection}{0}\relax 
\contentsline {chapter}{\numberline {2}Grundbegriffe der Grammatik}{25}{chapter.2}
\defcounter {refsection}{0}\relax 
\contentsline {section}{\numberline {2.1}Merkmale und Werte}{25}{section.2.1}
\defcounter {refsection}{0}\relax 
\contentsline {section}{\numberline {2.2}Relationen}{27}{section.2.2}
\defcounter {refsection}{0}\relax 
\contentsline {subsection}{\numberline {2.2.1}Kategorien}{27}{subsection.2.2.1}
\defcounter {refsection}{0}\relax 
\contentsline {subsection}{\numberline {2.2.2}Paradigma und Syntagma}{30}{subsection.2.2.2}
\defcounter {refsection}{0}\relax 
\contentsline {subsection}{\numberline {2.2.3}Strukturbildung}{34}{subsection.2.2.3}
\defcounter {refsection}{0}\relax 
\contentsline {subsection}{\numberline {2.2.4}Rektion und Kongruenz}{37}{subsection.2.2.4}
\defcounter {refsection}{0}\relax 
\contentsline {section}{\numberline {2.3}Valenz}{40}{section.2.3}
\defcounter {refsection}{0}\relax 
\contentsline {chapter}{\numberline {3}Grammatik und Lehramt}{49}{chapter.3}
\defcounter {refsection}{0}\relax 
\contentsline {section}{\numberline {3.1}Grammatik in der Schule}{49}{section.3.1}
\defcounter {refsection}{0}\relax 
\contentsline {subsection}{\numberline {3.1.1}Bildungssprache und ihr Erwerb}{49}{subsection.3.1.1}
\defcounter {refsection}{0}\relax 
\contentsline {subsection}{\numberline {3.1.2}Sprachbetrachtung und Konzepte der Deutschdidaktik}{52}{subsection.3.1.2}
\defcounter {refsection}{0}\relax 
\contentsline {subsection}{\numberline {3.1.3}System und Funktion im Grammatikunterricht}{55}{subsection.3.1.3}
\defcounter {refsection}{0}\relax 
\contentsline {section}{\numberline {3.2}Grammatik im Lehramtsstudium}{58}{section.3.2}
\defcounter {refsection}{0}\relax 
\contentsline {subsection}{\numberline {3.2.1}Aufgaben der Linguistik im Lehramtsstudium}{58}{subsection.3.2.1}
\defcounter {refsection}{0}\relax 
\contentsline {subsection}{\numberline {3.2.2}Form und Funktion in der Grammatik}{61}{subsection.3.2.2}
\defcounter {refsection}{0}\relax 
\contentsline {subsection}{\numberline {3.2.3}Grammatikkenntnisse von Studierenden}{65}{subsection.3.2.3}
\defcounter {refsection}{0}\relax 
\contentsline {subsection}{\numberline {3.2.4}Studentische Sichtweisen auf Studium und Schulunterricht}{70}{subsection.3.2.4}
\defcounter {refsection}{0}\relax 
\contentsline {part}{\numberline {II}Phonetik und Phonologie}{77}{part.2}
\defcounter {refsection}{0}\relax 
\contentsline {chapter}{\numberline {4}Phonetik}{79}{chapter.4}
\defcounter {refsection}{0}\relax 
\contentsline {section}{\numberline {4.1}Grundlagen der Phonetik}{79}{section.4.1}
\defcounter {refsection}{0}\relax 
\contentsline {subsection}{\numberline {4.1.1}Das akustische Medium}{79}{subsection.4.1.1}
\defcounter {refsection}{0}\relax 
\contentsline {subsection}{\numberline {4.1.2}Orthographie und Graphematik}{80}{subsection.4.1.2}
\defcounter {refsection}{0}\relax 
\contentsline {subsection}{\numberline {4.1.3}Segmente und Merkmale}{82}{subsection.4.1.3}
\defcounter {refsection}{0}\relax 
\contentsline {section}{\numberline {4.2}Anatomische Grundlagen}{83}{section.4.2}
\defcounter {refsection}{0}\relax 
\contentsline {subsection}{\numberline {4.2.1}Zwerchfell, Lunge und Luftröhre}{84}{subsection.4.2.1}
\defcounter {refsection}{0}\relax 
\contentsline {subsection}{\numberline {4.2.2}Kehlkopf und Rachen}{85}{subsection.4.2.2}
\defcounter {refsection}{0}\relax 
\contentsline {subsection}{\numberline {4.2.3}Mundraum, Zunge und Nase}{85}{subsection.4.2.3}
\defcounter {refsection}{0}\relax 
\contentsline {section}{\numberline {4.3}Artikulationsart}{87}{section.4.3}
\defcounter {refsection}{0}\relax 
\contentsline {subsection}{\numberline {4.3.1}Passiver und aktiver Artikulator}{87}{subsection.4.3.1}
\defcounter {refsection}{0}\relax 
\contentsline {subsection}{\numberline {4.3.2}Stimmhaftigkeit}{88}{subsection.4.3.2}
\defcounter {refsection}{0}\relax 
\contentsline {subsection}{\numberline {4.3.3}Obstruenten}{89}{subsection.4.3.3}
\defcounter {refsection}{0}\relax 
\contentsline {subsection}{\numberline {4.3.4}Approximanten}{91}{subsection.4.3.4}
\defcounter {refsection}{0}\relax 
\contentsline {subsection}{\numberline {4.3.5}Nasale}{91}{subsection.4.3.5}
\defcounter {refsection}{0}\relax 
\contentsline {subsection}{\numberline {4.3.6}Vokale}{92}{subsection.4.3.6}
\defcounter {refsection}{0}\relax 
\contentsline {subsection}{\numberline {4.3.7}Oberklassen für Artikulationsarten}{93}{subsection.4.3.7}
\defcounter {refsection}{0}\relax 
\contentsline {section}{\numberline {4.4}Artikulationsort}{95}{section.4.4}
\defcounter {refsection}{0}\relax 
\contentsline {subsection}{\numberline {4.4.1}Das IPA-Alphabet}{95}{subsection.4.4.1}
\defcounter {refsection}{0}\relax 
\contentsline {subsection}{\numberline {4.4.2}Laryngale}{96}{subsection.4.4.2}
\defcounter {refsection}{0}\relax 
\contentsline {subsection}{\numberline {4.4.3}Uvulare}{96}{subsection.4.4.3}
\defcounter {refsection}{0}\relax 
\contentsline {subsection}{\numberline {4.4.4}Velare}{97}{subsection.4.4.4}
\defcounter {refsection}{0}\relax 
\contentsline {subsection}{\numberline {4.4.5}Palatale}{97}{subsection.4.4.5}
\defcounter {refsection}{0}\relax 
\contentsline {subsection}{\numberline {4.4.6}Palatoalveolare und Alveolare}{97}{subsection.4.4.6}
\defcounter {refsection}{0}\relax 
\contentsline {subsection}{\numberline {4.4.7}Labiodentale und Bilabiale}{98}{subsection.4.4.7}
\defcounter {refsection}{0}\relax 
\contentsline {subsection}{\numberline {4.4.8}Affrikaten}{99}{subsection.4.4.8}
\defcounter {refsection}{0}\relax 
\contentsline {subsection}{\numberline {4.4.9}Vokale und Diphthonge}{99}{subsection.4.4.9}
\defcounter {refsection}{0}\relax 
\contentsline {section}{\numberline {4.5}Phonetische Merkmale}{103}{section.4.5}
\defcounter {refsection}{0}\relax 
\contentsline {section}{\numberline {4.6}Besonderheiten der Transkription}{104}{section.4.6}
\defcounter {refsection}{0}\relax 
\contentsline {subsection}{\numberline {4.6.1}Endrand-Desonorisierung}{105}{subsection.4.6.1}
\defcounter {refsection}{0}\relax 
\contentsline {subsection}{\numberline {4.6.2}Silbische Nasale und Approximanten}{105}{subsection.4.6.2}
\defcounter {refsection}{0}\relax 
\contentsline {subsection}{\numberline {4.6.3}Orthographisches \textit {n}}{106}{subsection.4.6.3}
\enlargethispage{\baselineskip}
\defcounter {refsection}{0}\relax 
\contentsline {subsection}{\numberline {4.6.4}Orthographisches \textit {s}}{106}{subsection.4.6.4}
\defcounter {refsection}{0}\relax 
\contentsline {subsection}{\numberline {4.6.5}Orthographisches \textit {r}}{107}{subsection.4.6.5}
\defcounter {refsection}{0}\relax 
\contentsline {chapter}{\numberline {5}Phonologie}{111}{chapter.5}
\defcounter {refsection}{0}\relax 
\contentsline {section}{\numberline {5.1}Segmente}{111}{section.5.1}
\defcounter {refsection}{0}\relax 
\contentsline {subsection}{\numberline {5.1.1}Segmente und Verteilungen}{111}{subsection.5.1.1}
\defcounter {refsection}{0}\relax 
\contentsline {subsection}{\numberline {5.1.2}Zugrundeliegende Formen und Strukturbedingungen}{114}{subsection.5.1.2}
\defcounter {refsection}{0}\relax 
\contentsline {subsection}{\numberline {5.1.3}Endrand-Desonorisierung}{116}{subsection.5.1.3}
\defcounter {refsection}{0}\relax 
\contentsline {subsection}{\numberline {5.1.4}Gespanntheit, Betonung und Länge}{117}{subsection.5.1.4}
\defcounter {refsection}{0}\relax 
\contentsline {subsection}{\numberline {5.1.5}Verteilung von [ç] und [χ]}{121}{subsection.5.1.5}
\defcounter {refsection}{0}\relax 
\contentsline {subsection}{\numberline {5.1.6}/ʁ/-Vokalisierungen}{122}{subsection.5.1.6}
\defcounter {refsection}{0}\relax 
\contentsline {section}{\numberline {5.2}Silben und Wörter}{123}{section.5.2}
\defcounter {refsection}{0}\relax 
\contentsline {subsection}{\numberline {5.2.1}Phonotaktik}{123}{subsection.5.2.1}
\defcounter {refsection}{0}\relax 
\contentsline {subsection}{\numberline {5.2.2}Silben}{124}{subsection.5.2.2}
\defcounter {refsection}{0}\relax 
\contentsline {subsection}{\numberline {5.2.3}Silbenstruktur}{126}{subsection.5.2.3}
\defcounter {refsection}{0}\relax 
\contentsline {subsection}{\numberline {5.2.4}Der Anfangsrand im Einsilbler}{128}{subsection.5.2.4}
\defcounter {refsection}{0}\relax 
\contentsline {subsection}{\numberline {5.2.5}Der Endrand im Einsilbler}{130}{subsection.5.2.5}
\defcounter {refsection}{0}\relax 
\contentsline {subsection}{\numberline {5.2.6}Sonorität}{133}{subsection.5.2.6}
\defcounter {refsection}{0}\relax 
\contentsline {subsection}{\numberline {5.2.7}Die Systematik der Ränder}{138}{subsection.5.2.7}
\defcounter {refsection}{0}\relax 
\contentsline {subsection}{\numberline {5.2.8}Einsilbler und Zweisilbler}{145}{subsection.5.2.8}
\defcounter {refsection}{0}\relax 
\contentsline {subsection}{\numberline {5.2.9}Maximale Anfangsränder}{151}{subsection.5.2.9}
\defcounter {refsection}{0}\relax 
\contentsline {section}{\numberline {5.3}Wortakzent}{152}{section.5.3}
\defcounter {refsection}{0}\relax 
\contentsline {subsection}{\numberline {5.3.1}Prosodie}{152}{subsection.5.3.1}
\defcounter {refsection}{0}\relax 
\contentsline {subsection}{\numberline {5.3.2}Wortakzent im Deutschen}{154}{subsection.5.3.2}
\defcounter {refsection}{0}\relax 
\contentsline {subsection}{\numberline {5.3.3}Prosodische Wörter}{159}{subsection.5.3.3}
\defcounter {refsection}{0}\relax 
\contentsline {part}{\numberline {III}Morphologie}{167}{part.3}
\defcounter {refsection}{0}\relax 
\contentsline {chapter}{\numberline {6}Wortklassen}{169}{chapter.6}
\defcounter {refsection}{0}\relax 
\contentsline {section}{\numberline {6.1}Wörter}{169}{section.6.1}
\defcounter {refsection}{0}\relax 
\contentsline {subsection}{\numberline {6.1.1}Definitionsprobleme}{169}{subsection.6.1.1}
\defcounter {refsection}{0}\relax 
\contentsline {subsection}{\numberline {6.1.2}Wörter und Wortformen}{172}{subsection.6.1.2}
\defcounter {refsection}{0}\relax 
\contentsline {section}{\numberline {6.2}Klassifikationsmethoden}{174}{section.6.2}
\defcounter {refsection}{0}\relax 
\contentsline {subsection}{\numberline {6.2.1}Semantische Klassifikation}{174}{subsection.6.2.1}
\defcounter {refsection}{0}\relax 
\contentsline {subsection}{\numberline {6.2.2}Paradigmatische Klassifikation}{176}{subsection.6.2.2}
\defcounter {refsection}{0}\relax 
\contentsline {subsection}{\numberline {6.2.3}Syntagmatische Klassifikation}{178}{subsection.6.2.3}
\defcounter {refsection}{0}\relax 
\contentsline {section}{\numberline {6.3}Wortklassen des Deutschen}{180}{section.6.3}
\defcounter {refsection}{0}\relax 
\contentsline {subsection}{\numberline {6.3.1}Filtermethode}{180}{subsection.6.3.1}
\defcounter {refsection}{0}\relax 
\contentsline {subsection}{\numberline {6.3.2}Flektierbare Wörter}{181}{subsection.6.3.2}
\defcounter {refsection}{0}\relax 
\contentsline {subsection}{\numberline {6.3.3}Verben und Nomina}{182}{subsection.6.3.3}
\defcounter {refsection}{0}\relax 
\contentsline {subsection}{\numberline {6.3.4}Substantive}{183}{subsection.6.3.4}
\defcounter {refsection}{0}\relax 
\contentsline {subsection}{\numberline {6.3.5}Adjektive}{184}{subsection.6.3.5}
\defcounter {refsection}{0}\relax 
\contentsline {subsection}{\numberline {6.3.6}Präpositionen}{184}{subsection.6.3.6}
\defcounter {refsection}{0}\relax 
\contentsline {subsection}{\numberline {6.3.7}Komplementierer}{185}{subsection.6.3.7}
\defcounter {refsection}{0}\relax 
\contentsline {subsection}{\numberline {6.3.8}Adverben, Adkopulas und Partikeln}{186}{subsection.6.3.8}
\defcounter {refsection}{0}\relax 
\contentsline {subsection}{\numberline {6.3.9}Adverben und Adkopulas}{187}{subsection.6.3.9}
\defcounter {refsection}{0}\relax 
\contentsline {subsection}{\numberline {6.3.10}Satzäquivalente}{188}{subsection.6.3.10}
\defcounter {refsection}{0}\relax 
\contentsline {subsection}{\numberline {6.3.11}Konjunktionen}{189}{subsection.6.3.11}
\defcounter {refsection}{0}\relax 
\contentsline {subsection}{\numberline {6.3.12}Gesamtübersicht}{189}{subsection.6.3.12}
\defcounter {refsection}{0}\relax 
\contentsline {chapter}{\numberline {7}Morphologie}{195}{chapter.7}
\defcounter {refsection}{0}\relax 
\contentsline {section}{\numberline {7.1}Formen und ihre Struktur}{195}{section.7.1}
\defcounter {refsection}{0}\relax 
\contentsline {subsection}{\numberline {7.1.1}Form und Funktion}{195}{subsection.7.1.1}
\defcounter {refsection}{0}\relax 
\contentsline {subsection}{\numberline {7.1.2}Morphe}{198}{subsection.7.1.2}
\defcounter {refsection}{0}\relax 
\contentsline {subsection}{\numberline {7.1.3}Wörter, Wortformen und Stämme}{200}{subsection.7.1.3}
\defcounter {refsection}{0}\relax 
\contentsline {subsection}{\numberline {7.1.4}Umlaut und Ablaut}{202}{subsection.7.1.4}
\defcounter {refsection}{0}\relax 
\contentsline {section}{\numberline {7.2}Morphologische Strukturen}{205}{section.7.2}
\defcounter {refsection}{0}\relax 
\contentsline {subsection}{\numberline {7.2.1}Lineare Beschreibung}{205}{subsection.7.2.1}
\defcounter {refsection}{0}\relax 
\contentsline {subsection}{\numberline {7.2.2}Strukturformat}{206}{subsection.7.2.2}
\defcounter {refsection}{0}\relax 
\contentsline {section}{\numberline {7.3}Flexion und Wortbildung}{208}{section.7.3}
\defcounter {refsection}{0}\relax 
\contentsline {subsection}{\numberline {7.3.1}Statische Merkmale}{208}{subsection.7.3.1}
\defcounter {refsection}{0}\relax 
\contentsline {subsection}{\numberline {7.3.2}Abgrenzung von Flexion und Wortbildung}{208}{subsection.7.3.2}
\defcounter {refsection}{0}\relax 
\contentsline {subsection}{\numberline {7.3.3}Lexikonregeln}{212}{subsection.7.3.3}
\defcounter {refsection}{0}\relax 
\contentsline {chapter}{\numberline {8}Wortbildung}{221}{chapter.8}
\defcounter {refsection}{0}\relax 
\contentsline {section}{\numberline {8.1}Komposition}{221}{section.8.1}
\defcounter {refsection}{0}\relax 
\contentsline {subsection}{\numberline {8.1.1}Definition und Überblick}{221}{subsection.8.1.1}
\defcounter {refsection}{0}\relax 
\contentsline {subsection}{\numberline {8.1.2}Kompositionstypen}{224}{subsection.8.1.2}
\defcounter {refsection}{0}\relax 
\contentsline {subsection}{\numberline {8.1.3}Rekursion}{226}{subsection.8.1.3}
\defcounter {refsection}{0}\relax 
\contentsline {subsection}{\numberline {8.1.4}Kompositionsfugen}{228}{subsection.8.1.4}
\defcounter {refsection}{0}\relax 
\contentsline {section}{\numberline {8.2}Konversion}{232}{section.8.2}
\defcounter {refsection}{0}\relax 
\contentsline {subsection}{\numberline {8.2.1}Konversionsphänomene}{232}{subsection.8.2.1}
\defcounter {refsection}{0}\relax 
\contentsline {subsection}{\numberline {8.2.2}Konversion im Deutschen}{234}{subsection.8.2.2}
\defcounter {refsection}{0}\relax 
\contentsline {section}{\numberline {8.3}Derivation}{236}{section.8.3}
\defcounter {refsection}{0}\relax 
\contentsline {subsection}{\numberline {8.3.1}Derivationsphänomene}{236}{subsection.8.3.1}
\defcounter {refsection}{0}\relax 
\contentsline {subsection}{\numberline {8.3.2}Derivation ohne Wortklassenwechsel}{237}{subsection.8.3.2}
\defcounter {refsection}{0}\relax 
\contentsline {subsection}{\numberline {8.3.3}Derivation mit Wortklassenwechsel}{240}{subsection.8.3.3}
\defcounter {refsection}{0}\relax 
\contentsline {chapter}{\numberline {9}Nominalflexion}{247}{chapter.9}
\defcounter {refsection}{0}\relax 
\contentsline {section}{\numberline {9.1}Nominale Flexionskategorien}{248}{section.9.1}
\defcounter {refsection}{0}\relax 
\contentsline {subsection}{\numberline {9.1.1}Numerus}{248}{subsection.9.1.1}
\defcounter {refsection}{0}\relax 
\contentsline {subsection}{\numberline {9.1.2}Kasus}{250}{subsection.9.1.2}
\defcounter {refsection}{0}\relax 
\contentsline {subsection}{\numberline {9.1.3}Person}{253}{subsection.9.1.3}
\defcounter {refsection}{0}\relax 
\contentsline {subsection}{\numberline {9.1.4}Genus}{256}{subsection.9.1.4}
\defcounter {refsection}{0}\relax 
\contentsline {subsection}{\numberline {9.1.5}Die nominalen Merkmale im Überblick}{256}{subsection.9.1.5}
\defcounter {refsection}{0}\relax 
\contentsline {section}{\numberline {9.2}Flexion der Substantive}{257}{section.9.2}
\defcounter {refsection}{0}\relax 
\contentsline {subsection}{\numberline {9.2.1}Traditionelle Flexionsklassen}{257}{subsection.9.2.1}
\defcounter {refsection}{0}\relax 
\contentsline {subsection}{\numberline {9.2.2}Numerusflexion}{259}{subsection.9.2.2}
\defcounter {refsection}{0}\relax 
\contentsline {subsection}{\numberline {9.2.3}Kasusflexion}{262}{subsection.9.2.3}
\enlargethispage{\baselineskip}
\defcounter {refsection}{0}\relax 
\contentsline {subsection}{\numberline {9.2.4}Schwache Substantive}{264}{subsection.9.2.4}
\defcounter {refsection}{0}\relax 
\contentsline {subsection}{\numberline {9.2.5}Revidiertes Klassensystem}{266}{subsection.9.2.5}
\defcounter {refsection}{0}\relax 
\contentsline {section}{\numberline {9.3}Flexion der Artikel und Pronomina}{267}{section.9.3}
\defcounter {refsection}{0}\relax 
\contentsline {subsection}{\numberline {9.3.1}Gemeinsamkeiten und Unterschiede}{267}{subsection.9.3.1}
\defcounter {refsection}{0}\relax 
\contentsline {subsection}{\numberline {9.3.2}Übersicht über die Flexionsmuster}{271}{subsection.9.3.2}
\defcounter {refsection}{0}\relax 
\contentsline {subsection}{\numberline {9.3.3}Pronomina und definite Artikel}{272}{subsection.9.3.3}
\defcounter {refsection}{0}\relax 
\contentsline {subsection}{\numberline {9.3.4}Indefinite Artikel und Possessivartikel}{275}{subsection.9.3.4}
\defcounter {refsection}{0}\relax 
\contentsline {section}{\numberline {9.4}Flexion der Adjektive}{276}{section.9.4}
\defcounter {refsection}{0}\relax 
\contentsline {subsection}{\numberline {9.4.1}Klassifikation}{276}{subsection.9.4.1}
\defcounter {refsection}{0}\relax 
\contentsline {subsection}{\numberline {9.4.2}Nominale Flexion}{278}{subsection.9.4.2}
\defcounter {refsection}{0}\relax 
\contentsline {subsection}{\numberline {9.4.3}Komparation}{281}{subsection.9.4.3}
\defcounter {refsection}{0}\relax 
\contentsline {chapter}{\numberline {10}Verbalflexion}{287}{chapter.10}
\defcounter {refsection}{0}\relax 
\contentsline {section}{\numberline {10.1}Verbale Flexionskategorien}{287}{section.10.1}
\defcounter {refsection}{0}\relax 
\contentsline {subsection}{\numberline {10.1.1}Person und Numerus}{287}{subsection.10.1.1}
\defcounter {refsection}{0}\relax 
\contentsline {subsection}{\numberline {10.1.2}Tempus}{288}{subsection.10.1.2}
\defcounter {refsection}{0}\relax 
\contentsline {subsection}{\numberline {10.1.3}Tempusformen}{293}{subsection.10.1.3}
\defcounter {refsection}{0}\relax 
\contentsline {subsection}{\numberline {10.1.4}Modus}{294}{subsection.10.1.4}
\defcounter {refsection}{0}\relax 
\contentsline {subsection}{\numberline {10.1.5}Finitheit und Infinitheit}{296}{subsection.10.1.5}
\defcounter {refsection}{0}\relax 
\contentsline {subsection}{\numberline {10.1.6}Genus verbi}{298}{subsection.10.1.6}
\defcounter {refsection}{0}\relax 
\contentsline {subsection}{\numberline {10.1.7}Die verbalen Merkmale im Überblick}{299}{subsection.10.1.7}
\defcounter {refsection}{0}\relax 
\contentsline {section}{\numberline {10.2}Verbale Flexion}{300}{section.10.2}
\defcounter {refsection}{0}\relax 
\contentsline {subsection}{\numberline {10.2.1}Unterklassen}{300}{subsection.10.2.1}
\defcounter {refsection}{0}\relax 
\contentsline {subsection}{\numberline {10.2.2}Tempus, Numerus und Person}{303}{subsection.10.2.2}
\defcounter {refsection}{0}\relax 
\contentsline {subsection}{\numberline {10.2.3}Konjunktiv}{305}{subsection.10.2.3}
\defcounter {refsection}{0}\relax 
\contentsline {subsection}{\numberline {10.2.4}Zur Schwa-Tilgung}{307}{subsection.10.2.4}
\defcounter {refsection}{0}\relax 
\contentsline {subsection}{\numberline {10.2.5}Infinite Formen}{308}{subsection.10.2.5}
\defcounter {refsection}{0}\relax 
\contentsline {subsection}{\numberline {10.2.6}Formen des Imperativs}{309}{subsection.10.2.6}
\defcounter {refsection}{0}\relax 
\contentsline {subsection}{\numberline {10.2.7}Kleine Verbklassen}{311}{subsection.10.2.7}
\vspace{\baselineskip}
\defcounter {refsection}{0}\relax 
\contentsline {part}{\numberline {IV}Syntax}{321}{part.4}
\defcounter {refsection}{0}\relax 
\contentsline {chapter}{\numberline {11}Konstituentenstruktur}{323}{chapter.11}
\defcounter {refsection}{0}\relax 
\contentsline {section}{\numberline {11.1}Syntaktische Struktur}{323}{section.11.1}
\defcounter {refsection}{0}\relax 
\contentsline {section}{\numberline {11.2}Konstituenten}{329}{section.11.2}
\defcounter {refsection}{0}\relax 
\contentsline {subsection}{\numberline {11.2.1}Konstituententests}{330}{subsection.11.2.1}
\defcounter {refsection}{0}\relax 
\contentsline {subsection}{\numberline {11.2.2}Konstituenten und Satzglieder}{334}{subsection.11.2.2}
\defcounter {refsection}{0}\relax 
\contentsline {subsection}{\numberline {11.2.3}Strukturelle Ambiguität}{336}{subsection.11.2.3}
\defcounter {refsection}{0}\relax 
\contentsline {section}{\numberline {11.3}Analysen von Konstituentenstrukturen}{337}{section.11.3}
\defcounter {refsection}{0}\relax 
\contentsline {subsection}{\numberline {11.3.1}Terminologie für Baumdiagramme}{337}{subsection.11.3.1}
\defcounter {refsection}{0}\relax 
\contentsline {subsection}{\numberline {11.3.2}Phrasenschemata}{339}{subsection.11.3.2}
\defcounter {refsection}{0}\relax 
\contentsline {subsection}{\numberline {11.3.3}Phrasen, Köpfe und Merkmale}{340}{subsection.11.3.3}
\newpage
\defcounter {refsection}{0}\relax 
\contentsline {chapter}{\numberline {12}Phrasen}{347}{chapter.12}
\defcounter {refsection}{0}\relax 
\contentsline {section}{\numberline {12.1}Bäume und Klammern}{347}{section.12.1}
\defcounter {refsection}{0}\relax 
\contentsline {section}{\numberline {12.2}Koordination}{348}{section.12.2}
\defcounter {refsection}{0}\relax 
\contentsline {section}{\numberline {12.3}Nominalphrase}{350}{section.12.3}
\defcounter {refsection}{0}\relax 
\contentsline {subsection}{\numberline {12.3.1}Die Struktur der NP}{350}{subsection.12.3.1}
\defcounter {refsection}{0}\relax 
\contentsline {subsection}{\numberline {12.3.2}Innere Rechtsattribute}{353}{subsection.12.3.2}
\defcounter {refsection}{0}\relax 
\contentsline {subsection}{\numberline {12.3.3}Rektion und Valenz in der NP}{354}{subsection.12.3.3}
\defcounter {refsection}{0}\relax 
\contentsline {subsection}{\numberline {12.3.4}Adjektivphrasen und Artikelwörter in der NP}{357}{subsection.12.3.4}
\defcounter {refsection}{0}\relax 
\contentsline {section}{\numberline {12.4}Adjektivphrase}{360}{section.12.4}
\defcounter {refsection}{0}\relax 
\contentsline {section}{\numberline {12.5}Präpositionalphrase}{364}{section.12.5}
\defcounter {refsection}{0}\relax 
\contentsline {subsection}{\numberline {12.5.1}Normale PP}{364}{subsection.12.5.1}
\defcounter {refsection}{0}\relax 
\contentsline {subsection}{\numberline {12.5.2}PP mit flektierbaren Präpositionen}{365}{subsection.12.5.2}
\defcounter {refsection}{0}\relax 
\contentsline {section}{\numberline {12.6}Adverbphrase}{366}{section.12.6}
\defcounter {refsection}{0}\relax 
\contentsline {section}{\numberline {12.7}Komplementiererphrase}{367}{section.12.7}
\defcounter {refsection}{0}\relax 
\contentsline {section}{\numberline {12.8}Verbphrase und Verbkomplex}{369}{section.12.8}
\defcounter {refsection}{0}\relax 
\contentsline {subsection}{\numberline {12.8.1}Verbphrase}{369}{subsection.12.8.1}
\defcounter {refsection}{0}\relax 
\contentsline {subsection}{\numberline {12.8.2}Verbkomplex}{372}{subsection.12.8.2}
\defcounter {refsection}{0}\relax 
\contentsline {section}{\numberline {12.9}Konstruktion von Konstituentenanalysen}{376}{section.12.9}
\defcounter {refsection}{0}\relax 
\contentsline {chapter}{\numberline {13}Sätze}{383}{chapter.13}
\defcounter {refsection}{0}\relax 
\contentsline {section}{\numberline {13.1}Hauptsatz und Matrixsatz}{383}{section.13.1}
\defcounter {refsection}{0}\relax 
\contentsline {subsection}{\numberline {13.1.1}Formale Grundbegriffe für Satzstrukturen}{383}{subsection.13.1.1}
\defcounter {refsection}{0}\relax 
\contentsline {subsection}{\numberline {13.1.2}Funktionen von satzartigen Konstituenten}{385}{subsection.13.1.2}
\defcounter {refsection}{0}\relax 
\contentsline {subsection}{\numberline {13.1.3}Funktionale Unterschiede zwischen Nebensatztypen}{387}{subsection.13.1.3}
\defcounter {refsection}{0}\relax 
\contentsline {section}{\numberline {13.2}Konstituentenstellung und Feldermodell}{391}{section.13.2}
\defcounter {refsection}{0}\relax 
\contentsline {subsection}{\numberline {13.2.1}Konstituentenstellung in unabhängigen Sätzen}{391}{subsection.13.2.1}
\defcounter {refsection}{0}\relax 
\contentsline {subsection}{\numberline {13.2.2}Das Feldermodell}{393}{subsection.13.2.2}
\defcounter {refsection}{0}\relax 
\contentsline {subsection}{\numberline {13.2.3}Eingebettete Nebensätze und der LSK-Test}{399}{subsection.13.2.3}
\defcounter {refsection}{0}\relax 
\contentsline {section}{\numberline {13.3}Schemata für Sätze}{401}{section.13.3}
\defcounter {refsection}{0}\relax 
\contentsline {subsection}{\numberline {13.3.1}Verb-Zweit-Sätze}{401}{subsection.13.3.1}
\defcounter {refsection}{0}\relax 
\contentsline {subsection}{\numberline {13.3.2}Verb-Erst-Sätze}{404}{subsection.13.3.2}
\defcounter {refsection}{0}\relax 
\contentsline {subsection}{\numberline {13.3.3}Syntax der Partikelverben}{406}{subsection.13.3.3}
\defcounter {refsection}{0}\relax 
\contentsline {subsection}{\numberline {13.3.4}Kopulasätze}{406}{subsection.13.3.4}
\defcounter {refsection}{0}\relax 
\contentsline {section}{\numberline {13.4}Nebensätze}{408}{section.13.4}
\defcounter {refsection}{0}\relax 
\contentsline {subsection}{\numberline {13.4.1}Relativsätze}{408}{subsection.13.4.1}
\defcounter {refsection}{0}\relax 
\contentsline {subsection}{\numberline {13.4.2}Komplementsätze}{413}{subsection.13.4.2}
\defcounter {refsection}{0}\relax 
\contentsline {subsection}{\numberline {13.4.3}Adverbialsätze}{416}{subsection.13.4.3}
\defcounter {refsection}{0}\relax 
\contentsline {chapter}{\numberline {14}Relationen und Prädikate}{421}{chapter.14}
\defcounter {refsection}{0}\relax 
\contentsline {section}{\numberline {14.1}Semantische Rollen}{421}{section.14.1}
\defcounter {refsection}{0}\relax 
\contentsline {subsection}{\numberline {14.1.1}Verbsemantik und Rollen}{421}{subsection.14.1.1}
\defcounter {refsection}{0}\relax 
\contentsline {subsection}{\numberline {14.1.2}Semantische Rollen und Valenz}{424}{subsection.14.1.2}
\defcounter {refsection}{0}\relax 
\contentsline {section}{\numberline {14.2}Prädikate und prädikative Konstituenten}{425}{section.14.2}
\enlargethispage{\baselineskip}
\defcounter {refsection}{0}\relax 
\contentsline {subsection}{\numberline {14.2.1}Das Prädikat}{425}{subsection.14.2.1}
\defcounter {refsection}{0}\relax 
\contentsline {subsection}{\numberline {14.2.2}Prädikative}{427}{subsection.14.2.2}
\defcounter {refsection}{0}\relax 
\contentsline {section}{\numberline {14.3}Subjekte}{429}{section.14.3}
\defcounter {refsection}{0}\relax 
\contentsline {subsection}{\numberline {14.3.1}Subjekte als Nominativ-Ergänzungen}{429}{subsection.14.3.1}
\defcounter {refsection}{0}\relax 
\contentsline {subsection}{\numberline {14.3.2}Arten von \textit {es} im Nominativ}{432}{subsection.14.3.2}
\defcounter {refsection}{0}\relax 
\contentsline {section}{\numberline {14.4}Passiv}{436}{section.14.4}
\defcounter {refsection}{0}\relax 
\contentsline {subsection}{\numberline {14.4.1}\textit {werden}-Passiv und Verbtypen}{436}{subsection.14.4.1}
\defcounter {refsection}{0}\relax 
\contentsline {subsection}{\numberline {14.4.2}\textit {bekommen}-Passiv}{439}{subsection.14.4.2}
\defcounter {refsection}{0}\relax 
\contentsline {section}{\numberline {14.5}Objekte, Ergänzungen und Angaben}{441}{section.14.5}
\defcounter {refsection}{0}\relax 
\contentsline {subsection}{\numberline {14.5.1}Akkusative und direkte Objekte}{441}{subsection.14.5.1}
\defcounter {refsection}{0}\relax 
\contentsline {subsection}{\numberline {14.5.2}Dative und indirekte Objekte}{442}{subsection.14.5.2}
\defcounter {refsection}{0}\relax 
\contentsline {subsection}{\numberline {14.5.3}PP-Ergänzungen und PP-Angaben}{445}{subsection.14.5.3}
\defcounter {refsection}{0}\relax 
\contentsline {section}{\numberline {14.6}Bindung}{447}{section.14.6}
\defcounter {refsection}{0}\relax 
\contentsline {section}{\numberline {14.7}Analytische Tempora}{449}{section.14.7}
\defcounter {refsection}{0}\relax 
\contentsline {section}{\numberline {14.8}Modalverben und Ähnliches}{453}{section.14.8}
\defcounter {refsection}{0}\relax 
\contentsline {subsection}{\numberline {14.8.1}Ersatzinfinitiv und Oberfeldumstellung}{453}{subsection.14.8.1}
\defcounter {refsection}{0}\relax 
\contentsline {subsection}{\numberline {14.8.2}Kohärenz}{454}{subsection.14.8.2}
\defcounter {refsection}{0}\relax 
\contentsline {subsection}{\numberline {14.8.3}Modalverben und Halbmodalverben}{457}{subsection.14.8.3}
\defcounter {refsection}{0}\relax 
\contentsline {section}{\numberline {14.9}Infinitivkontrolle}{459}{section.14.9}
\defcounter {refsection}{0}\relax 
\contentsline {part}{\numberline {V}Graphematik}{467}{part.5}
\defcounter {refsection}{0}\relax 
\contentsline {chapter}{\numberline {15}Phonologische Schreibprinzipien}{469}{chapter.15}
\defcounter {refsection}{0}\relax 
\contentsline {section}{\numberline {15.1}Status der Graphematik}{469}{section.15.1}
\defcounter {refsection}{0}\relax 
\contentsline {subsection}{\numberline {15.1.1}Graphematik als Teil der Grammatik}{469}{subsection.15.1.1}
\defcounter {refsection}{0}\relax 
\contentsline {subsection}{\numberline {15.1.2}Ziele und Vorgehensweise}{473}{subsection.15.1.2}
\defcounter {refsection}{0}\relax 
\contentsline {section}{\numberline {15.2}Buchstaben und phonologische Segmente}{475}{section.15.2}
\defcounter {refsection}{0}\relax 
\contentsline {subsection}{\numberline {15.2.1}Konsonantenschreibungen}{475}{subsection.15.2.1}
\defcounter {refsection}{0}\relax 
\contentsline {subsection}{\numberline {15.2.2}Vokalschreibungen}{478}{subsection.15.2.2}
\defcounter {refsection}{0}\relax 
\contentsline {section}{\numberline {15.3}Graphematik der Silben und Wörter}{480}{section.15.3}
\defcounter {refsection}{0}\relax 
\contentsline {subsection}{\numberline {15.3.1}Dehnungs- und Schärfungsschreibungen}{480}{subsection.15.3.1}
\defcounter {refsection}{0}\relax 
\contentsline {subsection}{\numberline {15.3.2}Eszett an der Silbengrenze}{483}{subsection.15.3.2}
\defcounter {refsection}{0}\relax 
\contentsline {subsection}{\numberline {15.3.3}\textit {h} zwischen Vokalen}{487}{subsection.15.3.3}
\defcounter {refsection}{0}\relax 
\contentsline {section}{\numberline {15.4}Betonung und Hervorhebung}{488}{section.15.4}
\defcounter {refsection}{0}\relax 
\contentsline {section}{\numberline {15.5}Ausblick auf den Nicht-Kernwortschatz}{489}{section.15.5}
\defcounter {refsection}{0}\relax 
\contentsline {chapter}{\numberline {16}Morphosyntaktische Schreibprinzipien}{495}{chapter.16}
\defcounter {refsection}{0}\relax 
\contentsline {section}{\numberline {16.1}Wortbezogene Schreibungen}{495}{section.16.1}
\defcounter {refsection}{0}\relax 
\contentsline {subsection}{\numberline {16.1.1}Wörter und Spatien}{495}{subsection.16.1.1}
\defcounter {refsection}{0}\relax 
\contentsline {subsection}{\numberline {16.1.2}Die Substantivgroßschreibung als wortklassenbezogene Schreibung}{496}{subsection.16.1.2}
\defcounter {refsection}{0}\relax 
\contentsline {subsection}{\numberline {16.1.3}Graphematik der Wortbildung}{500}{subsection.16.1.3}
\defcounter {refsection}{0}\relax 
\contentsline {subsection}{\numberline {16.1.4}Abkürzungen und Auslassungen}{502}{subsection.16.1.4}
\defcounter {refsection}{0}\relax 
\contentsline {subsection}{\numberline {16.1.5}Konstantschreibungen}{505}{subsection.16.1.5}
\newpage
\defcounter {refsection}{0}\relax 
\contentsline {section}{\numberline {16.2}Schreibung von Phrasen und Sätzen}{507}{section.16.2}
\defcounter {refsection}{0}\relax 
\contentsline {subsection}{\numberline {16.2.1}Graphematik der Phrasen}{507}{subsection.16.2.1}
\defcounter {refsection}{0}\relax 
\contentsline {subsection}{\numberline {16.2.2}Graphematik unabhängiger Sätze}{508}{subsection.16.2.2}
\defcounter {refsection}{0}\relax 
\contentsline {subsection}{\numberline {16.2.3}Graphematik von Nebensätzen und Verwandtem}{512}{subsection.16.2.3}
\defcounter {refsection}{0}\relax 
\contentsline {chapter}{Lösungen zu den Übungen}{517}{section*.240}
\defcounter {refsection}{0}\relax 
\contentsline {chapter}{\nonumberline Literatur}{561}{chapter*.319}
\defcounter {refsection}{0}\relax 
\contentsline {chapter}{\nonumberline Index}{569}{chapter*.321}
%\defcounter {refsection}{0}\relax 
%\contentsline {chapter}{Index}{569}{section*.320}
