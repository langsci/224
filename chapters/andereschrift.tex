\chapter{Morphosyntaktische Schreibprinzipien}

\label{sec:andereschrift}

\section{Wortbezogene Schreibungen}

\subsection{Wörter}

\label{sec:spatien}
\index{Spatium}

In diesem Kapitel geht es überblickshaft um Schreibprinzipien, die funktional auf der Wortebene und der Satzebene angesiedelt sind.
Das wahrscheinlich wichtigste Prinzip, das vielleicht so selbstverständlich erscheint, dass wir es leicht aus dem Auge verlieren, ist, dass wir syntaktische Wörter (Wortformen) in der Schrift durch \textit{Leerzeichen} (\textit{Spatien}) trennen, vgl.\ Satz~\ref{satz:spatien}.

\Satz{Prinzip der Spatienschreibung}{\label{satz:spatien}%
Syntaktische Wörter werden durch Spatien getrennt.
\index{Schreibprinzip!Spatienschreibung}
\index{Wort!graphematisch}
}

Dieses Prinzip galt historisch im Deutschen nicht immer, und auch viele moderne Sprachen kommen ohne Worttrennung durch Spatien aus (\zB Chinesisch und Japanisch).
Ein Beispiel wie (\ref{ex:msschr001}) zeigt, dass beim Verzicht auf die Spatien die Lesbarkeit nicht völlig zusammenbricht.
Trotzdem ist das Spatium ohne Zweifel eine wichtige Lesehilfe.
Offensichtlich ist, dass Elemente der Wortbildung (\ref{ex:msschr002a}) und Flexion (\ref{ex:msschr002b}) nicht getrennt werden.

\begin{exe}
  \ex{\label{ex:msschr001} SiekönneneinenSatzohneSpatienwahrscheinlichtrotzdemlesen.}
  \ex\label{ex:msschr002} 
  \begin{xlist}
    \ex[*]{\label{ex:msschr002a} Vanessa hat Gelegen heit, die Schreib ung von Wörtern und Sätzen gründ lich zu unter suchen.}
    \ex[*]{\label{ex:msschr002b} Oma koch t der ausgekühlt en Vanessa ein en heiß en Tee.}
  \end{xlist}
\end{exe}

Andererseits werden Wörter nicht einfach so zusammengeschrieben, \zB weil sie zusammen eine analytische Tempusform ergeben (\ref{ex:msschr003a}) oder eine Phrase bilden (\ref{ex:msschr003b}).

\begin{exe}
  \ex\label{ex:msschr003} 
  \begin{xlist}
    \ex[*]{\label{ex:msschr003a} Vanessa istgeritten.}
    \ex[*]{\label{ex:msschr003b} Vanessa reitet indenwald.}
  \end{xlist}
\end{exe}

Die Wörter \textit{ist} und \textit{geritten} behandeln wir \zB deshalb als getrennte syntaktische Wörter, weil sie durch einfache Umstellungen im Satz voneinander getrennt werden können.
Außerdem haben beide eine klar erkennbare Valenz (\textit{ist} verlangt ein Verb im dritten Status, \textit{geritten} eine NP im Nominativ).
Vor allem kann aber \textit{geritten} in diesem Satz durch eine sehr große Menge anderer intransitiver Verben ersetzt werden, die das Perfekt mit \textit{sein} bilden.
Außerdem hat \textit{geritten} hier seine ganz normale Bedeutung.
Anders gesagt verhalten sich die Wörter erkennbar autonom und haben je eigene Funktionen, Bedeutungen und grammatische Eigenschaften, weswegen wir sie als syntaktische Wörter bezeichnen (vgl.\ Abschnitt~\ref{sec:woerterwortformen}).
Genau deswegen können sie eben auch nicht zusammengeschrieben werden.%
\footnote{Man könnte zusätzlich noch die Akzentverhältnisse bemühen, da Wörter typischerweise genau einen Hauptakzent haben (vgl.\ Abschnitt~\ref{sec:wortakzentimdeutschen}).}

Das Prinzip aus Satz~\ref{satz:spatien} ist einfach, klar und scheinbar unverfänglich.
Trotzdem gibt es charakteristische Schwierigkeiten in der Normierung (und sekundär in der Orthographievermittlung) bezüglich der Zusammen- und Getrenntschreibung.
Abfolgen von Wörtern, die sich im Gebrauch stark aneinander binden und dabei ihren semantischen Gehalt teilweise verlieren bzw.\ zu grammatischen Funktionswörtern werden, tendieren dazu, zu einfachen Wörtern zu verschmelzen und damit auch zusammengeschrieben zu werden.
Dieser Prozess wird \textit{Univerbierung} genannt, und er wurde bereits in Vertiefung~\ref{vert:rueckbildunguniverbierung} auf S.~\pageref{vert:rueckbildunguniverbierung} diskutiert.\index{Univerbierung}
Typisch für das Deutsche sind \zB Bildungen von sekundären Präpositionen wie \textit{anstatt} oder \textit{zulasten}.
Klassische Zweifelsfälle entstehen auch bei der potentiellen Bildung neuer Verbpartikeln aus Adjektiven wie \textit{weichklopfen} oder \textit{schlechtreden} und im Bereich der Adjektive durch Verschmelzung mit einer vorangestellten Partikel wie im Fall von \textit{nichtöffentlich} statt \textit{nicht öffentlich}.
In Vertiefung~\ref{vert:rueckbildunguniverbierung} wurde bereits argumentiert, dass es sich um sprachgeschichtlich relevante \textit{Grammatikalisierungen}, nicht aber um produktive Prozesse handelt.\index{Grammatikalisierung}
Die Zusammenschreibung sollte also umso häufiger werden, je weniger die Wortgruppe noch semantisch transparent ist, und je mehr die beteiligten Wörter als ein prosodisches Wort mit einem Hauptakzent auftreten.
Die Normierung setzt also zu spät an, wenn sie sich nur auf die Schreibung bezieht.
Vielmehr wird Schreibern mit einer Regelung, die \zB ausschließlich \textit{zu Lasten} oder \textit{zulasten} erlaubt, statt einer \textit{orthographischen} Form eine bestimmte \textit{morphologische} Form vorgeschrieben.
Auch wenn man für die verschiedenen Typen von Univerbierungen gute heuristische Tests ansetzen kann, die die eine oder andere Variante sinnvoller erscheinen lassen, sind wahrscheinlich häufig beide Varianten plausibel und der Streitfall vorprogrammiert.
Eine unstrittige Normierung im Sinne einer Empfehlung für den unauffälligen Sprachgebrauch (s.\ Abschnitt~\ref{sec:normalsbeschreibung}) könnte einzig auf Basis der Verwendungshäufigkeiten erfolgen.

\subsection{Wortklassen}

\label{sec:wortklassschreib}
\index{Wortklasse!Schreibung}

Nachdem wir jetzt festgestellt haben, dass Wörter in der Schreibung dadurch identifiziert werden, dass sie durch Spatien getrennt sind, werden ausgewählte Phänomene auf Wortebene betrachtet.
Ein wichtiges Merkmal jedes Wortes ist seine Klassenzugehörigkeit.
Im Normalfall gibt es keine direkte Markierung der Wortklasse in der Schreibung.
Natürlich sind Schreibungen bestimmter Wörter gut geeignet, die Klassenzugehörigkeit der Wörter anzuzeigen, wie \zB Wörter, die auf \textit{ig} oder \textit{keit} enden.
Solche Situationen kommen nur auf Umwegen zustande.
Die Buchstaben kodieren Segmente, die zusammen ein Wortbildungssuffix ergeben, welches wiederum ein Indikator für eine Wortklasse ist, weil es ein wortartveränderndes Suffix ist (s.\ Abschnitt~\ref{sec:deriv}).
Das hat mit einer spezifisch graphematischen Markierung nichts zu tun.

Neben vereinzelten Markierungen von Wortklassenunterschieden durch Varianten der Schreibung (Komplementierer \textit{dass} und Artikel bzw.\ Pronomen \textit{das}) gibt es nur eine für das Deutsche charakteristische systematische Wortklassenmarkierung, nämlich die Großschreibung von Eigennamen und Substantiven.
Wie in vielen anderen Sprachen werden \textit{Eigennamen} in allen Position großgeschrieben.
Die \textit{Substantivgroßschreibung} ist eine Besonderheit des Deutschen, vgl.\ Satz~\ref{satz:grosschreib}.

\Satz{Positionsunabhängige Majuskelschreibung}{\label{satz:grosschreib}%
Eigennamen und Substantive werden unabhängig von ihrer Position immer mit einleitender Majuskel geschrieben.
\index{Majuskel}
}

Der Kern der Substantivgroßschreibung ist völlig unproblematisch.
Was ein Substantiv ist, wird immer (auch mitten im Satz oder in der \textit{Nennform}, in Listen, in Wörterbüchern usw.) großgeschrieben.
Gemäß Filter~\ref{wfilt:subst} auf S.~\pageref{wfilt:subst} wissen wir genau, was ein Substantiv ist, nämlich ein Nomen mit festem Genus.
Gemäß der davorstehenden Filter~\ref{wfilt:flektierbare} und \ref{wfilt:verbennomina} sind Nomina solche Wörter, die für Numerus spezifiziert sind, aber nicht finit (also für Tempus spezifiziert) sein können.
Das beste diagnostische Kriterium dafür, ob die Substantivgroßschreibung greifen sollte oder nicht, ist also, ob ein genusspezifischer Artikel vor einem Wort steht oder stehen kann.
\index{Konversion}
\index{Substantivierung}
Problemfälle mit diesem Test treten vor allem bei Konversionen von Adjektiven auf, vgl.\ (\ref{ex:msschr006}).

\begin{exe}
  \ex\label{ex:msschr006} 
  \begin{xlist}
    \ex{\label{ex:msschr006c} An der Nacht auf dem Land schätze ich vor allem das Dunkle.}
    \ex{\label{ex:msschr006b} Alle Pferde müssen geputzt werden. Vanessa putzt das schwarze.}
    \ex{\label{ex:msschr006a} Vanessa trägt in der Oper das Schwarze.}
  \end{xlist}
\end{exe}

Satz (\ref{ex:msschr006c}) ist der ganz typische Fall eines substantivierten Adjektivs, weil sich hier auf das Dunkle an sich (die Eigenschaft, dunkel zu sein) bezogen wird.
In (\ref{ex:msschr006b}) liegt eine typische Situation für eine Ellipse vor, also eine Auslassung eines oder mehrerer Wörter in einem Folgesatz (hier das Substantiv \textit{Pferd}), die ansonsten eine Wiederholung darstellen würden.
Daher ist das Adjektiv nicht substantiviert und wird nicht großgeschrieben.
In (\ref{ex:msschr006a}) wird \textit{das Schwarze} als Bezeichnung für \textit{das schwarze Kleid} benutzt, und das Adjektiv ist daher substantiviert und wird groß geschrieben.
Allerdings ist (\ref{ex:msschr006a}) ein sehr plakativ gewähltes Beispiel, und der Übergang von Fällen wie (\ref{ex:msschr006b}) zu solchen wie (\ref{ex:msschr006a}) ist normalerweise alles andere als scharf.
Das Schreibsystem bietet hier zwei Möglichkeiten, genau zwei syntaktische Strukturen zu kodieren, und ein Regelungsbedarf auf Seiten der Orthographie besteht eigentlich nicht.
Man muss sich vielmehr bewusst sein, dass die meisten orthographischen Regelungen (wie schon bei den Univerbierungen) eigentlich grammatische Regelungen darstellen.

\index{Substantiv!Großschreibung}
Schwieriger zu entscheidende Fälle im Bereich der Großschreibung betreffen ehemalige Substantive wie in \textit{im übrigen}, \textit{recht geben} bzw.\ \textit{rechtgeben}, \textit{instand setzen} bzw.\ \textit{instandsetzen}, \textit{im trüben fischen}, die hier unabhängig von orthographischen Regeln der Illustration halber alle kleingeschrieben werden.
Hier wird Kleinschreibung und ggf.\ sogar Zusammenschreibung als Zeichen der Univerbierung sinnvoll, weil die Wortsequenzen ihre Eigenständigkeit und ursprüngliche Bedeutung verlieren, so dass man bei \textit{übrigen} und \textit{recht} usw.\ nicht mehr von Substantiven sprechen kann.
Hier gibt es Übergangsbereiche zwischen selbständigen Substantiven, festen Fügungen mit verblassten Substantiven und vollständigen Univerbierungen in Form von Adverben, Partikelverben usw.
Ein Kriterium, das uns entscheiden hilft, ob ein Substantiv noch voll produktiv als Substantiv verwendet wird, ist die morphosyntaktische Kombinierbarkeit des Wortes.
In \textit{im übrigen} kann \zB das potentielle Substantiv \textit{das Übrige} \zB nicht mehr modifiziert werden, vgl.\ (\ref{ex:msschr007}).

\begin{exe}
  \ex\label{ex:msschr007} 
  \begin{xlist}
    \ex[*]{\label{ex:msschr007a} im literarischen Übrigen}
    \ex[*]{\label{ex:msschr007b} Im Übrigen\slash In dem Übrigen, von dem wir gestern schon gesprochen haben, ist dieses Buch langweilig.}
  \end{xlist}
\end{exe}

Als alleiniges Argument greift die Modifizierbarkeit hier aber zu kurz.
Erstens steht vor \textit{übrigen} ein artikelwertiges Wort (s.\ Abschnitt~\ref{sec:syntaxflektierbareprp}), und Artikel können per Definition nur vor Substantiven stehen.
Neben seiner Artikelfunktion ist \textit{im} vor allem eine Präposition, die den Dativ zuweist, und die Form von \textit{übrigen} ist einwandfrei die eines substantivierten Adjektivs.
Insgesamt ist \textit{im übrigen} syntaktisch also nicht von einer normalen PP zu unterscheiden.
Zweitens handelt es sich bei \textit{im übrigen} um zwei phonologische Wörter, von einer Univerbierung kann also nicht gesprochen werden.
Sprecher scheinen keine Bestrebungen zu haben, \textit{im übrigen} so wie \textit{anstatt} phonologisch erkennbar zu univerbieren.
Drittens sind in vielen Redensarten und Idiomen Substantive de facto ebenso nicht modifizierbar, wie \zB in \textit{unter einer Decke stecken}, \textit{Schwein haben} usw.
In solchen Fügungen werden die Substantive aber auch nicht kleingeschrieben (*\textit{unter einer decke stecken}, *\textit{schwein haben}), nur weil sie ihre wörtliche Bedeutung verloren haben.

\index{Univerbierung}
Die Situation bei Verbpartikeln, die sich aus Substantiven oder PPs gebildet haben, ist eindeutiger, also \zB \textit{rechtgeben} und \textit{instandsetzen}.
Modifizierbarkeit ist nicht gegeben, denn weder Artikel und Adjektive noch Relativsätze können \textit{recht} und \textit{stand} noch modifizieren, s.\ (\ref{ex:msschr34657}).

\begin{exe}
  \ex\label{ex:msschr34657}
  \begin{xlist}
  	\ex[*]{\label{ex:msschr34657a} Edgar gab dem Kunden fachmännisches Recht.}
  	\ex[*]{\label{ex:msschr34657b} Edgar setzte den Cadillac in einwandfreien Stand.}
  \end{xlist}
\end{exe}

In \textit{rechtgeben} und \textit{instandsetzen} bilden alle Bestandteile zusammen (im Gegensatz zu \textit{im übrigen}) außerdem ein einziges prosodisches Wort, innerhalb dessen nur ein Hauptakzent zugewiesen wird, \zB \textit{in\Akz standsetzen} aus \textit{in\Akz stand} und \textit{\Akz setzen}.
Das ist ein guter Hinweis auf eine echte Univerbierung, und im Sinn von Kapitel~\ref{sec:wortbildung} können wir damit in der Analyse \textit{recht=geben} und \textit{instand=setzen} schreiben.
Die Kleinschreibung ist dann insofern plausibel, als im Rahmen der Univerbierung eine echte Verbpartikel entstanden ist, die wie alle Verbpartikeln kleingeschrieben wird.
Es kann natürlich nicht ausgeschlossen werden, dass Verfechter von Schreibungen wie \textit{in Stand setzen} in ihrer Varietät des Deutschen hier tatsächlich drei prosodische Wörter realisieren und deswegen zur Großschreibung von \textit{Stand} und damit auch zur Getrenntschreibung der drei Wörter tendieren.
Manchen Sprechern erscheinen Beispiele wie (\ref{ex:msschr34657}) auch umso akzeptabler, je öfter sie sie hören bzw.\ je länger sie darauf starren.
Formale Kriterien helfen hier also nur begrenzt weiter und über semantische ist oft nur schwer Einigkeit zu erzielen.
Normierungsinitiativen wären gut beraten, einen statistischen Ansatz zu verfolgen und den \textit{üblichen} (also \textit{häufigen}) Gebrauch dieser Einheiten einschließlich ihrer \textit{üblichen} Schreibung zu ermitteln, darauf aufbauend \textit{Empfehlungen} statt verbindlicher Regelungen auszusprechen.

\index{Eigenname!Schreibung}
Wir kommen zu den Eigennamen.
Für Personennamen wie \textit{Willy Brandt} ist Satz~\ref{satz:grosschreib} völlig unproblematisch, bei potentiellen Namen wie in (\ref{ex:msschr004}) kommen grammatisch, semantisch und damit auch oft graphematisch mehrere Möglichkeiten infrage.

\begin{exe}
  \ex\label{ex:msschr004} 
  \begin{xlist}
    \ex{\label{ex:msschr004a} der Deutsche Bundestag}
    \ex{\label{ex:msschr004b} eine Molekulare Ratsche}
    \ex{\label{ex:msschr004c} am Schwarzen Brett}
    \ex{\label{ex:msschr004d} Das ist Der Spiegel von letzter Woche.}
  \end{xlist}
\end{exe}

In diesen Fällen geht es vor allem darum, zu entscheiden, ob die jeweiligen Begriffe so speziell gelesen werden, dass sie Eigennamen darstellen.
Was genau ein Eigenname ist, ist allerdings (jenseits der Personennamen) nicht einfach zu entscheiden.
Als wichtigstes Kriterium gilt gemeinhin, dass mit einem Eigennamen immer genau ein Ding in der Welt bezeichnet wird, und nicht eine Sorte von Dingen.
Für den \textit{Deutschen Bundestag} ist dies der Fall, denn es gibt ja nicht mehrere deutsche Bundestage.
Bei der \textit{Molekularen Ratsche} ist dies nicht der Fall, denn Molekulare Ratschen könnte es mehrere geben, wenn sie jemand bauen würde.
Allerdings handelt es sich nicht um irgendwelche Ratschen, die irgendetwas mit Molekülen zu tun haben, sondern der Begriff \textit{Molekulare Ratsche} ist auf einen sehr spezifischen Apparat festgelegt.
Mit der Großschreibung wird hier ggf.\ diese Spezifizität verbunden, also die nicht vollständig kompositionelle Benennung einer recht spezifischen Sorte von Gegenständen (zum Kompositionalitätsbegriff s.\ Abschnitt~\ref{sec:sprachealssymbolsystem}).
Das Gleiche gilt für das \textit{Schwarze Brett}, wobei hier sogar die Transparenz der Bildung abhanden gekommen ist (s.\ Definition~\ref{def:transparenz} auf S.~\pageref{def:transparenz}), da die betreffenden Objekte nicht einmal schwarz und eigentlich auch kein Brett sein müssen.
Die wahrscheinlich zuverlässigste Normierung für alle diese Fälle wäre, bei echten Eigennamen mit singulärem Bezugsobjekt in der Welt die Großschreibung festzulegen (\textit{Deutscher Bundestag}) und alle anderen Adjektive prinzipiell kleinzuschreiben (\textit{schwarzes Brett}, \textit{molekulare Ratsche}).

Ein eher randständiges Problem sind Namen, bei denen der Artikel Teil des Namens ist wie in (\ref{ex:msschr004d}).
Hierbei ist es ungewöhnlich, den Artikel weiter großzuschreiben, wenn er flektiert.

\begin{exe}
  \ex\label{ex:msschr005} 
  \begin{xlist}
    \ex[?]{\label{ex:msschr005a} Ich lese Den Spiegel.}
    \ex[?]{\label{ex:msschr005b} Ich lese Der Spiegel.}
  \end{xlist}
\end{exe}

Während (\ref{ex:msschr005a}) wahrscheinlich von vielen Lesern als sehr auffällig empfunden wird, gibt es für (\ref{ex:msschr005b}) zumindest die Lesart, bei der \textit{Der Spiegel} ohne zu flektieren immer in der Zitatform verwendet wird.
Für dieses Problem gibt es vielleicht keine für alle Sprecher zufriedenstellende Lösung, was angesichts seiner Randständigkeit aber sicherlich auch nicht das gesamte System zum Einsturz bringt.


\subsection{Wortbildung}

\label{sec:wortbildschreib}

Die in Abschnitt~\ref{sec:wortklassschreib} besprochene Substantivgroßschreibung gehört eigentlich gleichzeitig in diesen Abschnitt, in dem es um Phänomene der Wortbildung geht, die einen Effekt in der Schreibung haben.
Konversion zum Substantiv äußert sich in der Großschreibung, \zB \textit{laufen} als Verb zu \textit{das Laufen} als Substantiv.
Im Bereich der Konversion und Derivation sind sonst keine besonderen Anmerkungen innerhalb der Graphematik zu machen.

\index{Kompositum!Schreibung}
Für die Komposition gilt dies nicht ganz, denn im Bereich der \textit{Normschreibung} und \textit{Gebrauchsschreibung} gibt es verschiedene Varianten für die Schreibung von Substantiv-Komposita, s.\ (\ref{ex:msschr008}).
\begin{exe}
  \ex\label{ex:msschr008} 
  \begin{xlist}
    \ex{\label{ex:msschr008a} Abendausritt}
    \ex{\label{ex:msschr008b} Abend-Ausritt}
    \ex{\label{ex:msschr008c} Abend Ausritt}
    \ex{\label{ex:msschr008d} AbendAusritt}
  \end{xlist}
\end{exe}

Neben der Zusammenschreibung (\ref{ex:msschr008a}) findet man die Schreibung mit Bindestrich (\ref{ex:msschr008b}), in der Gebrauchsschreibung aber auch die Spatienschreibung (\ref{ex:msschr008c}) und die Schreibung mit der sogenannten \textit{Binnenmajuskel} (\ref{ex:msschr008d}).
Da Komposita ein einziges Wort bilden (vgl.\ Abschnitt~\ref{sec:komp}), ist (\ref{ex:msschr008a}) im Rahmen des Spatienprinzips (Abschnitt~\ref{sec:spatien}) unstrittig.
Bei \textit{Abendausritt} handelt es sich um ein Wort, und dementsprechend enthält es keine Spatien.

\index{Bindestrich}
Für die Beschreibung von (\ref{ex:msschr008b}) muss zunächst der Bindestrich -- ein Nicht-Buchstabe -- an sich eingeordnet werden.
Er tritt auf in Komposita, als Silbentrennzeichen und in Koordinationen mit Ellipse wie \textit{Luft- und Raumfahrt}.
Es ist schwer, eine einheitliche Funktion für den Bindestrich zu finden, aber er kommt ganz offensichtlich nur im Bereich der Wortschreibung zum Einsatz (also nicht als Satzzeichen) und ist damit ein \textit{Wortzeichen}.\index{Wortzeichen}%
\footnote{Diese Aussage beruht auf einer unterstellten Trennung zwischen dem \textit{Bindestrich} - und dem hier gar nicht behandelten \textit{Gedankenstrich} --.
Der Gedankenstrich ist zumindest in Druckerzeugnissen länger und hat üblicherweise auch eine feinere Strichstärke.
Das typographische Minuszeichen $-$ ist zwar dem Gedankenstrich ähnlich, liegt aber auf einer anderen Höhe, weil es dem horizontalen Strich des Pluszeichens entspricht. 
Es wird vor allem deshalb selten zwischen diesen drei Zeichen unterschieden, weil handschriftlich die Unterschiede kaum zuverlässig markiert werden können, handelsübliche Schreibmaschinen für alle drei Zeichen nur eine Type hatten, und in Textverarbeitungen der Gedankenstrich und das Minus üblicherweise von Anwendern nicht gefunden werden.
Es ist also wahrscheinlich, dass hier für Schreiber ein Zeichen mit mehreren Funktionen vorliegt.}

\Definition{Wortzeichen}{\label{def:wortzeich}%
\textit{Wortzeichen} sind Nicht-Buchstaben, die im Bereich der Wortschreibungen verwendet werden.
\index{Zeichen!Wort--}
}

Aus Sicht der Grammatik kommt es infrage, den Bindestrich in Komposita als Markierung der Grenze zwischen den phonologischen Wörtern, die im Kompositum zusammen ein prosodisches Wort bilden, zu analysieren (s.\ Abschnitt~\ref{sec:prosodischewoerter}).
Dazu passt allerdings nicht, dass er nur sehr sporadisch in dieser Position eingesetzt wird, und dass es vermutlich ganz anders motivierte Faktoren gibt, die die Bindestrichschreibung begünstigen.
Die Länge und die Komplexität (Anzahl der Glieder) des Kompositums, Eigennamen- und Lehnwortbeteiligung (\textit{Brandt-Regierung} statt \textit{Brandtregierung} und \textit{Email-Ablage} statt \textit{Emailablage}), seine Häufigkeit bzw.\ die Produktivität seiner Bildung oder spezifische semantische Relationen zwischen seinen Gliedern spielen eine Rolle bei der Bindestrichsetzung.
Komposita wie \textit{Kindergarten} sind \zB nicht sonderlich produktiv gebildet, und hier scheint \textit{Kinder-Garten} ausgeschlossen.

\index{Majuskel}
\index{Spatium}
Die Varianten mit Spatium (\ref{ex:msschr008c}) und Binnenmajuskel (\ref{ex:msschr008d}) sind bezüglich ihrer Einordnung problematisch.
Die Schreibung mit Spatium verletzt das Prinzip der Spatienschreibung (Satz~\ref{satz:spatien}), denn Wörter wie \textit{Abendausritt} sind nach der hier vertretenen Grammatik genau ein syntaktisches Wort.
Wie in Abschnitt~\ref{sec:komp} gezeigt wurde, haben solche Komposita nämlich eine grammatische Merkmalsausstattung, \zB nur einen Kasus, ein Genus, ein Numerus.
Das Element, das nicht der Kopf ist, verliert seine grammatischen Merkmale und ist damit in der Syntax keine Einheit.
Auch wenn sie gelegentlich vorkommt, passt die Spatienschreibung von Komposita also eigentlich nicht ins System.
Angesichts der Tendenz des Deutschen, sehr lange Komposita zu bilden, ist allerdings auch fraglich, ob sich eine solche Schreibung jemals in größerem Ausmaß durchsetzen könnte, da sie sehr wahrscheinlich zu Ungunsten der Übersichtlichkeit gehen würde.
Die Schreibung mit Binnenmajuskel (\ref{ex:msschr008d}) ist sehr idiosynkratisch.
Sie verletzt die Prinzipien der positionsunabhängigen Majuskelschreibung (Satz~\ref{satz:grosschreib}) und der Satzschreibung (Satz~\ref{satz:grunabhsatz} auf S.~\pageref{satz:grunabhsatz}) insofern, als sie eine neue Umgebung und Funktion für die Majuskel eröffnet.
Welche Funktion das genau ist, ist fraglich, aber vermutlich nah an der der Bindestrichschreibung.

\subsection{Abkürzungen und Auslassungen}

\label{sec:abkuerz}

\index{Kurzwort}
\index{Substantiv!s-Flexion}
\index{Akronym}
Zu den Abkürzungen und Auslassungen gehören zunächst echte \textit{Kurzwortbildungen}.
Einerseits gibt es sie von einem trunkierenden (abschneidenden) Typus wie \textit{Lok}, \textit{Vopo} oder \textit{Schweini} (vgl.\ Übung~\ref{u74} auf S.~\pageref{u74}).
Diese Wörter werden graphematisch nicht besonders markiert und verhalten sich grammatisch und graphematisch wie andere Wörter.%
\footnote{Mit der Besonderheit, dass viele von ihnen auf Vollvokal enden und damit nicht ganz perfekte Wörter des Kernwortschatzes sind.}
Den graphematisch interessanten Typus stellen Abkürzungen dar, die aus explizit gelesenen Anfangsbuchstaben von Wortfolgen oder Gliedern von Komposita bestehen (\textit{Akronyme}), \zB \textit{LKW} \textipa{[PElkave:]} (\textit{Lastkraftwagen}), \textit{AU} \textipa{[Pa:Pu:]} (\textit{astronomische Einheit}), \textit{SHK} \textipa{[PEshaka:]} (\textit{studentische Hilfskraft}).
Hierbei handelt es sich um genuine Wörter, die sich zwar aus einer rein graphischen Abkürzungskonvention ergeben, die aber als Klasse klar benennbare grammatische Eigenschaften haben, wie \zB die Betonung auf der letzten Silbe.
In Fällen, wo sich dies anbietet (typischerweise wenn sich eine Folge aus Vokal, Konsonant und Vokal ergibt), werden sie allerdings auch gerne mit Erstsilbenakzent und nicht buchstabierend gelesen, also \textit{ASU} \textipa{[Pa:zu]} (\textit{Abgassonderuntersuchung}).
Der besondere Charakter dieser Bildungen (und ihre Verankerung im grammatischen System) zeigt sich auch an der \textit{s}"=Plural-Bildung, die an die Stelle der Pluralbildung des Vollwortes tritt, also \textit{LKWs}, \textit{AUs}, \textit{SHKs}, \textit{ASUs}.
Gelegentliche gespreizte Schreibungen wie *\textit{den SHKen} für \textit{den studentischen Hilfskräften} sind insofern ungewöhnlich, als es sich bei den betreffenden Wörtern eben nicht um reine Buchstaben-Abkürzungen handelt, denen man die Flexion des zugehörigen Vollwortes verpassen kann, sondern um Kurzwörter, die wie zu erwarten die \textit{s}"=Flexion nehmen.
Kaum hört man dementsprechend die Realisierung *\textipa{[PEshaka:@n]}.
Mit gleichem Recht könnte man sonst auch *\textit{den Loken} (für \textit{den Lokomotiven}) oder *\textit{den Fundin} (für \textit{den Fundamentalpolitikern}) schreiben und sprechen.

Die echten, mit Punkten markierten Schreib-Abkürzungen wie \textit{Abk.}, \textit{usw.} oder \textit{z.H.} sind graphematisch vor allem interessant, weil sie das Funktionsspektrum des Punktes über seine Kernfunktion (Abschnitt~\ref{sec:hauptsatzschreib}) hinaus erweitern.
Sie haben in der Regel keine eigene phonologische Korrespondenz und werden beim lauten Lesen zu vollen Wörtern rekonstruiert (also \zB \textit{usw.} zu \textit{und so weiter}), gelegentliche Späße wie \zB \textipa{[PUz@v@]} (oder ähnlich) für \textit{usw.} ausgenommen.

Den interessantesten Fall von Abkürzungen im weitesten Sinn findet man in der Form von sogenannten \textit{Klitisierungsphänomenen} und ihrer Verschriftung.
Klitisierung ist ein Prozess, bei dem Wörter typische Worteigenschaften verlieren und sich eher in die Richtung eines Affixes entwickeln, ohne jedoch (zunächst) ganz dort anzukommen.
Sie werden zum Klitikon, vgl.\ Definition~\ref{def:klitikon}.

\Definition{Klitisierung (Klitikon)}{\label{def:klitikon}%
Ein Wort \textit{klitisiert} (wird zum \textit{Klitikon}), wenn es seinen Wortakzent verliert und sich prosodisch einem vorangehenden (\textit{Enklise}) oder folgenden Wort (\textit{Proklise}) anschließt.
Morphologisch bleibt es selbständig, wird also nicht zum Affix.
\index{Klitikon}
}

\index{Gebrauchsschreibung}
Klitisierungen findet man im deutschen Standard wenige, in der gesprochenen Sprache und in Gebrauchsschreibungen aber durchaus mehr.
Relevante Beispiele sind in (\ref{ex:msschr010})--(\ref{ex:msschr012}) zusammengefasst.

\begin{exe}
  \ex{\label{ex:msschr010} im, zum, zur, ins}
  \ex\label{ex:msschr011} 
  \begin{xlist}
    \ex{\label{ex:msschr011a} durch's, auf's, mit'm, so'n, so'nen}
    \ex{\label{ex:msschr011b} ich's, geht's, hat's}
  \end{xlist}


  \ex\label{ex:msschr012} 
  \begin{xlist}
    \ex{\label{ex:msschr012a} durchs, aufs, mitm, son, sonen}
    \ex{\label{ex:msschr012b} ichs, gehts, hats}
  \end{xlist}
\end{exe}

\index{Apostroph}
In (\ref{ex:msschr010}) sind bereits vollständig standardisierte Klitisierungsprodukte als Endergebnis einer historischen Entwicklung zu sehen.
In diesen Fällen haben sich die Klitika \textit{m}, \textit{r} und \textit{s} (teilweise bereits unsegmentierbar) mit dem vorangehenden Wort verbunden und verhalten sich wie Affixe (s.\ auch Abschnitt~\ref{sec:syntaxflektierbareprp}).
Auf der Ebene der Schreibung wird dies dadurch abgebildet, dass sie auch wie Affixe (also ohne Spatien oder sonstige Kennzeichen) geschrieben werden.
In (\ref{ex:msschr011}) sind weniger stark standardisierte Klitisierungen mit Apostroph geschrieben.
Der Apostroph zeigt hier wahrscheinlich nicht nur das Fehlen von Buchstaben, sondern auch die Stammgrenze an.
Wenn wir diese Klitisierungen schreiben wie in (\ref{ex:msschr012}), sind vor allem Wörter wie \textit{mitm} auffällig, weil sie einen silbischen Nasal (s.\ Abschnitt~\ref{sec:silbischenasaleapproximanten}) verschriften und dadurch eine Buchstabensequenz ergeben, die sonst nicht vorkommt, und bei der die morphologische Segmentierung recht unklar ist.
Obwohl der Apostroph also bei diesen nicht kanonischen Klitisierungen eine gut benennbare und wichtige Funktion hat, findet man in entsprechenden Registern durchaus Schreibungen wie in (\ref{ex:msschr012}).
Falsch sind diese allerdings insofern auch nicht, als die Entwicklung zu einer Situation wie in (\ref{ex:msschr010}) für einzelne Schreiber unterschiedlich stark fortgeschritten sein kann, so dass (\ref{ex:msschr012}) die angemessenere Schreibung für sie ist.
Wie so oft bieten das grammatische System inklusiver der Schreibprinzipien mehrere Möglichkeiten an, und Sprecher bzw.\ Schreiber bedienen sich ihrer individuell verschieden.

\index{Artikel!indefinit}
Bezüglich \textit{son} und dem verkürzten Indefinitartikel \textit{n}, \textit{ne}, \textit{nen} usw.\ hat man im Übrigen festgestellt, dass besondere Entwicklungen innerhalb der Sprecher- und Schreibergemeinschaft im Gange sind.
Bei \textit{son} ist vor allem die Ausbildung eines Plurals wie in \textit{sone Pferde} markant, der sich nicht auf eine einfache Klitisierung reduzieren lässt, weil \textit{ein} keinen Plural hat (*\textit{so eine Pferde}).
Bei \textit{son} handelt es sich also vielmehr um ein neues Pronomen als um das Ergebnis einer Klitisierung, weswegen Schreibungen wie \textit{so'n} nicht (mehr) angemessen sind.

Der Indefinitartikel \textit{n} kommt \zB auch satzinitial vor, kann also zumindest nicht immer eine Enklise darstellen.\label{abs:nen}%
\footnote{Proklise wird für das Deutsche weitgehend ausgeschlossen.}
Da die Apo\-stroph\-schrei\-bung mehr ist als eine einfache Markierung von fehlendem Material, sondern eben auch Stammgrenzen markiert, ist besonders die satzinitiale Verwendung von apostrophiertem \textit{n} dem System fremd, vgl. (\ref{ex:msschr013}).
Variante (\ref{ex:msschr013b}) ist gegenüber (\ref{ex:msschr013a}) die schlechtere Lösung.


\begin{exe}
  \ex\label{ex:msschr013} 
  \begin{xlist}
    \ex{\label{ex:msschr013a} Nen Ausritt hat Vanessa heute nicht mehr geplant.}
    \ex{\label{ex:msschr013b} 'Nen Ausritt hat Vanessa heute nicht mehr geplant.}
  \end{xlist}
\end{exe}

Außerdem findet man die aufgefüllte Form \textit{nen} wie in \textit{nen Kind}, die ebenfalls dafür spricht, dass es sich nicht mehr nur um einen einfachen Reduktionsprozess handelt (*\textit{einen Kind}).
Damit ist also auch bei diesem neuen Artikel die Apo\-stroph\-schrei\-bung nur noch begrenzt einschlägig.
Hinzu kommt, dass die Genitive \textit{nes} und \textit{ner} wie in *\textit{der Mustang nes Freundes} oder *\textit{die Corvette ner Freundin} nicht verwendet werden, obwohl sie im Rahmen eines Klitisierungsprozesses ja durchaus verfügbar sein sollten.
Es bildet sich hier vielmehr ein neuer Indefinitartikel heraus, der zunächst umgangssprachlich und in Gebrauchsschreibungen alternativ zum Artikel \textit{ein} existiert und sehr spezifische Eigenschaften hat, die \textit{ein} nicht hat.

\index{Genitiv!sächsisch}
Angesichts der genannten Funktion des Apostrophs kann man nun auch das vieldiskutierte apostrophierte \textit{s} in Fällen wie (\ref{ex:msschr016}) bewerten.

\begin{exe}
  \ex\label{ex:msschr016} 
  \begin{xlist}
    \ex[*]{\label{ex:msschr016a} Emma's Lebkuchen}
    \ex[*]{\label{ex:msschr016b} legendäre Auto's}
  \end{xlist}
\end{exe}

Dass hier wie manchmal vermutet der Einfluss des Englischen am Werk ist, könnte für (\ref{ex:msschr016a}) -- den sogenannten \textit{sächsischen Genitiv} -- eventuell stimmen, aber für (\ref{ex:msschr016b}) definitiv nicht.
Selbst wenn das Englische beim Genitiv das Vorbild wäre, würde uns das noch nichts darüber sagen, ob die Schreibung ins deutsche System passt oder nicht.
Allein aus dem deutschen System heraus lässt sich aber argumentieren, dass sie \textit{nicht} hineinpasst.%
\footnote{Damit sage ich nicht, dass diese Schreibung in irgendeinem Sinn \textit{verboten gehört} (s.\ Abschnitt~\ref{sec:normalsbeschreibung}).
Ich möchte genausowenig voraussagen, dass sie sich nicht durchsetzen könnte.
Die linguistisch interessante -- hier aus Platzgründen nicht zu führende -- Diskussion wäre, inwiefern sich das System verändern würde, falls sie sich durchsetzen könnte.}
Es handelt sich bei \textit{-s} um Flexionsaffixe, die Genitiv und Plural der \textit{s}"=Flexion anzeigen.
Wie weiter oben in diesem Abschnitt argumentiert wurde, werden sonst im Deutschen Flexionsaffixe nie durch Apostroph abgetrennt.
Es gibt in (\ref{ex:msschr016}) weder einen Bedarf, die Stammgrenze zu markieren, noch ist irgendwelches ausgelassenes Material zu rekonstruieren, und der Apostroph wird damit jenseits seiner sonst im System verankerten Funktion verwendet.
Das Englische hat im Übrigen eine zusätzliche Motivation, das \textit{'s} des Genitivs mit dem Apostroph graphematisch besonders zu markieren.
Es ist nämlich eigentlich kein Flexionsaffix, sondern eine Art Klitikon, das auch an komplexere NPs angefügt werden kann, vgl.\ (\ref{ex:msschr017}) und vor allem (\ref{ex:msschr017b}).


\begin{exe}
  \ex\label{ex:msschr017} 
  \begin{xlist}
    \ex{\label{ex:msschr017a} Emma's gingerbread}
    \ex{\label{ex:msschr017b} the Queen of Denmark's gingerbread}
  \end{xlist}
\end{exe}

Die Situation ist also eine ganz andere als im Deutschen, wo \textit{-s} ein ganz normales Flexionsaffix ist.

\subsection{Konstantschreibungen}

\label{sec:konstanz}

Im Bereich der Wortschreibungen geht es in diesem Abschnitt abschließend um ein wichtiges Prinzip, mit dem wir uns unter anderem nochmal auf Schärfungs- und Silbengelenkschreibungen sowie \textit{h}-Schreibungen an der Silbengrenze (s.\ dazu Abschnitt~\ref{sec:laengeschreib}) rückbeziehen.
Dort wurde besprochen, dass im Kernwortschatz der kurze geschlossene Einsilbler bis auf wenige Ausnahmen mit Schärfungsschreibung geschrieben wird (\textit{platt}, \textit{Rock}, \textit{Kamm}).
Andererseits wurde betont, dass die Schärfungsschreibung vor allem als Silbengelenkschreibung motiviert ist (\textit{platter}, \textit{Röcke}, \textit{Kämme}).
Wenn man das Prinzip der Konstantschreibung zugrundelegt, kann man erklären, warum die eigentlich überflüssige Schärfungsschreibung im Einsilbler erfolgt.
Es wird Satz~\ref{satz:konstantschr} aufgestellt.

\Satz{Prinzip der Konstantschreibung}{\label{satz:konstantschr}%
Wortformen eines Stammes (in zweiter Näherung eines Wortes) werden möglichst konstant (also im weitesten Sinn \textit{ähnlich}) geschrieben.
\index{Schreibprinzip!Konstanz}
}

Da nun in Formen wie \textit{platter}, \textit{Röcke} und \textit{Kämme} die Schärfungsschreibung als Gelenkschreibung nötig ist, um die phonetischen Korrelate *\textipa{[pla:t5]}, *\textipa{[r\o:k@]} und *\textipa{[kE:m@]} zu verhindern, kann man dem Prinzip der Konstantschreibung bei diesen Stämmen nur gerecht werden, wenn die einsilbige Form die Schärfungsschreibung übernimmt.
Anders gesagt wird \textit{Rock} also vor allem deshalb nicht *\textit{Rok} geschrieben, weil *\textit{Röke} nur *\textipa{[r\o:k@]} und nicht \textipa{[r{\oe}\Sgel{k}@]} gelesen werden könnte.
Damit kann man die Schärfungsschreibung weitgehend als Silbengelenkschreibung umdeuten (aber siehe Übung~\ref{u151}), und die kurzen geschlossenen Einsilbler mit Schärfungsschreibung gehen auf das Konto der Konstantschreibung.

Mit dem Prinzip der Konstantschreibung lassen sich auch \textit{ß}"=Schreibungen wie \textit{aß} (statt *\textit{as} wie in \textit{las}) erklären.
Zwar ist \textit{aß} nicht besonders konstant zum Präsensstamm \textit{ess}, aber dieser ist eben auch ein anderer Stamm.%
\footnote{Die Schreibprinzipien sind sozusagen nicht dafür verantwortlich, dass es suppletive unregelmäßige Verben mit gibt (vgl.\ Abschnitt~\ref{sec:kleineverbklassen}, insbesondere S.~\pageref{abs:suppletiv}).
Um ein extremeres Beispiel zu nennen:
Es ist sicherlich kein Bruch des Prinzips der Konstantschreibung, dass \textit{war} nicht konstant zu \textit{sei} geschrieben wird.
Die Schreibung kann schließlich nicht konstanter sein als die Morphophonologie es erlaubt.}
Da innerhalb der Formen des Präteritalstamms \textit{aßen} nur mit \textit{ß} geschrieben werden kann, weil sowohl *\textit{asen} als auch *\textit{assen} nicht das gewünschte phonologische Korrelat haben (s.\ Abschnitt~\ref{sec:eszett}), ist \textit{aß} die konstanteste aller Möglichkeiten.
Schreibungen wie *\textit{Fluß} aus der Zeit vor der Orthographiereform von 1996 waren aber die Normierung einer ungrammatischen Form, weil einerseits (untypisch vor \textit{ß}) ein kurzes /\textipa{U}/ vorliegt, und weil *\textit{Fluß} außerdem keine Konstantschreibung zu \textit{Flüsse} ist.

\index{Dehnungsschreibung}
Formen wie \textit{siehst} sind nun ebenfalls systematisch erklärbar.
Das \textit{ie} für /\textipa{i:}/ ist als einzige Dehnungsschreibung nahezu obligatorisch, und das \textit{h} stellt eine Konstantschreibung zu den Formen des Verbs dar, in denen \textit{h} die Silbengrenze markiert (Abschnitt~\ref{sec:intervokh}).
Es liegt also keine doppelte Dehnungsschreibung vor, sondern eine obligatorische Dehnungsschreibung und eine Konstantschreibung.

\index{Umlaut!Schreibung}
Üblicherweise wird auch die starke graphische Ähnlichkeit der Umlautvokale als Zeichen des Prinzips der Konstantschreibung gewertet.
Graphisch sind also Wörter wie \textit{öfter} zu \textit{oft} und \textit{brünstig} zu \textit{Brunst} stammkonstanter, als wenn eigene Buchstaben mit stark abweichender Form verwendet würden.
Damit argumentiert man allerdings überwiegend historisch (sprachgeschichtlich), weil im gegenwärtigen System schlicht keine Alternativen zu den Buchstaben \textit{ö} und \textit{ü} existieren.%
\footnote{Die Alternative \textit{y} zu \textit{ü} kann vernachlässigt werden, weil sie extrem selten und auf Lehnwörter -- in der Regel außerhalb des Kernwortschatzes -- beschränkt ist.}
Nur \textit{a} und \textit{ä} sowie \textit{au} und \textit{äu} sind im gegenwärtigen System für diese Art der Konstantschreibung relevant.
Im Fall von \textit{ä}-Schreibungen, die mit ungespanntem \textipa{[E]} korrespondieren, existiert durchaus die alternative Schreibung \textit{e}.
In Stämmen mit umgelautetem \textit{a} wird allerdings konsequent der ähnlichere Buchstabe \textit{ä} geschrieben (\textit{Säcke} statt *\textit{Secke} zu \textit{Sack}).
Für gespanntes \textipa{[E:]} wie in \textit{Stäbe} \textipa{[StE:b@]} zu \textit{Stab} \textipa{[Sta:p]} käme eine \textit{e}-Schreibung allerdings gar nicht infrage, weil *\textit{Stebe} im Standard immer *\textipa{[Ste:b@]} entspräche.
Die \textit{ä}-Schreibungen für gespanntes \textipa{[E:]} erklären sich in diesem Fall also ohne das Prinzip der Konstanz.
Beim umgelauteten Diphthong \textit{äu} \textipa{[\t{O\oe}]} zu \textit{au} -- für den die Alternative \textit{eu} existiert --, wird hingegen im Sinne der Stammkonstanz niemals auf \textit{eu} ausgewichen.
Dass man also niemals *\textit{Reume} statt \textit{Räume} oder *\textit{leuft} statt \textit{läuft} schreibt, hat durchaus damit zu tun, dass diese Schreibung konstanter zu denen der Stämme \textit{Raum} und \textit{lauf} sind.
Wenn man die Konstanz der Silbengelenkschreibung und der Umlautgraphien zusammen betrachtet, gäbe es ohne das Prinzip der Konstantschreibung im Paradigma eines Wortes also (auf Basis der anderen Schreibprinzipien) ungünstige Schreibungen wie *\textit{Kam} und *\textit{Kemme} statt \textit{Kamm} und \textit{Kämme}.
Inwieweit das System der Konstantschreibung im Verbund mit der Schärfungsschreibung als Gelenkschreibung noch eine produktive Regularität des Systems ist, ist allerdings schwierig zu beurteilen.
Es ist gut möglich, dass wir in vielen Fällen das Produkt einer historischen Entwicklung beobachten und es sich bei Konstant- und Gelenkschreibung um diachrone Prinzipien handelt (s.\ auch Übung~\ref{u151}).

Damit endet der (alles andere als vollständige) Überblick über wortbezogene Schreibungen im Deutschen.
Wörter wurden in diesem Buch als die kleinsten Einheiten der Syntax behandelt, die Phrasen und Sätze bilden.
Ob und wie Phrasen und Sätze in der Schreibung kodiert werden, ist Thema von Abschnitt~\ref{sec:satzschreib}.


\Zusammenfassung{%
Spatien trennen syntaktische Wörter.
Morphosyntaktische Univerbierung kann zur Zusammenschreibung führen.
Wortklassen werden i.\,d.\,R.\ in der Schreibung nicht markiert, außer im Fall der positionsunabhängigen Substantiv- und Eigennamengroßschreibung.
Ehemalige Substantive, die ihren Status als Substantiv verloren haben, können dementsprechend kleingeschrieben werden.
Aus Buchstaben-Abkürzungen hervorgegangene Wörter wie \textit{LKW}, die komplett in Majuskeln geschrieben werden, sind Kurzwörter mit spezifischen grammatischen Eigenschaften. 
Klitisierung wird im Deutschen durch den Apostroph markiert, kann bei voranschreitender Konventionalisierung aber auch zu einer Zusammenschreibung führen.
Konstantschreibungen sichern die Wiedererkennbarkeit von Stämmen bei Flexions- und Wortbildungsprozessen -- eventuell aber eher als historisches Prinzip.
Wir finden sie prominent in Form der Schärfungsschreibungen im Einsilbler.
}


\section{Schreibung von Phrasen und Sätzen}

\label{sec:satzschreib}

\subsection{Phrasen}

\label{sec:phrasenschrift}
\label{sec:koordinschreib}

Mit diesem Abschnitt kommen wir zur \textit{Interpunktion}.
Innerhalb der Interpunktion beschränken wir uns vor allem auf die zentralen Zeichen \textit{Komma} und \textit{Punkt}.
Graphematisch betrachtet ist im Bereich der Phrasen (unterhalb der Nebensatz- und Satzebene) nicht viel zu holen, zumindest wenn man wie hier die Schreibungen von Nebensätzen und ähnlichem in den Bereich der Satzschreibungen (Abschnitte~\ref{sec:hauptsatzschreib} und~\ref{sec:nebensatzschreib}) verschiebt.
Ein wichtiger Bereich, in dem das Komma eine seiner Kernfunktionen hat, sind allerdings Aufzählungen.
In Abschnitt~\ref{sec:koor} wurde argumentiert, dass Konjunktionen zwei Phrasen gleichen Typs (also zwei NPs, zwei APs usw.) zu einer Phrase desselben Typs verbinden können.
Obwohl es bisher nicht ausdrücklich erwähnt wurde, gilt dies auch für satzförmige Strukturen wie Komplement- oder Relativsätze.
Wie schon in Abschnitt~\ref{sec:adjektiveundartikelwoerter} in der Diskussion von koordinierten APs angedeutet, kann man das Komma in Aufzählungsstrukturen als eine rein graphische Konjunktion betrachten.
Wie in (\ref{ex:msschr014}) wird dabei die letzte Phrase üblicherweise mit einer normalen Konjunktion statt mit Komma angefügt.

\begin{exe}
  \ex{\label{ex:msschr014} Vanessa putzt Tarek, Bird Brain und Dragonfly.}
\end{exe}

\index{Koordination!Schreibung}
\index{Konjunktion}
\index{Komma}
Zum Komma und seiner Funktion kommen wir in Abschnitt~\ref{sec:nebensatzschreib} nochmals zurück.
Es gibt diverse weitere Möglichkeiten, Aufzählungen graphisch zu kennzeichnen, die im Kern alle ähnlich wie das Komma funktionieren, aber graphische Besonderheiten aufweisen und teilweise stark auf bestimmte Kontexte oder Schreibstile beschränkt sind.
Zu diesen Zeichen zählen das Et-Zeichen \textit{\&}, der Listenstrich -- und das Pluszeichen $+$.
Auf jeden Fall ist die Strukturebene, auf der das Komma und die anderen genannten Zeichen angesiedelt sind, nicht das Wort, sondern die Ebene der Phrasen und Sätze.
Es sind \textit{syntaktische Zeichen} gemäß Definition~\ref{def:syntzeich}.

\Definition{Syntaktisches Zeichen}{\label{def:syntzeich}%
\textit{Syntaktische Zeichen} sind Nicht-Buchstaben, die im Bereich der Phrasen- und Satzschreibungen verwendet werden.
\index{Zeichen!syntaktisch}
}

\index{Parenthese}
Im Bereich der Phrasenschreibung sind ansonsten noch \textit{Parenthesen} zu berücksichtigen.
Unter Parenthesen verstehen wir hier phrasenförmige Einschübe, die in den Satzzusammenhang gestellt werden, aber zu diesem keine reguläre syntaktische Beziehung haben.
Sie sind im Normalfall keine syntaktischen Konstituenten des Satzes.
Für Parenthesen kommen verschiedene Markierungen infrage, insbesondere der Gedankenstrich, die Klammer und das Komma.
Vgl.\ die Beispiele in (\ref{ex:msschr0015}).


\begin{exe}
  \ex\label{ex:msschr0015} 
  \begin{xlist}
    \ex[ ]{\label{ex:msschr0015a} Emma ist -- es war wohl gestern -- alleine ausgeritten.}
    \ex[ ]{\label{ex:msschr0015b} Emma ist (es war wohl gestern) alleine ausgeritten.}
    \ex[ ]{\label{ex:msschr0015c} Emma ist, es war wohl gestern, alleine ausgeritten.}
  \end{xlist}
\end{exe}

\index{Gedankenstrich}\index{Klammer}
Der Gedankenstrich in (\ref{ex:msschr0015a}), die Klammern in (\ref{ex:msschr0015b}) und das Komma in (\ref{ex:msschr0015c}) sind für diesen Zweck ähnlich oder gleich gut funktionierende Markierungen der Nicht-Integriertheit der Parenthese in die Syntax des Satzes.
Die Parenthese \textit{es war wohl gestern} ist selber ein vollständiger Verb-Zweit-Satz und lässt sich nach den Schemata aus Kapitel~\ref{sec:phrasen} und Kapitel~\ref{sec:saetze} nicht in den größeren Satz integrieren.
Nach meinem Kenntnisstand ist es empirisch nicht geklärt, welche Markierung für Parenthesen (ggf.\ abhängig von der Art und Position der Parenthese) Schreiber bevorzugen.
Die Rolle des Kommas im Bereich der Satzverknüpfungen ist hingegen relativ klar, und um sie geht es unter anderem in den Abschnitten~\ref{sec:hauptsatzschreib} und~\ref{sec:nebensatzschreib}.

\subsection{Unabhängige Sätze}

\label{sec:hauptsatzschreib}

\index{Satz!Schreibung}
Unabhängige Sätze sind in erster Näherung das, was mit Definition~\ref{def:satz} (S.~\pageref{def:satz}) eingegrenzt wurde, also ein finites Verb mit all seinen Ergänzungen, das nicht von einer anderen Konstituente abhängt.
Während (\ref{ex:msschr0016}) also ein unabhängiger Satz ist, ist das eingeklammerte Material in (\ref{ex:msschr0017}) nach dieser Definition jeweils kein unabhängiger Satz.

\begin{exe}
  \ex{\label{ex:msschr0016} Die Freundinnen reiten aus.}
  \ex\label{ex:msschr0017}
  \begin{xlist}
    \ex[ ]{\label{ex:msschr0017a} Emma weiß, dass [die Freundinnen ausreiten].}
    \ex[ ]{\label{ex:msschr0017b} Emma reitet aus, weil [das Wetter schön ist].}
    \ex[ ]{\label{ex:msschr0017c} Emma behauptet, [die Freundinnen reiten aus].}
  \end{xlist}
\end{exe}

\index{Punkt}
In (\ref{ex:msschr0017a}) und (\ref{ex:msschr0017b}) wird das eingeklammerte Material (jeweils eine VP) von einem Komplementierer regiert.
Diese VPs können nur innerhalb eines vollständigen Satzes vorkommen, aber selber keinen bilden.
In (\ref{ex:msschr0017c}) hängt eine Hauptsatzstruktur (V2), die völlig identisch zu (\ref{ex:msschr0016}) ist, vom Verb \textit{behaupten} ab.
Auch wenn dieser Satz unabhängig sein könnte, ist er es im gegebenen Fall nicht, denn er wird von einem anderen Verb regiert.

Die graphematischen Markierungen unabhängiger Sätze sind bekannt.
Einerseits wird im graphematischen unabhängigen Satz das erste Wort positionsbedingt mit einer einleitenden Majuskel geschrieben, andererseits finden wir den Punkt als das wichtigste \textit{Satzendezeichen}.
Wird Material, das nicht satzfähig ist, entsprechend markiert, ergeben sich in erster Näherung ungrammatische Konstruktionen wie in (\ref{ex:msschr00179}).


\begin{exe}
  \ex\label{ex:msschr00179}
  \begin{xlist}
    \ex[*]{\label{ex:msschr00179a} Die Freundinnen ausreiten.}
    \ex[*]{\label{ex:msschr00179b} Das Wetter schön ist.}
    \ex[*]{\label{ex:msschr00179c} Reiten aus.}
  \end{xlist}
\end{exe}

In (\ref{ex:msschr00179a}) und (\ref{ex:msschr00179b}) handelt es sich um die nicht satzfähigen VPs aus (\ref{ex:msschr0017}).
In (\ref{ex:msschr00179c}) ist der Satz nicht vollständig, weil das Subjekt fehlt.
Zumindest im Bereich der Gebrauchsschreibungen ist die Angelegenheit allerdings ein bisschen komplexer, worauf am Ende dieses Abschnitts kurz eingegangen wird, \zB mit Beispiel (\ref{ex:msschr7771}) auf S.~\pageref{ex:msschr7771}.

Alternativ zum Punkt kommt optional das \textit{Ausrufezeichen} als Satzendezeichen infrage.
In Fragen ist das \textit{Fragezeichen} obligatorisch.
Ausrufezeichen und Fragezeichen haben im Gegensatz zum Punkt zusätzliche semantische bzw.\ pragmatische Funktionen, prototypisch die Markierung von Ausruf und Frage.
Sie können sehr eingeschränkt auch an anderen Positionen vorkommen, vgl. (\ref{ex:msschr7770}).

\begin{exe}
  \ex\label{ex:msschr7770}
  \begin{xlist}
  	\ex{\label{ex:msschr7770a} Emma ist alleine(!) ins Outback geritten.}
  	\ex{\label{ex:msschr7770b} Ich habe gehört, Vanessa(?) hätte Tarek geputzt.}
  \end{xlist}
\end{exe}

Die Verwendungen in (\ref{ex:msschr7770}) können aber im Gegensatz zur Verwendung als Satzschlusszeichen durchaus als randständig bezeichnet werden.
Wir gehen also hier ganz klassisch davon aus, dass der Punkt, das Ausrufezeichen und das Fragezeichen die Satzschlusszeichen sind, und dass die beiden letztgenannten zusätzliche Funktionen haben (und in weiteren Kontexten auftreten) können.

\index{Satz!Koordination}
Das bisher besprochene System der syntaktischen Zeichen erlaubt eine Doppeldeutigkeit im Bereich der Satzinterpunktion.
Weil das Komma genauso wie Konjunktionen alle Arten von Konstituenten koordinieren kann, sind Alternativschreibungen wie in (\ref{ex:msschr0018}) möglich.

\begin{exe}
  \ex\label{ex:msschr0018} 
  \begin{xlist}
    \ex{\label{ex:msschr0018a} Emma reitet aus, Vanessa putzt Dragonfly.}
    \ex{\label{ex:msschr0018b} Emma reitet aus. Vanessa putzt Dragonfly.}
  \end{xlist}
\end{exe}

Zweifelsohne sind die beiden Sätze vor und nach dem Komma in (\ref{ex:msschr0018a}) nach unserer Definition unabhängig.
Außerdem zeigt (\ref{ex:msschr0018b}), dass der Satzende-Punkt hier durchaus eine gute Option ist.
Trotzdem kann das Koordinationskomma zum Einsatz kommen, also ein syntaktisches Zeichen, das eigentlich zur Markierung syntaktischer Unabhängigkeit schlechter geeignet ist als die Satzendezeichen.
Schreiber haben wahrscheinlich gute Gründe, Variante (\ref{ex:msschr0018a}) oder (\ref{ex:msschr0018b}) zu wählen.
Bei der Erklärung ist zu berücksichtigen, dass sich bestimmte Satzpaare besser für eine Kommaschreibung eignen als andere, vgl.\ (\ref{ex:msschr0019}).

\begin{exe}
  \ex\label{ex:msschr0019} 
  \begin{xlist}
    \ex[?]{\label{ex:msschr0019a} Gestern wurden die Pferde neu behuft, die Lichtgeschwindigkeit ist in allen Bezugssystemen konstant.}
    \ex[ ]{\label{ex:msschr0019b} Gestern wurden die Pferde neu behuft und die Lichtgeschwindigkeit ist in allen Bezugssystemen konstant.}
  \end{xlist}
\end{exe}

Aus ganz verschiedenen inhaltlichen bzw.\ semantischen und syntaktischen Gründen eignen sich die Sätze in (\ref{ex:msschr0019}) nicht zur Koordination, wobei die Koordination mit \textit{und} wenigstens näherungsweise grammatisch wirkt, was für die Version mit Komma stark angezweifelt werden kann.
Im Vergleich dazu ist (\ref{ex:msschr0018}) ein ideales Beispiel für gut koordinierbare Sätze, weil sich eine Reihe aus zwei syntaktisch und semantisch gleich strukturierten Sätzen ergibt.
Der Sprecher oder Schreiber stellt durch die Reihung der Sätze den Kontrast zwischen dem, was Emma tut, und dem, was Vanessa tut, in den Vordergrund.
Eine umfassende Beschreibung der Faktoren, die die verschiedenen möglichen Strukturen hier begünstigen, würde den gegebenen Rahmen sprengen.
Als Erkenntnis kann aber Satz~\ref{satz:grunabhsatz} festgehalten werden.

\Satz{Graphematisch unabhängiger Satz}{\label{satz:grunabhsatz}%
Der graphematisch unabhängige Satz wird mit einleitender Majuskel geschrieben und durch ein Satzendezeichen abgeschlossen.
Er besteht ggf.\ aus mehreren syntaktisch unabhängigen Sätzen, die durch Komma getrennt sind.
\index{Satz!graphematisch}
}

Wir haben behauptet, dass die drei Satzendezeichen (in Zusammenarbeit mit der Großschreibung) in ihrer prototypischen Verwendung eine längere Rede in Teile trennen, innerhalb derer die syntaktischen Regularitäten (s.\ Kapitel~\ref{sec:phrasen} bis~\ref{sec:relationenpraedikate}) beobachtbar sind.
Jenseits der Satzgrenze -- im Schriftmedium markiert durch Satzendezeichen -- gelten sie nicht.
Man erwartet \zB nach einem Punkt nicht, dass noch Ergänzungen oder Angaben folgen, die zum abgeschlossenen Satz gehören, vgl.\ (\ref{ex:msschr77715}).
Genauso können Nebensätze zwar ganz ans Ende nach ihren Matrixsatz gestellt werden wie in (\ref{ex:msschr77716a}), aber nicht nach den nachfolgenden Satz wie in (\ref{ex:msschr77716b}).

\begin{exe}
  \ex[*]{\label{ex:msschr77715} Wir haben gestern den ganzen Tag behuft. Die Pferde.}
  \ex\label{ex:msschr77716}
  \begin{xlist}
  	\ex[ ]{\label{ex:msschr77716a} Wir haben das Pferd gestern neu behufen lassen, das immer so gerne ins Gelände reitet. Der Hufschmied meinte, es wäre höchste Zeit.}
  	\ex[*]{\label{ex:msschr77716b} Wir haben das Pferd gestern neu behufen lassen. Der Hufschmied meinte, es wäre höchste Zeit. Das immer so gerne ins Gelände reitet.}
  \end{xlist}
\end{exe}

Im Gegensatz dazu stellen Beispiele wie die in (\ref{ex:msschr7771}) ein Problem dar, weil hier Satzendezeichen verwendet werden, aber das abgetrennte Material nach unserer Definition nicht satzförmig ist.

\begin{exe}
  \ex\label{ex:msschr7771}
  \begin{xlist}
  	\ex{\label{ex:msschr7771a} Edgar öffnete die Motorhaube. Totalschaden.}
  	\ex{\label{ex:msschr7771b} Ich, ein Lügner?}
  	\ex{\label{ex:msschr7771c} Raus!}
  \end{xlist}
\end{exe}

Wir sprechen diese Probleme jetzt eher an, als sie zu lösen.
Den Beispielen in (\ref{ex:msschr7771}) fehlt vor allem ein finites Verb.
Man kann jetzt versuchen, alle diese Fälle als Ellipsen zu interpretieren und damit zu unterstellen, dass hier Material weggelassen wurde, das man aber rekonstruieren kann, um den Satzstatus sicherzustellen.
Es käme so etwas wie in (\ref{ex:msschr7772}) heraus.

\begin{exe}
  \ex\label{ex:msschr7772}
  \begin{xlist}
  	\ex{\label{ex:msschr7772a} Edgar öffnete die Motorhaube. Es war ein Totalschaden.}
  	\ex{\label{ex:msschr7772b} Ich soll ein Lügner sein?}
  	\ex{\label{ex:msschr7772c} Mach, dass du rauskommst!}
  \end{xlist}
\end{exe}

Gegen diesen Ansatz als allgemeine Erklärung spricht vor allem, dass er nicht erklärt, warum bestimmte Typen solcher angeblicher Ellipsen bevorzugt sind, andere aber nicht.
Die interessante Frage wäre also, warum \textit{Raus!} häufiger ist als \textit{Pullover!} oder \textit{Dürfen!} usw.
Hier fehlt der Raum, um auf solche Fragen einzugehen.
Sprecher und Schreiber geben hier ganz allgemein gesagt bestimmten Einheiten, die keine prototypischen Sätze sind, durch die graphischen Markierungen eine größere Unabhängigkeit.
Damit nehmen diese Einheiten Züge einer Äußerung an.
Mit den grammatischen Eigenschaften eines Satzes hat das Ganze also eventuell nur bedingt zu tun, mit seinen semantischen und pragmatischen Eigenschaften (die in Abschnitt~\ref{sec:hauptsatzmatrixsatz} kurz angesprochen wurden) aber unter Umständen sehr viel.
Mit einer kategorischen Eins-zu-eins-Definition von graphematischen Sätzen oder Interpunktionszeichen kommt man in der Sache auf jeden Fall nicht weiter.

\subsection{Nebensätze und Verwandtes}

\label{sec:nebensatzschreib}

\index{Nebensatz!Schreibung}
Dass Nebensätze mit Komma abgetrennt werden (sollen), ist ein elementarer und jedem Absolventen einer Schule im deutschsprachigen Raum bekannter Grundsatz.
Egal, wo sie stehen, werden Nebensätze (zunächst alle Strukturen, die in Abschnitt~\ref{sec:nebensaetze} besprochen wurden) links und rechts (falls der linke Rand nicht mit dem Satzanfang bzw.\ der rechte Rand nicht mit dem Satzende zusammenfällt) mit einem Komma begrenzt, vgl.\ (\ref{ex:msschr0020}).

\begin{exe}
  \ex\label{ex:msschr0020} 
  \begin{xlist}
    \ex{\label{ex:msschr0020a} Jeder, der gerne reitet, mag eigentlich auch Pferde.}
    \ex{\label{ex:msschr0020b} Falls das Wetter gut ist, reiten Emma und Vanessa aus.}
    \ex{\label{ex:msschr0020c} Alle wissen, dass Tarek kein Pferd für Anfänger ist.}
  \end{xlist}
\end{exe}

Dadurch, dass Nebensätze oft im Vorfeld und Nachfeld stehen, ist für diese Feldergrenzen das Komma sehr charakteristisch, aber natürlich kein wirklich zuverlässiges Kennzeichen.
Das Nebensatzkomma ist also niemals ein \textit{Felderkomma}, sondern wird durch Satz~\ref{satz:nebensatzkomma} charakterisiert.
Die Bezeichnung \textit{Nebensatzkomma} ist dabei nicht optimal, wie sich gleich im Anschluss zeigen wird.

\Satz{Nebensatzkomma}{\label{satz:nebensatzkomma}%
Das \textit{Nebensatzkomma} markiert Konstituenten, die prototypisch eine VP enthalten und innerhalb ihres Matrixsatzes Satzglieder sind.
}


Die Frage ist, ob dasselbe in Sätzen wie in (\ref{ex:msschr0021}) gilt.
Anders formuliert ist die Frage, ob die Infinitiv-VPs in (\ref{ex:msschr0021}) grammatisch und graphematisch in die Nähe von Nebensätzen mit finitem Verb gestellt werden können.

\begin{exe}
  \ex\label{ex:msschr0021} 
  \begin{xlist}
    \ex{\label{ex:msschr0021a} ob Vanessa öfter auszureiten scheint}
    \ex{\label{ex:msschr0021b} ob Vanessa wünscht(,) am Abend auszureiten}
    \ex{\label{ex:msschr0021c} ob Emma glaubt, dass Vanessa wünscht, am Abend auszureiten}
    \ex{\label{ex:msschr0021d} ob Vanessa Tarek putzt, um später auf ihm auszureiten}
  \end{xlist}
\end{exe}

\index{Kohärenz!Schreibung}
\index{Infinitiv}
\index{Komplementsatz}
In allen Sätzen aus (\ref{ex:msschr0021}) liegen \textit{zu}-Infinitive vor (vgl.\ Abschnitte~\ref{sec:halbmodale} und~\ref{sec:kontrollinfinitive}).
In (\ref{ex:msschr0021a}) regiert das Halbmodalverb \textit{scheinen} diesen \textit{zu}"=Infinitiv, der obligatorisch kohärent konstruiert.
Im optional inkohärenten Fall mit Kontrollinfinitiven in (\ref{ex:msschr0021b}) scheint das Komma mehr oder weniger optional zu sein.
Wenn der \textit{zu}"=Infinitiv über eine Nebensatzgrenze nach rechts versetzt wird wie in (\ref{ex:msschr0021c}), was nur mit inkohärenten Infinitiven geht, steht das Komma obligatorisch.
Es kann natürlich nicht entschieden werden, ob hier die rechte Grenze des Komplementsatzes markiert wird oder die linke Grenze des Infinitivs.
Bei Angaben in Form des \textit{zu}"=Infinitivs wie in (\ref{ex:msschr0021d}) ist das Komma im Grunde nicht weglassbar.
Diese werden zudem gerne ins Vor- oder Nachfeld und nur selten ins Mittelfeld gestellt.

Der \textit{zu}"=Infinitiv hat also eine Nähe zu Nebensätzen, weil er ähnliche Positionen im Satz einnimmt wie diese, und weil er unter Umständen eine eigene Phrase bilden kann, die unabhängig vom Verbkomplex ist.
Die wichtigen Unterschiede des \textit{zu}"=Infinitivs zum Nebensatz sind, dass das Verb nicht finit ist und dass er prototypisch ohne Komplementierer auftritt.
Ob man daher die \textit{zu}"=Infinitive zu den Nebensätzen zählen will oder nicht, ist abhängig von der spezifischen Theorie und persönlichem Geschmack.
Die Interpunktion hat jedenfalls eine deutliche Tendenz, anhand der Kohärenz des \textit{zu}"=Infinitivs zu entscheiden, ob mit Komma getrennt wird oder nicht.

Viele Interpunktionszeichen sind hier aus Platzgründen unberücksichtigt geblieben, \zB das Semikolon oder der Doppelpunkt.
Die weiterführende Literatur hat auch zu ihnen eine Fülle von Daten und theoretischen Analysen parat.
Und wie bei allen grammatischen Phänomenen haben wir als Sprecher, Hörer, Schreiber und Leser jederzeit die Möglichkeit, selber nach Regularitäten im Gebrauch zu suchen.


\Zusammenfassung{%
Phrasen werden im Allgemeinen nicht interpunktiert.
Ausnahmen können durch parenthetische Phrasen gebildet werden (Gedankenstrich, Klammer, Komma.)
Unabhängige Sätze werden durch Satzschlusszeichen und initiale Majuskel markiert.
Äußerungsartige Fragmente, die syntaktisch nicht satzfähig sind, können aber ebenfalls wie Sätze interpunktiert werden.
Nebensätze und VPs mit Satzgliedstatus (\textit{zu}"=Infinitive) werden durch Komma markiert.
}


\Uebungen

\Uebung[\tristar] \label{u151} Warum sind Wörter wie \textit{dann} und \textit{wenn} oder \textit{Mett}, \textit{Müll}, \textit{Suff} problematisch, wenn man die Schärfungsschreibung vollständig auf eine Silbengelenkschreibung und Konstantschreibungen reduzieren möchte?

\Uebung \label{u153} Setzen Sie im folgenden Text alle Kommas, die Sie für angemessen halten und bestimmen Sie die Funktion gemäß den in diesem Kapitel genannten Grundsätzen.%
\footnote{\url{http://de.wikipedia.org/wiki/Molekulare_Ratsche} vom 21.03.2015, editiert vom Verfasser.}
Bei Nebensätzen stellen Sie zudem fest, ob ein Adverbialsatz, Komplementsatz, Relativsatz eingeleitet wird.
Markieren Sie die Kommas, die eine Struktur beenden, gesondert, und geben Sie an, welche Struktur es jeweils ist.

\begin{quote}
\small
Eine molekulare Ratsche oder auch Brownsche Ratsche ist eine Nanomaschine die aus brownscher Molekularbewegung (also aus Wärme) gerichtete Bewegung erzeugt.
Dies kann nur funktionieren wenn von außen Energie in das System gebracht wird.
Solche Systeme werden in der Literatur meistens Brownsche Motoren (siehe Literatur/Links) genannt.
Eine molekulare Ratsche ohne von außen zugeführter Energie wäre ein Perpetuum Mobile zweiter Art und funktioniert somit nicht.
Der Physiker Richard Feynman zeigte in einem Gedankenexperiment 1962 als erster wie eine molekulare Ratsche prinzipiell aussehen könnte und erklärte warum sie nicht funktioniert.

Eine molekulare Ratsche besteht aus einem Flügelrad und einer Ratsche mit Sperrzahn.
Die gesamte Maschine muss sehr klein sein (wenige Mikrometer) damit die Stöße des umgebenden Gases keinen nennenswerten Einfluss auf sie haben.
Die Funktionsweise ist denkbar einfach:
Ein Gasteilchen das das Flügelrad beispielsweise so trifft wie durch den grünen Pfeil markiert bewirkt ein Drehmoment das sich über die Achse auf die Ratsche überträgt und diese eine Stellung weiterdrehen kann.
Ein Teilchen das wie durch den roten Pfeil markiert auftrifft bewirkt keine Drehung da der Sperrzahn die Ratsche blockiert.
Die molekulare Ratsche sollte also aus Wärmeenergie eine gerichtete Bewegung erzeugen was aber nach dem zweiten Hauptsatz der Thermodynamik nicht möglich ist.

Der Sperrzahn funktioniert nur wenn er mit einer Feder gegen die Ratsche gedrückt wird.
Auch er unterliegt dem Bombardement der Brownschen Molekularbewegung.
Wird er durch diese ausgelenkt beginnt er auf die Ratsche zu schlagen was zu einem Nettodrehmoment entgegen der zuvor angenommen Drehrichtung führt.
Die Wahrscheinlichkeit für die Auslenkung des Sperrzahns die groß genug ist um eine Ratschenposition zu überspringen ist $exp(-\Delta E/k_BT)$ wobei $\Delta E$ die Energie die benötigt wird um die Feder des Sperrzahns auszulenken ist $T$ die Temperatur und $k_B$ die Boltzmann-Konstante.
Die Drehung über das Flügelrad muss aber auch die Feder spannen um in die nächste Position der Ratsche zu gelangen das heißt, die Wahrscheinlichkeit ist ebenfalls $exp(-\Delta E/k_BT)$.
Folglich dreht sich die Ratsche im Mittel nicht.

Anders sieht es aus wenn ein Temperaturunterschied zwischen Flügelscheibe und Ratsche vorliegt.
Ist die Umgebung des Flügelrades wärmer als die der Ratsche dreht sich die molekulare Ratsche wie zuvor angenommen.
Ist die Umgebung der Ratsche wärmer dreht sich die Maschine in die entgegengesetzte Richtung.

\end{quote}

\Uebung[\tristar] \label{u154} Vor Wörtern wie \textit{aber} und \textit{sondern} soll nach den Orthographieregeln ein Komma stehen.
Ordnen Sie die Wörter in eine Wortart ein und stellen Sie fest, welche syntaktischen Strukturen durch die Regel mit Komma getrennt werden.
Wird das Komma hier gemäß seiner in diesem Kapitel etablierten Funktionen verwendet?
Recherchieren Sie ggf., wie die Orthographieregel motiviert wird.

\Uebung[\tristar] \label{u155} Analysieren Sie die Schreibungen der Formen des Verbs \textit{nehmen} mit Bezug auf Dehnungs- und Schärfungsschreibungen sowie das Prinzip der Konstantschreibung.

