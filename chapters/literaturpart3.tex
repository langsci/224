\WeitereLiteratur{

\begin{sloppypar}

\paragraph*{Einführungen und Gesamtdarstellungen}

Wie immer kann der \textit{Grundriss} \citep{Eisenberg2013a} zur Vertiefung verwendet werden, genauso \citet{Engel2009}.
Zu allen Aspekten der deutschen Morphologie bietet \citet{HentschelVogel2009} gut lesbare Artikel.
Die hier vorgestellte Klassifikation der Wortarten ist eine Vereinfachung zu \citet{Engel2009a} und \citet{Engel2009}.
Etwas anders klassifiziert die Duden-Grammatik \citep{Duden8}.
Gut lesbare, allerdings nur auf Englisch verfügbare Einführungen in die Morphologie sind \citet{Katamba2006} und \citet{Booj2007}.

\paragraph*{Wortbildung}

Zur Einführung in die Wortbildung kann \citet{Altmann2011} verwendet werden.
Eine Gesamtdarstellung der deutschen Wortbildung ist \citet{FleischerBarz1995}.
Weiterführende Lesevorschläge: 
\citet{BreindlThurmair1992} gegen die Annahme von nominalen Kopulativkomposita im Deutschen;
\citet{Gallmann1999} und \citet{NueblingSzczepaniak2009} zu den Fugenelementen;
\citet{EisenbergSayatz2002} zu Reihen von Wortbildungssuffixen.

\paragraph*{Flexion}

Einen Überblick über die Flexion des Deutschen bietet \citet{ThieroffVogel2009}.
Der Status von Komparation als Flexion bzw.\ Wortbildung wird \zB in der IDS-Grammatik \citep[47f.]{ZifonunEa1997} und Abschnitt~5.2 von \citet{Eisenberg2013a} sowie Abschnitt~12.3 aus \citet{Eisenberg2013b} besprochen.

Weiterführende Lesevorschläge zu Nomina:
\citet{Wiese2012} zur Substantivflexion;
\citet{KoepckeZubin1995} zum Genus;
\citet{Wegener2004} zu Pluralbildungen von Lehnwörtern;
\citet{Koepcke1995} und \citet{Thieroff2003} zu schwachen Maskulina;
\citet{Wiese2009} und \citet{Nuebling2011} zu Aspekten der Adjektivflexion;
\citet{Vogel1997} zu unflektierten Adjektiven;
\citet{Baerentzen2002} zu \textit{deren} und \textit{derer}.
Von besonderer Bedeutung ist schließlich die historische Betrachtung der Morphosyntax deutscher Nomina, da das System auch heute noch im Umbruch ist.
\citet{Demske2000} bespricht hierzu eine Fülle von Daten.

Weiterführende Lesevorschläge zu Verben:
\citet{HelbigSchenkel1991} zur Subklassifikation der Verben nach ihren Valenzmustern;
\citet{Wiese2008} zu Klassifikation des Ablauts.

\paragraph*{Tempus und Modus}

Ausführlichere Einführungen zum Tempus sind \citet{Rothstein2007} und \citet{Vater2007}.
Weiterführende Lesevorschläge:
\citet{Leibukt2011} zur sogenannten \textit{Höflichkeitsfunktion} des Konjunktivs;
\citet{Fabriciushansen1997} zum Konjunktiv;
\citet{Fabriciushansen2000} zur \textit{würde}-Paraphrase.

\end{sloppypar}

}
