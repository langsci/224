\chapter{Konstituentenstruktur}
\label{sec:konstituentenstruktur}

\section{Syntaktische Struktur}
\label{sec:syntaktischestruktur}

In der Phonologie (Kapitel~\ref{sec:phonologie}) waren die wichtigsten zwei Fragen, welche Merkmale die phonologischen Bausteine (Segmente) haben, und nach welchen Regularitäten diese Bausteine zu Strukturen (\zB Silben) zusammengefügt werden.%
\footnote{Zu Beginn dieses Kapitels sollte zunächst Abschnitt~\ref{sec:valenz} (S.~\pageref{sec:valenz}) wiederholt werden.}
In der Morphologie (Teil \ref{part:morphologie}) ging es um morphologische Bausteine (Stämme, Affixe) und wie sie als Konstituenten morphologischer Strukturen (Wörter und Wortformen) fungieren.
Auf diesen beiden Ebenen waren auch wichtige klassifikatorische Aufgaben zu erledigen:
In der Phonologie hat es sich \zB als fruchtbringend erwiesen, die Segmente in bestimmte Klassen einzuteilen, die jeweils unterschiedliche Positionen in der Sonoritätshierarchie einnehmen (Abschnitt~\ref{sec:sonoritaet}).
In der Morphologie ist die Einteilung der Wörter in Klassen (Kapitel~\ref{sec:wortklassen}) eine Voraussetzung für eine systematische Beschreibung des Wortschatzes.
Würde man nicht definieren, was \zB Nomina und Verben sind, so wäre eine Darstellung dieser Wortklassen (wie in den Kapiteln~\ref{sec:nominalflexion} und~\ref{sec:verbalflexion}) nicht möglich.

In diesem Kapitel beginnt nun die Beschreibung der Regularitäten, nach denen Wortformen (also die Ergebnisse der Wortbildung und Flexion) zu größeren Strukturen (Gruppen, Satzgliedern, Sätzen) zusammengesetzt werden.
Dabei wird, wie in der Phonologie und Morphologie, eine hierarchische Struktur angenommen, also ein Aufbau von größeren syntaktischen Strukturen aus kleineren syntaktischen Teilstrukturen -- den \textit{Konstituenten}.
In der Phonologie haben wir Konstituentenstrukturen angenommen, indem \zB Silben als bestehend aus Anfangsrand, Kern und Endrand analysiert wurden.
Silben selber fügen sich zu den nächstgrößeren Einheiten -- den phonologischen Wörtern -- zusammen, vgl.\ das Beispiel in Abbildung~\ref{fig:syntaktischestruktur001}.

\begin{figure}[!htbp]
  \centering
  \begin{forest}
    [Silbe, calign=last
      [Anfangsrand, ake, calign=first
        [f][K]
      ]
      [Reim, calign=first
        [Kern,ake
          [E]
        ]
        [Endrand, ake, calign=last
          [m][t]
        ]
      ]
    ]
  \end{forest}
  \caption{Beispiel für Konstituentenstruktur in der Phonologie}
  \label{fig:syntaktischestruktur001}
\end{figure}

Auch in der Morphologie haben wir \zB bei der Bildung von Komposita angenommen, dass Komposita immer zwei Glieder haben, die aber wieder mit anderen Stämmen zu neuen Komposita verbunden werden können, so dass eine mehrschichtige Struktur entsteht.
Generell haben wir die gesamte Morphologie (also auch die Derivation und die Flexion) so dargestellt, dass die Konstituentenstruktur innerhalb einer Wortform eindeutig bestimmt werden kann, vgl.\ Abbildung~\ref{fig:syntaktischestruktur002} für die Wortform \textit{ver:säg-e-st} (Konjunktiv Präsens 2.~Person Singular).

\begin{figure}[!htbp]
  \begin{forest}
    [Verb-Wortform
      [{[Verb-Wortform]}
        [Verb-Stamm
          [Wortbildungspräfix, tier=preterm
            [\textit{ver-}]
          ]
          [Verb-Stamm, tier=preterm
            [\textit{säg-}]
          ]
        ]
        [Flexionssuffix, tier=preterm
          [\textit{-e}]
        ]
      ]
      [Flexionssuffix, tier=preterm
        [\textit{-st}]
      ]
    ]
  \end{forest}
  \caption{Beispiel für Konstituentenstruktur in der Morphologie}
  \label{fig:syntaktischestruktur002}
\end{figure}

Syntaktische Strukturen werden sehr komplex, und der Analyse der Struktur ist daher in der Syntax eine besonders große Aufmerksamkeit zu widmen.
Ganz ähnlich wie dem phonologischen Wort in Abbildung~\ref{fig:syntaktischestruktur001} und der Wortform in Abbildung~\ref{fig:syntaktischestruktur002} sollen Sätzen und Satzteilen Konstituentenstrukturen wie in Abbildung~\ref{fig:syntaktischestruktur004} zugewiesen werden.
Es handelt sich um das Baumdiagramm zum Satzteil (\ref{ex:syntaktischestruktur003}).

\begin{exe}
  \ex{\label{ex:syntaktischestruktur003} rote Zahnbürsten des Königs, die benutzt waren}
\end{exe}

\begin{figure}[!htbp]
  \centering
  \begin{forest}
    [NP, calign=child, calign child=2
      [AP, tier=preterminal
        [\textit{rote}]
      ]
      [N, tier=preterminal
        [\textit{Zahnbürsten}]
      ]
      [NP, tier=preterminal
        [\textit{des Königs}]
      ]
      [RS, tier=preterminal
        [\textit{die benutzt waren}, narroof]
      ]
    ]
  \end{forest}
  \caption{Vorschau auf Konstituentenstruktur in der Syntax}
  \label{fig:syntaktischestruktur004}
\end{figure}

Im Grunde verwenden wir auf allen Ebenen (Phonologie, Morphologie und Syntax) das gleiche Strukturformat.
Die höhere Komplexität der syntaktischen Struktur ist aber offensichtlich, zumal wenn man bedenkt, dass die Dreiecke in Abbildung~\ref{fig:syntaktischestruktur004} nur Abkürzungen für Teilstrukturen sind und teilweise selber vergleichsweise komplexe Bäume abkürzen.
Daher führen wir in diesem Kapitel einerseits in einige Tests ein, mit denen plausible syntaktische Strukturen heuristisch ermittelt werden können.
Andererseits werden Baupläne für syntaktische Konstituentenstrukturen genau angegeben -- und zwar wesentlich mehr als in der Phonologie und Morphologie.
Um den prinzipiellen hierarchischen Aufbau syntaktischer Struktur geht es jetzt zunächst in diesem Abschnitt.
In Abschnitt~\ref{sec:konstituenten} werden einige Tests beschrieben, die helfen können, plausible syntaktische Strukturen zu ermitteln.
In Abschnitt~\ref{sec:analysenvonkonstituentenstrukturen} wird überlegt, wie man die Reihenfolge von Teilkonstituenten in größeren Konstituenten und die hierarchische Struktur beschreiben kann.

\index{Grammatikalität}

Eine Grammatik ist gemäß Definition~\ref{def:grammatik} (S.~\pageref{def:grammatik}) ein System von Regularitäten, nach denen einfache sprachliche Einheiten zu komplexen Einheiten (Strukturen) zusammengesetzt werden.
Die syntaktische Komponente der Grammatik muss also spezifizieren, wie Sätze aus Wortformen (die in der Syntax die einfachsten Einheiten sind) aufgebaut werden, vgl.\ Definition~\ref{def:syntax}.

\Definition{Syntax}{\label{def:syntax}%
Die \textit{Syntax} formuliert die Generalisierungen, die genau die Sätze einer natürlichen Sprache (nicht mehr oder weniger oder andere Sätze) beschreiben.
Sie trennt zwischen grammatischen und ungrammatischen Sätzen, indem sie grammatischen Sätzen eine Struktur zuweist, ungrammatischen Sätzen aber nicht.
\index{Syntax}
}

Konkret muss eine Syntaxtheorie für das Deutsche also unter anderem feststellen, dass (\ref{ex:syntaktischestruktur006}) grammatisch ist, aber (\ref{ex:syntaktischestruktur007}) und (\ref{ex:syntaktischestruktur008}) ungrammatisch sind.

\begin{exe}
  \ex\label{ex:syntaktischestruktur005}
  \begin{xlist}
    \ex[]{\label{ex:syntaktischestruktur006} Ein Snookerball ist eine Kugel aus Kunststoff.}
    \ex[*]{\label{ex:syntaktischestruktur007} Eines Snookerballs ist eine Kugel aus Kunststoff.}
    \ex[*]{\label{ex:syntaktischestruktur008} Ein eine aus ist Snookerball Kugel Kunststoff.}
  \end{xlist}
\end{exe}

Die Syntax macht diese Unterscheidung dadurch, dass sie Generalisierungen formuliert, die einem Satz entweder eine Struktur (von der Kategorie Satz) zuweist oder nicht.
An den gegebenen Beispielen lässt sich das gut illustrieren.
Beispiel (\ref{ex:syntaktischestruktur006}) sollte sich durch die Syntax eine Struktur zuweisen lassen, die Wortkette sollte von der Grammatik also als Satz erkannt werden.
Anders verhält es sich mit Beispiel (\ref{ex:syntaktischestruktur007}).
Hier sollte sich zwar einigen Teilen eine Struktur zuweisen lassen, aber in der Syntax sollte es keine Regel geben, die diese zu einem ganzen Satz verbindet, da der Satz von Sprechern des Deutschen \idR nicht als akzeptabel eingestuft wird.
Konkret sind [\textit{Eines Snookerballs}] und [\textit{ist eine Kugel aus Kunststoff}] zwar Satzteile, aber sie bilden wegen des Kasus von [\textit{eines Snookerballs}] zusammen keinen Satz.
In (\ref{ex:syntaktischestruktur008}) gibt es nichtmal zwei Wörter, die sich in der gegebenen Reihenfolge zu einem Satzteil verbinden lassen.
Dadurch stellt sich die Frage, ob die gesamte Wortkette einen Satz ergibt, noch weniger als in (\ref{ex:syntaktischestruktur007}).

\begin{figure}[!htbp]
  \centering
  \begin{forest}
    [Satz
      [\it Ein]
      [\it Snookerball]
      [\it ist]
      [\it eine]
      [\it Kugel]
      [\it aus]
      [\it Kunststoff]
    ]
  \end{forest}
  \caption{Naives Satzschema}
  \label{fig:syntaktischestruktur009}
\end{figure}

Man braucht nun für die Grammatik schematische Beschreibungen von allen Ketten von Wörtern, die zusammen in einer bestimmten Reihenfolge Sätze ergeben.
Es hat angesichts des in Definition~\ref{def:syntax} formulierten Vorhabens aber wenig Sinn, Sätze in der Syntax einfach im Ganzen als Ketten von Wortformen zu beschreiben.
Täte man dies, so müsste eine Grammatik des Deutschen einen Bauplan enthalten, der das konkrete Beispiel (\ref{ex:syntaktischestruktur006}) beschreibt.
So ein naiver Bauplan für (\ref{ex:syntaktischestruktur006}) könnte aussehen wie Abbildung~\ref{fig:syntaktischestruktur009}.
Dieser Bauplan besagt, dass eine ganz bestimmte Abfolge von Wortformen (nämlich \textit{ein}, \textit{Snookerball} usw.) ein möglicher Satz ist.
Damit erzeugt oder beschreibt dieser Bauplan aber eben auch nur genau einen Satz.
Für alle anderen Sätze bräuchte man entsprechend andere Baupläne, und sie alle müssten Teil der Grammatik sein.
Auf diese Weise wäre das Erlernen der Baupläne, die die Sätze des Deutschen beschreiben, gleichbedeutend damit, alle Sätze des Deutschen auswendig zu lernen.
Da wir kontinuierlich Sätze produzieren, die wir noch niemals zuvor gehört haben, ist auszuschließen, dass ein solcher Ansatz besonders zielführend ist.

Selbst, wenn wir den Bauplan aus Abbildung~\ref{fig:syntaktischestruktur009} etwas abstrakter gestalten und nicht mehr die Wörter, sondern nur noch die Wortklassen der Konstituenten im Bauplan festlegen wie in Abbildung~\ref{fig:syntaktischestruktur010}, wird die Grammatik nicht viel allgemeiner.

\begin{figure}[!htbp]
  \centering
  \begin{forest}
    [Satz
      [Art]
      [Subst]
      [Kopula-Verb]
      [Art]
      [Subst]
      [Prp]
      [Subst]
    ]
  \end{forest}
  \caption{Abstrakteres naives Satzschema}
  \label{fig:syntaktischestruktur010}
\end{figure}

Der Bauplan in Abbildung~\ref{fig:syntaktischestruktur010} besagt, dass eine Folge von einem Artikelwort, einem Substantiv usw.\ ein möglicher Satz ist.
Er beschreibt damit immerhin schon wesentlich mehr Sätze als der in Abbildung~\ref{fig:syntaktischestruktur009}, \zB auch den in (\ref{ex:syntaktischestruktur011}).

\begin{exe}
  \ex{\label{ex:syntaktischestruktur011} Der Seitan ist eine Spezialität aus Weizeneiweiß.}
\end{exe}

Allerdings sind nur sehr wenige deutsche Sätze genau so aufgebaut.
Eine Korpusanfrage in einem Archiv des DeReKo-Korpus, das rund eine Milliarde Wörter umfasst, bringt insgesamt die vier Sätze in (\ref{ex:syntaktischestruktur012}) zu Tage.%
\footnote{Archiv W-TAGGED öffentlich am 11.01.2011.
Das Archiv enthielt zu diesem Zeitpunkt 1.024.793.751 Wortformen gemäß der Korpusansicht.
Siglen der Belege: RHZ09/JAN.17891, WPD/VVV.02704 AHZ, RHZ08/MAI.22154, M07/FEB.05680.}

\begin{exe}
  \ex \label{ex:syntaktischestruktur012}
  \begin{xlist}
    \ex{Die Verlierer sind die Schulkinder in Weyerbusch.}
    \ex{Die Vienne ist ein Fluss in Frankreich.}
    \ex{Ein Baustein ist die Begegnung beim Spiel.}
    \ex{Das Problem ist die Ortsdurchfahrt in Großsachsen.}
  \end{xlist}
\end{exe}

Der Bauplan erklärt also gerade einmal die Strukturen für 24 Wortformen aus einem Korpus von einer Milliarde Wortformen.
Bei dieser Erfolgsquote bräuchte man $10^9\div24\approx41,7\cdot10^6$ (über 40 Millionen) Satzschemata, um die Grammatik zu spezifizieren, die allen Sätzen im Korpus eine Struktur zuweist.%
\footnote{Dieses Rechenbeispiel ist methodisch sehr naiv und dient vor allem der Illustration und der argumentativen Zuspitzung.
Es ist \zB anzunehmen, dass nicht alle Schemata gleich häufig wären, und dass andere Schemata für wesentlich mehr bzw.\ sogar weniger Sätze geeignet wären.
Auf jeden Fall wären es aber extrem viele Schemata.}

Es gibt extrem viele verschiedene Arten, Wörter zu einem Satz zusammenzusetzen, dass Baupläne, die Sätze als Reihen von Wortformen beschreiben, nicht allgemein genug sind.
Viel effektiver ist die Annahme, dass in der Syntax nicht Wortformen zu Sätzen, sondern Wortformen zu Gruppen zusammengesetzt werden, die wiederum Gruppen bilden, bis hin zur Ebene des Satzes.
Diese kleineren Strukturen sind wesentlich allgemeiner beschreibbar als ganze Sätze, und nur so kommt die nötige Abstraktion zustande, um mit relativ wenigen Schemata sehr viele Arten von Sätzen zu beschreiben.
Auch aus Sicht der kognitiven Verarbeitung von Sprache durch Sprecher ist es plausibel, anzunehmen, dass sprachliche Informationen in Strukturen verpackt werden, die durch ihren hierarchischen Aufbau mit möglichst geringem Aufwand produziert und verstanden werden können.

Als Beispiel diskutieren wir nun, wie eine entsprechend abstraktere Analyse der Schemata in Satz (\ref{ex:syntaktischestruktur006}) aussehen könnte, und welchen Vorteil man dadurch erzielt.
Wenn man einige strukturell ähnliche Sätze zu (\ref{ex:syntaktischestruktur006}) und (\ref{ex:syntaktischestruktur011}) hinzunimmt -- nämlich die in (\ref{ex:syntaktischestruktur013}) --, kommt man schnell auf einen allgemeinen Bauplan.

\begin{exe}
  \ex\label{ex:syntaktischestruktur013}
  \begin{xlist}
    \ex{\label{ex:syntaktischestruktur014} [Dieses Endspiel] ist [eine spannende Partie].}
    \ex{\label{ex:syntaktischestruktur015} [Eine Hose] war [eine Hose].}
    \ex{\label{ex:syntaktischestruktur016} [Sieger] wurde [ein Teilnehmer aus dem Vereinigten Königreich].}
    \ex{\label{ex:syntaktischestruktur017} [Lemmy] ist [Ian Kilmister].}
  \end{xlist}
\end{exe}

In allen Sätzen in (\ref{ex:syntaktischestruktur013}) steht jeweils eine NP (ggf.\ etwas erweitert, wie im Fall von \textit{ein Teilnehmer aus dem Vereinigten Königreich}) am Anfang und am Ende, dazwischen steht eine Form der Kopulaverben \textit{sein} und \textit{werden}.
Obwohl sie unterschiedlich aufgebaut sind, verhalten sich die NPs im Satz alle gleich.
Wenn man nun also die Bildung dieser NPs möglichst allgemein beschreibt, kann man sich im Bauplan des Satzes auf diese Beschreibung beziehen, ohne auf mögliche verschiedene Strukturen, die NPs intern haben können, dort noch eingehen zu müssen.%
\footnote{Schon in Kapitel~\ref{sec:nominalflexion} (Definition~\ref{def:vollstaendigesnominal} auf S.~\pageref{def:vollstaendigesnominal}) wurde die (vereinfachte) NP als eine Folge von kongruierendem Artikel, (optionalem) Adjektiv und Substantiv bezeichnet.
Um auch Fälle wie \textit{ein Teilnehmer aus dem Vereinigten Königreich} zu erfassen, erweitern wir später die Definition.}
Genau daraus ergibt sich ein Satzbauschema wie in Abbildung~\ref{fig:syntaktischestruktur018} und eine konkrete hierarchische Struktur wie in Abbildung~\ref{fig:syntaktischestruktur019}.
Diese Abbildung ist nur ein Vorschlag.
Genaues folgt vor allem in den Kapiteln~\ref{sec:phrasen} und~\ref{sec:saetze}.
Jetzt müsste nur noch ein genauer allgemeiner Bauplan für die NP angegeben werden, was aber ebenfalls verschoben wird (Schema~\ref{str:ngr} auf S.~\pageref{str:ngr}).

\begin{figure}[!htbp]
  \centering
  \begin{forest}
    [Satz
      [NP]
      [Kopula-Verb]
      [NP]
    ]
  \end{forest}
  \caption{Hypothetisches Schema für Sätze mit Kopula}
  \label{fig:syntaktischestruktur018}
\end{figure}

\begin{figure}[!htbp]
  \centering
  \begin{forest}
    [Satz
      [NP
        [\it Dieses Endspiel, narroof]
      ]
      [Kopula-Verb
        [\it ist]
      ]
      [NP
        [\it eine spannende Partie, narroof]
      ]
    ]
  \end{forest}
  \caption{Denkbare hierarchische Struktur eines Kopulasatzes}
  \label{fig:syntaktischestruktur019}
\end{figure}

Wichtig ist nun die Erkenntnis, dass es durch die Abstraktion von den verschiedenen Arten von NP im Satzbauschema egal ist, wie die NP selber aufgebaut sind.
Ob die NP nur aus einem Substantiv besteht wie \textit{Sieger} in (\ref{ex:syntaktischestruktur016}) oder aus Substantiv und Artikel wie \textit{eine Hose} in (\ref{ex:syntaktischestruktur015}) oder aus Substantiv, Artikel und Adjektiv wie \textit{eine spannende Partie} in (\ref{ex:syntaktischestruktur014}) usw.\ ist belanglos für die Anwendung des Satzbauplans in Abbildung~\ref{fig:syntaktischestruktur018}.
Der Bauplan verlangt nur, dass irgendeine NP als Konstituente eingesetzt wird, egal wie diese aussieht.
Wir müssen also überlegen, wie sich syntaktische Strukturen effektiv in kleinere Einheiten aufteilen lassen (also eine Konstituentenanalyse oder Satzgliedanalyse betreiben), und die entsprechenden Baupläne angeben.

\index{Rekursion!in der Syntax}

Dieses Vorgehen verdeutlicht im Übrigen auch ein gewisses Maß an \textit{Rekursion}, wie wir sie schon in der Morphologie (Abschnitt~\ref{sec:rekursion}) besprochen haben.
Auch Strukturen wie in (\ref{ex:syntaktischestruktur020}) -- eine Wiederholung von (\ref{ex:syntaktischestruktur016}) -- kann und sollte man als eine NP betrachten.

\begin{exe}
  \ex{\label{ex:syntaktischestruktur020} [ein Teilnehmer aus dem Vereinigten Königreich]}
\end{exe}

In dieser NP ist allerdings eine weitere NP eingebettet, nämlich [\textit{dem Vereinigten Königreich}].
Es gibt keinen Grund, anzunehmen, dass diese NP nicht wieder eine NP enthalten könnte, usw.
Wie in der Morphologie kann also das Ergebnis einer strukturbildenden Operation wieder für dieselbe Operation als Ausgangsmaterial verwendet werden.
Ähnlich und noch einfacher ist (\ref{ex:syntaktischestruktur022}).
Hier kann eine ebenfalls rekursive Koordinationsstruktur beliebig fortgesetzt werden.\label{abs:syntaktischestruktur021}

\begin{exe}
  \ex{\label{ex:syntaktischestruktur022} Dieser Wagen läuft und läuft und läuft und läuft\ldots}
\end{exe}

Dieser Satz wird angeblich nicht ungrammatisch, egal wie oft man \textit{und läuft} wiederholt.
Manchmal wird dies als Beweis genommen, dass es im Prinzip unendlich viele verschiedene Sätze in einer Sprache gibt, im Minimalfall durch endlose Koordination wie in (\ref{ex:syntaktischestruktur022}).
Ein solcher Beweis ist allerdings in Wirklichkeit nicht zu führen, und er beruht auf der Idee einer strikten Trennung zwischen den Möglichkeiten, die das Sprachsystem anbietet (\textit{Kompetenz}) und den Bedingungen, unter denen wir es benutzen (\textit{Performanz}), auf die wir in einem deskriptiven Rahmen nicht eingehen können und müssen.\index{Kompetenz}\index{Performanz}
Eine klare Begrenzung der Rekursion für den Menschen ist normalerweise ganz einfach schon dadurch gegeben, dass sehr lange Sätze schlicht nicht mehr verarbeitet werden können.
Inwiefern uns jetzt die Feststellung, dass aber \textit{im Prinzip} doch unendlich viele Sätze möglich wären, weiterbringt, ist fraglich.
Wir bleiben hier bescheiden und stellen fest, dass eingeschränkt rekursive Strukturen vorkommen (\zB NPs in NPs), und dass das syntaktische System offensichtlich so gebaut ist, dass wir ständig auf ziemlich viele Sätze treffen, die wir vorher noch nie gehört haben.

\Zusammenfassung{%
Die Syntax (als wissenschaftliche Disziplin) versucht, mit so wenig wie möglich Generalisierungen alle Sätze einer Sprache zu beschreiben.
Wenn eine Syntax eine gegebene Folge von Wörtern auf Basis ihrer Generalisierungen als Satz beschreiben kann, gilt der Satz relativ zu dieser Syntax als grammatisch.
Idealerweise klassifiziert die Syntax diejenigen Sätze als grammatisch, die auch von Sprechern als akzeptabel klassifiziert werden.
Sätze in der Syntax als Folgen von Wörtern zu beschreiben, ist nicht zielführend, weil es viel zu viele verschiedene Arten von Wortfolgen gibt, die grammatische Sätze sind.
Man beschreibt zunächst die Struktur kleinerer Konstituenten, aus denen dann größere Konstituenten und schließlich Sätze aufgebaut werden.
}

\section{Konstituenten}
\label{sec:konstituenten}

\index{Konstituententest}

In Abschnitt~\ref{sec:syntaktischestruktur} wurde von der komplexen hierarchischen Struktur in der Syntax gesprochen, ohne dass gezeigt wurde, nach welchen Methoden Syntaktiker sich auf ganz konkrete Strukturen zu einigen versuchen.
Um herauszufinden, was eventuell als eine syntaktische Konstituente behandelt werden sollte, gibt es eine Reihe von Tests, die hier jetzt besprochen werden.
Ein Warnhinweis ist vor der Einführung dieser Tests dringend erforderlich:
Die Tests funktionieren nicht immer so, wie Syntaktiker es sich wünschen.
Teilweise identifizieren sie Wortgruppen als Konstituenten, die dann doch nicht als Konstituenten betrachtet werden.
Andererseits gibt es Fälle, in denen etwas, das gemeinhin als Konstituente betrachtet wird, nur von wenigen Tests oder sogar keinem als Konstituente identifiziert wird.
Die Tests sind also nur als heuristisches Verfahren anzusehen.
Wenn sich im Laufe der Theoriebildung (im Sinne der Formulierung bzw.\ Formalisierung einer Syntax) herausstellt, dass es günstiger ist, in einigen Fällen die Ergebnisse der Tests nicht ernstzunehmen, ist dies unproblematisch.
Gerade wenn eine Grammatik formal ausgearbeitet ist, kann sie jederzeit einfach daran gemessen werden, ob sie Sätze korrekt als grammatisch oder ungrammatisch klassifiziert (vgl.\ Definition~\ref{def:syntax} auf S.~\pageref{def:syntax}).
Die Konstituentenstrukturen selbst sind Konstrukte unserer Theorie und nicht direkt beobachtbare Objekte.

\subsection{Konstituententests}
\label{sec:konstituententests}

Im Folgenden werden drei wichtige Konstituententests besprochen.
Die Tests beinhalten alle eine Umformung des ursprünglichen Materials (Hinzufügung, Umstellung, Austausch).
Die Testanwendung markieren wir mit \KTArr{Name\ des\ Tests}.
Davor steht der Ausgangssatz und dahinter der umgeformte Satz.
Die Umformung muss grammatisch sein, und in den meisten Fällen muss die Bedeutung erhalten bleiben.
Wenn ein Test fehlschlägt, steht hinter dem Pfeil ein Asterisk~\Ast.
Wir beginnen mit dem \textit{Pronominalisierungstest} (Definition~\ref{def:prontest}).

\Definition{Pronominalisierungstest (PronTest)}{\label{def:prontest}%
Wenn eine Kette von Wörtern in einem Satz durch einen Pronominalausdruck ersetzt werden kann, dann ist sie eine Konstituente.
\index{Pronominalisierungstest}
}

Beim Pronominalisierungstest sind der Ausgangssatz und der Satz mit der Ersetzung (der Testsatz) nicht bedeutungsgleich, denn durch die Ersetzung ist der Testsatz normalerweise nicht mehr situationsunabhängig eindeutig zu interpretieren.
Beispiele folgen in (\ref{ex:konstituententests023}).

\begin{exe}
  \ex\label{ex:konstituententests023}
  \begin{xlist}
    \ex{\label{ex:konstituententests024} Mausi isst [den leckeren Marmorkuchen]. \KTArr{PronTest} Mausi isst [ihn].}
    \ex{\label{ex:konstituententests025} [Mausi isst] den Marmorkuchen. \KTArr{PronTest} \Ast [Sie] den Marmorkuchen.}
  \end{xlist}
\end{exe}

Offensichtlich ist [\textit{den leckeren Marmorkuchen}] gemäß dem Pronominalisierungstest eine Konstituente, aber [\textit{Mausi isst}] ist keine.
Auch deutlich komplexere Strukturen (\zB mit Konjunktionen) können erfolgreich ersetzt werden, wie in (\ref{ex:konstituententests026}).

\begin{exe}
  \ex{\label{ex:konstituententests026} Mausi isst [den Marmorkuchen und das Eis mit Multebeeren].\\\KTArr{PronTest} Mausi isst [sie].}
\end{exe}

Typischerweise werden Wörter aus der Klasse der Pronomina eingesetzt.
Aber auch andere Arten von Konstituenten können durch Wörter wie \textit{da}, \textit{dann}, \textit{so} usw.\ ersetzt werden, s.\ (\ref{ex:konstituententests027}) und (\ref{ex:konstituententests028}).
Bei diesem Test wird also immer ein deiktisches oder anaphorisches Wort (vgl.\ Definition~\ref{def:deixis} auf S.~\pageref{def:deixis} und Definition~\ref{def:anaphern} auf S.~\pageref{def:anaphern}) statt einer semantisch spezifischen Konstituente eingesetzt.
Diese Wörter sind hier mit \textit{Pronominalausdruck} gemeint.

\begin{exe}
  \ex{\label{ex:konstituententests027} Ich treffe euch [am Montag] [in der Mensa der FU].\\
    \KTArr{PronTest} Ich treffe euch [dann] [dort].}
  \ex{\label{ex:konstituententests028} Er liest den Text [auf eine Art, die ich nicht ausstehen kann].\\\KTArr{PronTest} Er liest den Text [so].}
\end{exe}

Als nächstes folgt der \textit{Vorfeldtest} in Definition~\ref{def:vftest}.

\Definition{Vorfeldtest (VfTest)}{\label{def:vftest}%
Wenn eine Kette von Wörtern in einem Satz vorfeldfähig ist, dann ist sie eine Konstituente.
\index{Vorfeldtest}
}

Dieser Test bezieht sich auf die Definition von Vorfeldfähigkeit (Definition~\ref{def:vorfeldfaehig} auf S.~\pageref{def:vorfeldfaehig}).
Dort wurde die Vorfeldfähigkeit einzelner Wörter benutzt, um Adverben und Partikeln definitorisch voneinander zu trennen.
Hier geht es nicht nur um einzelne Wortformen, sondern auch um komplexere Konstituenten.
Vorfeldfähig ist eine Konstituente genau dann, wenn sie alleine vor dem finiten Verb stehen kann.
Bei der Anwendung dieses Tests auf sprachliches Material muss man ggf.\ also eine strukturelle Veränderung durchführen, wenn die zu untersuchende Konstituente nicht ohnehin schon alleine vor dem finiten Verb steht.
Wichtig ist, dass sich die Bedeutung nicht ändern darf, und dass kein Material weggelassen oder hinzugefügt werden darf.

\begin{exe}
  \ex\label{ex:konstituententests029}
  \begin{xlist}
    \ex{\label{ex:konstituententests030} Sarah sieht den Kuchen [durch das Fenster].\\
      \KTArr{VfTest} [Durch das Fenster] sieht Sarah den Kuchen.}
    \ex{\label{ex:konstituententests031} Er versucht [zu essen]. \KTArr{VfTest} [Zu essen] versucht er.}
    \ex{\label{ex:konstituententests032} Sarah möchte gerne [einen Kuchen backen].\\
      \KTArr{VfTest} [Einen Kuchen backen] möchte Sarah gerne.}
    \ex{\label{ex:konstituententests033} Sarah möchte [gerne einen] Kuchen backen.\\
      \KTArr{VfTest} \Ast [Gerne einen] möchte Sarah Kuchen backen.}
  \end{xlist}
\end{exe}

Dieser Test bereitet Schwierigkeiten, wenn das finite Verb des Hauptsatzes nicht richtig ermittelt wird.
In den Sätzen in (\ref{ex:konstituententests034}) ist trotz großer oberflächlicher Ähnlichkeit das finite Verb des Hauptsatzes jeweils ein anderes finites Verb an zwei völlig verschiedenen Stellen.
In (\ref{ex:konstituententests035}) ist \textit{glaubt} das finite Verb des Hauptsatzes, in (\ref{ex:konstituententests036}) ist es \textit{irrt}.

\begin{exe}
  \ex\label{ex:konstituententests034}
  \begin{xlist}
    \ex{\label{ex:konstituententests035} [Wer] glaubt, dass Tiere im Tierheim ein schönes Leben haben?}
    \ex{\label{ex:konstituententests036} [Wer glaubt, dass Tiere im Tierheim ein schönes Leben haben], irrt.}
  \end{xlist}
\end{exe}

(\ref{ex:konstituententests036}) ist ein Beispiel, das auch ohne Umstellung (also ohne Anwendung des Tests) zeigt, dass [\textit{wer glaubt, dass Tiere im Tierheim ein schönes Leben haben}] eine Konstituente ist, weil es sowieso schon im Vorfeld steht.
Um dies zu erkennen, darf aber \textit{glaubt} auf keinen Fall fälschlicherweise als finites Verb des Hauptsatzes identifiziert werden.
Zu diesem Problem kann hier nur auf Kapitel~\ref{sec:saetze} (besonders Abschnitt~\ref{sec:eingebettetenebensaetzeundderlsktest}) verwiesen werden, in dem Diagnoseverfahren für das sogenannte Feldermodell angegeben werden.

Den Vorfeldtest kann man im Prinzip zu einem \textit{Bewegungstest} verallgemeinern, denn im Deutschen können auch innerhalb des Satzes Konstituenten relativ leicht umgestellt werden (\textit{Scrambling}, s.\ Abschnitt~\ref{sec:verbphrase}).
In (\ref{ex:konstituententests037}) werden die drei Konstituenten zwischen \textit{hat} und \textit{gewonnen} bewegt.
Sie sind zur Verdeutlichung in [~] gesetzt.
Dass diese Tests im Deutschen funktionieren, illustriert im Übrigen die enorm flexible Wortstellung des Deutschen.

\begin{exe}
  \ex\label{ex:konstituententests037}
  \begin{xlist}
    \ex{\label{ex:konstituententests038} Gestern hat [Elena] [im Turmspringen] [die Goldmedaille] \\gewonnen.}
    \ex{\label{ex:konstituententests039} Gestern hat [im Turmspringen] [Elena] [die Goldmedaille] \\gewonnen.}
    \ex{\label{ex:konstituententests040} Gestern hat [im Turmspringen] [die Goldmedaille] [Elena] \\gewonnen.}
  \end{xlist}
\end{exe}

Als dritten und letzten Test betrachten wir den \textit{Koordinationstest} in Definition~\ref{def:koortest}.

\Definition{Koordinationstest (KoorTest)}{\label{def:koortest}%
Wenn eine Kette von Wörtern in einem Satz mit einer anderen Kette von Wörtern und einer Konjunktion (\zB \textit{und}, \textit{oder}) verbunden werden kann, dann ist sie eine Konstituente.
\index{Koordinationstest}
}

Der Name des Koordinationstests kommt daher, dass man Strukturen, die das Muster [A Konjunktion B] haben, \textit{Koordinationen} oder \textit{Koordinationsstrukturen} nennt.
Da man bei diesem Test Material hinzufügen muss, muss sich zwangsläufig die Bedeutung ändern.
Dieser Test ermittelt erfolgreich alles als Konstituente, was man normalerweise auch als eine solche auffasst.
Außerdem zeigt er gleichzeitig, dass die Wortkette, die man hinzufügt, ebenfalls eine Konstituente ist, und dass die gesamte Koordinationsstruktur auch eine Konstituente ist.
Daher klammern wir immer \zB [[A] \textit{und} [B]].
In (\ref{ex:konstituententests041}) finden sich Beispiele.

\begin{exe}
  \ex\label{ex:konstituententests041}
  \begin{xlist}
    \ex{\label{ex:konstituententests042} Wir essen [einen Kuchen].\\
      \KTArr{KoorTest} Wir essen [[einen Kuchen] und [ein Eis]].}
    \ex{\label{ex:konstituententests043} Wir [essen einen Kuchen].\\
      \KTArr{KoorTest} Wir [[essen einen Kuchen] und [lesen ein Buch]].}
    \ex{\label{ex:konstituententests044} Sarah hat versucht, [einen Kuchen zu backen].\\
      \KTArr{KoorTest} Sarah hat versucht, [[einen Kuchen zu backen] und \\{}[heimlich das Eis aufzuessen]].}
    \ex{\label{ex:konstituententests045} Wir sehen, [dass die Sonne scheint].\\
      \KTArr{KoorTest} Wir sehen, [[dass die Sonne scheint] und \\{}[wer alles seinen Rasen mäht]].}
    \ex{\label{ex:konstituententests046} Wir sehen, dass [die Sonne scheint].\\
      \KTArr{KoorTest} Wir sehen, dass [[die Sonne scheint] und \\{}[Mausi den Rasen mäht]].}
  \end{xlist}
\end{exe}

Wie oben gesagt, ist der Koordinationstest im Grunde in allen Fällen erfolgreich, in denen man dies auch möchte.
Leider ist er gleichzeitig der Test, der wahrscheinlich auch die meisten Fehler produziert, bei denen Wortketten als Konstituenten ausgewiesen werden, die dies nach allgemeiner Auffassung nicht sind.
Man kann eine volle Koordinationsstruktur nicht immer von einer Struktur unterscheiden, in der durch \textit{Ellipse} (Auslassung) ein Wort oder mehrere Wörter getilgt wurden, die die Konstituente vervollständigen würden.
So ist \zB (\ref{ex:konstituententests047}) ein Beispiel, in dem der Test erfolgreich ist, es aber idealerweise nicht sein sollte.

\begin{sloppypar}
\begin{exe}
  \ex{\label{ex:konstituententests047} Der Kellner notiert, dass [meine Kollegin einen Salat] möchte.\\
    \KTArr{KoorTest} Der Kellner notiert, dass [[meine Kollegin einen Salat] und\\{}[mein Kollege einen Sojaburger]] möchte.}
\end{exe}
\end{sloppypar}

Die meisten Syntaktiker würden Wortfolgen wie \zB [\textit{meine Kollegin einen Salat}] und [\textit{mein Kollege einen Sojaburger}] nicht als eine Konstituente betrachten, sondern als zwei, die höchstens zusammen mit einem Verb eine \textit{Verbphrase} bilden könnten (s.\ Abschnitt~\ref{sec:verbphraseundverbkomplex}).
Das \textit{und} kann dann hier so analysiert werden, dass es die vollständige Verbphrase [\textit{mein Kollege einen Sojaburger möchte}] und die unvollständige [\textit{meine Kollegin einen Salat}] koordiniert.
In der ersten Verbphrase findet dabei eine Ellipse (Weglassung) des Verbs statt, um eine Wiederholung zu vermeiden.\index{Ellipse}
Der Test wird damit aber durch theoriespezifische Zusatzannahmen modifiziert, die an gegebenen Sätzen nicht immer leicht umzusetzen sind.

Bei der Anwendung des Koordinationstests kann es außerdem zu Fehlern kommen, weil nicht eindeutig erkennbar ist, welche potentiellen Konstituenten genau koordiniert werden.
In (\ref{ex:konstituententests048}) ist genau so ein Fall illustriert.

\begin{exe}
  \ex{\label{ex:konstituententests048} Sie isst [einen leckeren großen Kuchen].}
  \begin{xlist}
    \ex{\label{ex:konstituententests049} \KTArr{KoorTest} Sie isst [[einen leckeren großen Kuchen] und \\{}[eine Orange]].}
    \ex{\label{ex:konstituententests050} \KTArr{KoorTest} \Ast Sie isst [[einen leckeren großen Kuchen] und \\{}[geht später joggen]].}
  \end{xlist}
\end{exe}

Die Beispiele in (\ref{ex:konstituententests048}) sehen so aus, als könne man zwei völlig verschiedene Dinge mit [\textit{einen leckeren großen Kuchen}] koordinieren, nämlich [\textit{eine Orange}] und [\textit{geht später joggen}].
Die vermeintliche Koordination mit [\textit{geht später joggen}] ist eine ungünstige Annahme.
Obwohl der Satz in (\ref{ex:konstituententests050}) als Folge von Wortformen völlig grammatisch ist, ist die Klammerung in (\ref{ex:konstituententests050}) nicht plausibel.
Eigentlich wird in diesem Fall nämlich das erste Verb \textit{isst} in die Koordination einbezogen, und die Klammerung müsste wie in (\ref{ex:konstituententests051}) gesetzt werden.
Wie man solche Fälle entscheidet, wird in den folgenden Kapiteln klar werden.

\begin{exe}
  \ex{\label{ex:konstituententests051} Sie [[isst einen leckeren großen Kuchen] und [geht später joggen]].}
\end{exe}

\subsection{Konstituenten und Satzglieder}
\label{sec:konstituentenundsatzglieder}

Damit sind einige wichtige Tests auf Konstituenz eingeführt.
Eine Unterscheidung zwischen verschiedenen Satz-Konstituenten, die in der Schulgrammatik eine größere Rolle spielt, kann man mit den Tests allerdings auch noch zeigen.
Die Sätze in (\ref{ex:konstituentenundsatzglieder052}) und (\ref{ex:konstituentenundsatzglieder055}) illustrieren das Phänomen.

\begin{sloppypar}
\begin{exe}
  \ex\label{ex:konstituentenundsatzglieder052}
  \begin{xlist}
    \ex{\label{ex:konstituentenundsatzglieder053} Sarah riecht den Kuchen [mit ihrer Nase].\\
      \KTArr{VfTest} [Mit ihrer Nase] riecht Sarah den Kuchen.}
    \ex{\label{ex:konstituentenundsatzglieder054} \KTArr{KoorTest} Sarah riecht den Kuchen [[mit ihrer Nase] und \\{}[trotz des Durchzugs]].}
  \end{xlist}
  \ex\label{ex:konstituentenundsatzglieder055}
  \begin{xlist}
    \ex{\label{ex:konstituentenundsatzglieder056} Sarah riecht den Kuchen [mit der Sahne].\\
      \KTArr{VfTest} \Ast [Mit der Sahne] riecht Sarah den Kuchen.}
    \ex{\label{ex:konstituentenundsatzglieder057} \KTArr{KoorTest} Sarah riecht den Kuchen [[mit der Sahne] und \\{}[mit den leckeren Rosinen]].}
  \end{xlist}
\end{exe}
\end{sloppypar}

Beide Ausgangssätze sehen zunächst strukturell identisch aus.
Der Koordinationstest gelingt auch in beiden Fällen, aber der Vorfeldtest scheitert in (\ref{ex:konstituentenundsatzglieder056}).
Dies hat nun nicht etwa rein semantische Gründe, sondern strukturelle.
In (\ref{ex:konstituentenundsatzglieder052}) ist [\textit{mit ihrer Nase}] ein \textit{Satzglied} des Satzes, und in (\ref{ex:konstituentenundsatzglieder055}) ist [\textit{mit der Sahne}] dann eben kein Satzglied des Satzes.
Manchmal sagt man, die Satzglieder seien die \textit{unmittelbaren Konstituenten} des Satzes und die Nicht-Satzglieder seien mittelbare Konstituenten (vgl.\ Abschnitt~\ref{sec:strukturbildung} zu diesen Begriffen).
Vereinfacht sähen die Strukturen also so aus wie in den Abbildungen~\ref{fig:konstituentenundsatzglieder058} und~\ref{fig:konstituentenundsatzglieder059}.

\begin{figure}[!htbp]
  \centering
  \begin{forest}
    [Satz
      [\it Sarah]
      [\it riecht]
      [\it den Kuchen]
      [\it mit ihrer Nase]
    ]
  \end{forest}
  \caption{Ein Satzglied als unmittelbare Satzkonstituente}
  \label{fig:konstituentenundsatzglieder058}
\end{figure}

\begin{figure}[!htbp]
  \centering
  \begin{forest}
    [Satz
      [\it Sarah, tier=term]
      [\it riecht, tier=term]
      [Konstituente
        [\it den Kuchen, tier=term]
        [\it mit der Sahne, tier=term]
      ]
    ]
  \end{forest}
  \caption{Ein Nicht-Satzglied als mittelbare Satzkonstituente}
  \label{fig:konstituentenundsatzglieder059}
\end{figure}

Die Auffassung, Satzglieder seien die unmittelbaren Konstituenten des Satzes, bringt einige Probleme mit sich.
Später (Kapitel~\ref{sec:phrasen} und~\ref{sec:saetze}) werden wir aus gutem Grund Strukturen annehmen, die anders aussehen, und in denen Satzglieder nicht automatisch unmittelbare Konstituenten des Satzes sind.
Auf jeden Fall ist aber die Erkenntnis korrekt, dass Nicht-Satzglieder normalerweise strukturell zu tief eingebettet sind, um \zB vorangestellt (oder erfragt) werden zu können.
Wir definieren die Satzglieder also nicht als unmittelbare Konstituenten des Satzes, sondern sind etwas vorsichtiger, s.\ Definition~\ref{def:satzglied}.

\Definition{Satzglied}{\label{def:satzglied}%
Ein \textit{Satzglied} ist eine Konstituente im Satz, die vorfeldfähig ist.
\index{Satzglied}
}

Die Definition ist nicht ganz wasserdicht, weil auch Material ins Vorfeld gestellt werden kann, das traditionell nicht als Satzglied angesehen wird.
Auf S.~\pageref{abs:daspraedikat020} gibt es dafür das Beispiel (\ref{ex:daspraedikat021}) und eine kurze Diskussion.
Vgl.\ auch Übung~\ref{exc:konstituentenstruktur03}.
Der Begriff wurde hier vor allem wegen seiner Relevanz in der Didaktik aufgenommen.
Eine weitere besondere Art von Konstituenten brauchen wir übrigens nicht erst zu testen, weil sie als trivial gegeben angesehen werden kann: die Wortform (vgl.\ schon Abschnitt~\ref{sec:definitionsprobleme}).
Die Wortform als minimale syntaktische Einheit wird in Definition~\ref{def:atomarekonstituente} eingeführt.

\Definition{Atomare syntaktische Konstituenten}{\label{def:atomarekonstituente}%
Die \textit{atomaren syntaktischen Konstituenten} (die kleinsten nicht weiter analysierbaren Einheiten in der Syntax) sind die syntaktischen Wörter.
\index{Konstituente!atomar}
}

Definition~\ref{def:atomarekonstituente} sagt also aus, dass unabhängig davon, wie komplex die hierarchische Struktur eines Satzes ist, jeder Satz letztendlich aus Wörtern besteht.
Diese Wörter können sehr mittelbare (indirekte) Konstituenten sein, aber sie sind immer Konstituenten.
Im Gegensatz dazu sind Segmente, Silben, Stämme oder Suffixe keine (auch nicht atomaren) Konstituenten des Satzes, weil die Regularitäten, nach denen sie zusammengefügt werden, nicht die der Syntax sind.

Damit haben wir eine Reihe von Tests an der Hand, die nicht nur Konstituenten an sich ermitteln, sondern sogar unterschiedliche Status von Konstituenten aufzeigen können.
Wenn mit diesen Tests Konstituentenstrukturen ermittelt wurden, können sie in der Syntax als allgemeine Baupläne kodiert werden, wozu irgendeine Art von Formalismus benötigt wird.
Wir verwenden hier keinen rigiden Formalismus, sondern machen uns nur möglichst vollständige Gedanken über lineare Abfolgen und eventuell nötige minimale hierarchische Gliederungen von Konstituenten.
Mit der Annahme, dass in der Syntax hierarchische Konstituentenstrukturen aufgebaut werden, lassen sich einige interessante Phänomene erklären.
Einem davon, sogenannten \textit{strukturellen Ambiguitäten}, wenden wir uns im nächsten Abschnitt zu.

\subsection{Strukturelle Ambiguität}
\label{sec:strukturelleambiguitaet}

Nehmen wir einen Satz wie (\ref{ex:strukturelleambiguitaet060}).

\begin{exe}
  \ex{\label{ex:strukturelleambiguitaet060} Scully sieht den Außerirdischen mit dem Teleskop.}
\end{exe}

Dieser Satz hat zwei mögliche Lesarten.
Einerseits kann das Teleskop das Werkzeug sein, dass Scully benutzt, um den Außerirdischen sehen zu können.
Andererseits beschreibt der Satz auch eine Situation, in der der Außerirdische ein Teleskop dabei hat und Scully ihn ohne Hilfsmittel sieht.
Dieser Bedeutungsunterschied kann nun auf einen syntaktischen zurückgeführt werden, die Analysen sind in (\ref{ex:strukturelleambiguitaet061}) gegeben.

\begin{exe}
  \ex\label{ex:strukturelleambiguitaet061}
  \begin{xlist}
    \ex{\label{ex:strukturelleambiguitaet062} [Scully sieht [den Außerirdischen] [mit dem Teleskop]].}
    \ex{\label{ex:strukturelleambiguitaet063} [Scully sieht [den Außerirdischen mit dem Teleskop]].}
  \end{xlist}
\end{exe}

Im ersten Fall bildet \textit{den Außerirdischen mit dem Teleskop} keine Konstituente, sondern [\textit{den Außerirdischen}] und [\textit{mit dem Teleskop}] sind separate Satzglieder.
Im zweiten Fall ist [\textit{den Außerirdischen mit dem Teleskop}] als Ganzes ein Satzglied, das [\textit{mit dem Teleskop}] als Teilkonstituente enthält.
Solche strukturellen Ambiguitäten kommen häufig vor, und alle möglichen Analysen sind aus Sicht der Grammatik jeweils gleichberechtigt, wenn vielleicht auch eine aus rein inhaltlichen Gründen als die naheliegende erscheint und oft die alternativen Analysen deswegen übersehen werden.
Es wird Definition~\ref{def:strambi} gegeben.

\Definition{Strukturelle Ambiguität}{\label{def:strambi}%
\textit{Strukturelle Ambiguität} liegt dann vor, wenn ein Satz mehrere mögliche Konstituentenanalysen hat.
Oft hat dies (wegen des Kompositionalitätsprinzips) auch eine Doppeldeutigkeit in der Bedeutung zur Folge.
\index{Ambiguität}
}

Im nächsten Abschnitt wird abschließend die Art und Weise vorgestellt, mit der in den folgenden Kapiteln die deskriptiven Generalisierungen über die Konstituentenstrukturen des Deutschen notiert werden.

\Zusammenfassung{%
Die Konstituententests sind eine Heuristik, mit deren Hilfe man sich einer zielführenden Konstituentenanalyse annähern kann.
Sie stellen keine notwendige oder hinreichende empirische Grundlage für die Syntax dar.
Selbständig bewegbare und vorfeldfähige Konstituenten werden Satzglieder genannt.
Eine Folge von Wörtern kann strukturell ambig sein, also mehr als eine angemessene syntaktische Analyse haben.
}

\section{Analysen von Konstituentenstrukturen}
\label{sec:analysenvonkonstituentenstrukturen}

\subsection{Terminologie für Baumdiagramme}
\label{sec:terminologiefuerbaumdiagramme}

\index{Baumdiagramm}
\index{Ast}
\index{Kante}
\index{Knoten}

Da jetzt vermehrt Baumdiagramme verwendet werden, soll hier kurz eine Terminologie eingeführt werden, mit der man über diese Diagramme redet.
Ein Baum besteht aus sogenannten \textit{Knoten}, die durch \textit{Äste} oder \textit{Kanten} verbunden sind.
Die Knoten werden mit beliebigen Informationen beschriftet.
In diesem Abschnitt sind es der Einfachheit halber abstrakte Großbuchstaben, in konkreten Analysen Namen und Merkmale von sprachlichen Einheiten.
In Abbildung~\ref{fig:terminologiefuerbaumdiagramme064} ist ein Baum mit den Knoten A, B und C abgebildet.
Die Kanten verbinden C und A sowie C und B.

\index{Kante}

\begin{figure}[!htbp]
  \centering
  \begin{forest}
    [C
      [A][B]
    ]
  \end{forest}
  \caption{Einfacher Baum}
  \label{fig:terminologiefuerbaumdiagramme064}
\end{figure}

\index{Knoten!Tochter--}
\index{Knoten!Mutter--}

Die Kanten sind \textit{gerichtet}, zeigen also immer von oben nach unten, wobei der obere Knoten an der Kante \textit{Mutterknoten} und der untere \textit{Tochterknoten} genannt wird.
In Abbildung~\ref{fig:terminologiefuerbaumdiagramme064} ist C der Mutterknoten und A und B sind Tochterknoten.
Zwei verschiedene Tochterknoten eines Mutterknotens werden, wie zu erwarten, \textit{Schwestern} genannt (\zB A und B in Abbildung~\ref{fig:terminologiefuerbaumdiagramme064}).
In einem Baum gibt es immer genau einen Knoten (ganz oben) ohne Mutterknoten, die \textit{Wurzel}.\index{Knoten!Wurzel--}
Jeder andere Knoten hat genau einen Mutterknoten.
In Abbildung~\ref{fig:terminologiefuerbaumdiagramme065} bis~\ref{fig:terminologiefuerbaumdiagramme067} finden sich noch ein paar Beispiele für Bäume und Nicht-Bäume.

\begin{figure}[!htbp]
  \centering
  \begin{forest}
    [C
      [A]
      [B
        [D][E][F]
      ]
    ]
  \end{forest}
  \caption{Komplexerer Baum}
  \label{fig:terminologiefuerbaumdiagramme065}
\end{figure}

\begin{figure}[!htbp]
  \centering
  \begin{forest}
    grow=north
    [A
      [C][B]
    ]
  \end{forest}
  \caption[Beispiel für Nicht-Baum]{Beispiel für Nicht-Baum (mehrere Wurzeln, A hat mehrere Mutterknoten)}
  \label{fig:terminologiefuerbaumdiagramme066}
\end{figure}

\begin{figure}[!htbp]
  \centering
  \begin{forest}
    [G, name=NotreeG
      [E
        [A][B]
      ]
      [F
        [C]
        {\draw[-] (.north) -- (NotreeG.south);}
        [D]
      ]
    ]
  \end{forest}
  \caption[Anderes Beispiel für Nicht-Baum]{Anderes Beispiel für Nicht-Baum (C hat mehrere Mutterknoten)}
  \label{fig:terminologiefuerbaumdiagramme067}
\end{figure}

Schließlich muss erwähnt werden, dass die eckigen Klammern in Textbeispielen eine Konstituentenstruktur genauso wie ein Baum beschreiben können.
Was in einem Baum unter einem Knoten hängt, wird im geklammerten Textbeispiel in eine eckige Klammer gesetzt.
Ein Baum wie in Abbildung~\ref{fig:terminologiefuerbaumdiagramme064} entspricht einer Klammerstruktur [\Sub{C}~A~B~].
Die Beschriftung des Mutterknotens wird jeweils tiefgestellt an die öffnende Klammer geschrieben.
Der Baum in Abbildung~\ref{fig:terminologiefuerbaumdiagramme065} kann also wie in (\ref{ex:terminologiefuerbaumdiagramme068}) geschrieben werden.

\begin{exe}
  \ex{\label{ex:terminologiefuerbaumdiagramme068} [$_\textrm{C}$~A~[$_\textrm{B}$~D~E~F~]~]}
\end{exe}

\subsection{Phrasenschemata}
\label{sec:phrasenschemata}

Wir müssen nun überlegen, wie wir die Baupläne für Konstituentenstrukturen aufschreiben wollen.
Bisher haben wir sowohl für Baupläne als auch für Analysen Baumdiagramme verwendet, \zB in Abbildung~\ref{fig:syntaktischestruktur018} (S.~\pageref{fig:syntaktischestruktur018}) und Abbildung~\ref{fig:syntaktischestruktur019} (S.~\pageref{fig:syntaktischestruktur019}).
In einem Baum ist es nun immer der Fall, dass die Töchter unter jedem Knoten eine festgelegte Reihenfolge (von links nach rechts) haben, weil sich Kanten nicht überkreuzen dürfen.
Wenn die Tochterknoten wiederum Tochterknoten haben, gilt für diese dasselbe, und es ergibt sich insgesamt eine hierarchische Struktur mit einer linearen Ordnung.
Wir geben hier als \textit{Phrasenschemata} jeweils Baupläne für einzelne Knoten an, aus denen Bäume in der Analyse zusammengebaut werden dürfen.
Phrasenschemata werden als abstrakte Bäume dargestellt, und die Knoten werden in Boxen gesetzt, damit sie von solchen Bäumen unterschieden werden können, die eine Analyse von tatsächlichen Phrasen und Sätzen darstellen.
Ein Beispiel ist in Abbildung~\ref{fig:phrasenschemata069} dargestellt.

\begin{figure}[!htbp]
  \centering
  \begin{forest}
    phrasenschema
    [NP, Ephr, calign=last
      [Artikel, Eopt, Emult
        [Pronomen, Eopt]
      ]
      [A, Eoptrec]
      [N, Ehd]
    ]
  \end{forest}
  \caption{Vorläufiges Phrasenschema für Nominalphrasen}
  \label{fig:phrasenschemata069}
\end{figure}

Das Phrasensymbol (hier NP) wird in einen Kreis gesetzt.
Die Kopftochter steht in einer grauen Box, alle anderen Töchter in nicht gefüllten Boxen (hier Artikel, Pronomen, A und N).
Wenn der Rahmen der Box gestrichelt (hier Artikel, Pronomen und A) ist, ist die Tochter fakultativ, kann also weggelassen werden.
Eine doppelt eingerahmte Box (hier A) signalisiert, dass die Tochter wiederholbar ist.
Eine doppelt gestrichelt eingerahmte Box (hier A) steht also für eine Tochter, die gar nicht stehen muss, aber einmal oder mehrmals stehen kann.
Wenn es mehrere (einander ausschließende) Möglichkeiten gibt, eine Position zu besetzen, stehen die zwei oder mehr Möglichkeiten in Boxen direkt untereinander (hier Artikel und Pronomen).
In konkreten Analysen werden die Rahmen weggelassen, lediglich das Symbol für den Kopf der Phrase wird fettgedruckt (zum Kopf vgl.\ Abschnitt~\ref{sec:phrasenkoepfeundmerkmale}).

Mit dem Schema in Abbildung~\ref{fig:phrasenschemata069} definieren wir, dass eine NP aus einem optionalen Artikel oder Pronomen, beliebig vielen optionalen Adjektiven und einem obligatorischen N-Kopf besteht.
Damit kann der Baum in Abbildung~\ref{fig:phrasenschemata070} als Analyse für \textit{ein leckerer geräucherter Tofu} gebaut werden.

\begin{figure}[!htbp]
  \centering
  \begin{forest}
    [NP, calign=last
      [Artikel
        [\it ein]
      ]
      [A
        [\it leckerer]
      ]
      [A
        [\it geräucherter]
      ]
      [\textbf{N}
        [\it Tofu]
      ]
    ]
  \end{forest}
  \caption{Nominalphrase (NP), vorläufige Analyse}
  \label{fig:phrasenschemata070}
\end{figure}

Schließlich werden in Phrasenschemata Rektionsbeziehungen durch Pfeile angezeigt, und zwar durch einen durchgehenden Pfeil für obligatorische Rektion und eine gestrichelten Pfeil für optionale Rektion.
Im vorläufigen Phrasenschema für die Präpositionalphrase in Abbildung~\ref{fig:phrasenschemata071} zeigt der Pfeil also \zB an, dass der P-Kopf die NP obligatorisch regiert.

\begin{figure}[!htbp]
  \centering
  \begin{forest}
    phrasenschema
    [PP, calign=first, Ephr
      [P, Ehd, name=Pkopfprelim]
      [NP, Eobl]
      {\draw [bend left=45, <-] (.south) to (Pkopfprelim.south);}
    ]
  \end{forest}
  \caption{Vorläufiges Phrasenschema für Präpositionalphrasen}
  \label{fig:phrasenschemata071}
\end{figure}

\subsection{Phrasen, Köpfe und Merkmale}
\label{sec:phrasenkoepfeundmerkmale}

\index{Phrase}

Die \textit{Phrase} ist neben dem Wort die wichtigste Einheit in der Syntax.
Während die Wörter die kleinsten Konstituenten und die Sätze die größten sind, bilden die Phrasen genau die Zwischenebene, die es uns erlaubt, Sätze eben gerade nicht als Abfolgen von Wörtern zu beschreiben,
sondern eleganter und effizienter als aus bereits größeren Konstituenten zusammengesetzt.
Die Idee ist dabei, dass (fast) jedes Wort als \textit{Kopf} zunächst eine eigene Phrase bildet, innerhalb derer es diejenigen anderen Wörter bzw.\ Phrasen zu sich nimmt, die von ihm abhängen.
Erst wenn die Phrase vollständig ist, kann sie in Sätze oder andere Phrasen eingesetzt werden.
Wir illustrieren zunächst in (\ref{ex:phrasenkoepfeundmerkmale072}) und (\ref{ex:phrasenkoepfeundmerkmale077}), was es bedeutet, dass Phrasen oder Wörter von Köpfen abhängen.

\begin{exe}
  \ex\label{ex:phrasenkoepfeundmerkmale072}
  \begin{xlist}
    \ex[]{\label{ex:phrasenkoepfeundmerkmale073} Die Bürger gedenken des Absturzes von Hasloh.}
    \ex[]{\label{ex:phrasenkoepfeundmerkmale074} Die Bürger stürmen das Kanzleramt.}
    \ex[*]{\label{ex:phrasenkoepfeundmerkmale075} Die Bürger gedenken (des) von Hasloh.}
    \ex[*]{\label{ex:phrasenkoepfeundmerkmale076} Die Bürger gedenken Stockhausens von Hasloh.}
  \end{xlist}
  \ex\label{ex:phrasenkoepfeundmerkmale077}
  \begin{xlist}
    \ex[]{\label{ex:phrasenkoepfeundmerkmale078} Wir nehmen an, dass supermassive schwarze Löcher existieren.}
    \ex[]{\label{ex:phrasenkoepfeundmerkmale079} Wir nehmen an, dass es regnet.}
    \ex[*]{\label{ex:phrasenkoepfeundmerkmale080} Wir nehmen an, dass es supermassive regnet.}
  \end{xlist}
\end{exe}

\index{Valenz}

In (\ref{ex:phrasenkoepfeundmerkmale072}) ist die Angelegenheit klar.
Die NPs \textit{des Absturzes} und \textit{das Kanzleramt} saturieren eine Valenzstelle der jeweiligen Verben \textit{gedenken} und \textit{stürmen}.
Ihre Kasus sind regiert, und ihre grammatische Existenz ist damit vollständig bedingt durch die Anwesenheit ihres Kopfes (des Verbs), erst recht in der spezifischen Kasusform.
Die von einer Präposition eingeleitete Gruppe \textit{von Hasloh} ist keine Ergänzung oder Angabe zum Verb, aber ihre Anwesenheit im Satz hängt von der Anwesenheit einer Ergänzung des Verbs (\textit{Absturzes}) ab, wie man in (\ref{ex:phrasenkoepfeundmerkmale075}) und (\ref{ex:phrasenkoepfeundmerkmale076}) sieht.
Lässt man das Substantiv weg, wird der Satz ungrammatisch (\ref{ex:phrasenkoepfeundmerkmale075}), egal ob der Artikel \textit{des} stehenbleibt oder nicht.
Aber auch wenn wie in (\ref{ex:phrasenkoepfeundmerkmale076}) die falsche Art von Nomen statt des normalen Substantivs genommen wird (zum Beispiel ein Eigenname), kann man die Präpositionalphrase \textit{von Hasloh} nicht mehr verwenden (zum Begriff der \textit{Präpositionalphrase} vgl.\ Abschnitt~\ref{sec:praepositionalphrase}).
Man sollte also \textit{von Hasloh} hier als Ergänzung oder Angabe von \textit{Absturzes} behandeln.

Ähnlich ist es in (\ref{ex:phrasenkoepfeundmerkmale077}).
Das Adjektiv \textit{supermassive} füllt mit Sicherheit keine Valenzstelle von \textit{Löcher}, aber sein Auftreten hängt in dieser Form eindeutig vom Substantiv \textit{Löcher} ab.
In einem Satz, in dem kein passendes Substantiv vorkommt, wie in (\ref{ex:phrasenkoepfeundmerkmale079}), kann es nicht stehen, was zur Ungrammatikalität von (\ref{ex:phrasenkoepfeundmerkmale080}) führt.
Ob es sich um Ergänzungen oder Angaben handelt, ist also aus diesem Blickwinkel egal:
Fast alle syntaktischen Einheiten in einem Satz hängen von einer anderen Einheit im selben Satz ab, können also nur auftreten, wenn diese andere Einheit auch auftritt.
Diese Relation nennt man auch \textit{Dependenz}, s.\ Definition~\ref{def:dependenz}.

\Definition{Dependenz}{\label{def:dependenz}%
Eine Konstituente A ist von einer Konstituente B im selben Satz \textit{abhängig} (\textit{dependent}), wenn die Anwesenheit von B eine Bedingung für die Anwesenheit und\slash oder die Form von A ist.
Dependenz ist nie zirkulär, keine Konstituente ist also von sich selber direkt oder indirekt abhängig.
\index{Dependenz}
}

Als Faustregel kann gelten, dass Ergänzungen zu dem Wort dependent sind, dessen Valenzstelle sie saturieren, so wie \textit{des Absturzes} zu \textit{gedenken} in (\ref{ex:phrasenkoepfeundmerkmale073}).
Außerdem sind alle Angaben zu den Wörtern dependent, welche sie modifizieren, so wie \textit{supermassive} zu \textit{Löchern} in (\ref{ex:phrasenkoepfeundmerkmale078}), wobei im nominalen Bereich im Deutschen dann Kongruenzrelationen bestehen.

Jetzt können wir die \textit{Phrase}, die einfach nur ein besonderer Typ von syntaktischer Konstituente ist, genauer definieren.
Der Begriff des \textit{Kopfes} wird zusammen mit dem der \textit{Phrase} in Definition~\ref{def:phrase} definiert, da Phrasen einen Kopf haben.

\Definition{Phrase und Kopf}{\label{def:phrase}%
Eine \textit{Phrase} ist eine syntaktische Konstituente, in der genau ein Wort der \textit{Kopf} ist.
Innerhalb der Phrase sind alle anderen Wörter und Phrasen zum Kopf dependent.
Die grammatischen Merkmale der Phrase werden durch die Merkmale des Kopfes bestimmt.
\index{Kopf!Phrase}
}

Den sehr wichtigen letzten Satz von Definition~\ref{def:phrase} müssen wir noch illustrieren.
Dazu können wir wieder die Beispiele (\ref{ex:phrasenkoepfeundmerkmale072}) und (\ref{ex:phrasenkoepfeundmerkmale077}) heranziehen.
Zunächst können wir feststellen, dass [\textit{des Absturzes von Hasloh}] eine Konstituente ist.%
\footnote{Um sich dies zu verdeutlichen, können die besprochenen Tests angewendet werden.}
In dieser Konstituente, genauer gesagt in dieser Phrase, kommen \textit{Absturzes} und \textit{Hasloh} als Köpfe infrage.
Beide können die nominale Valenzstelle eines Vollverbs saturieren.
Nur eines von beiden, nämlich \textit{Absturzes} steht allerdings in dem Kasus, der im Kontext des Satzes der richtige ist, nämlich im Genitiv.
Dies bedeutet, dass die gesamte Phrase seinem regierenden Verb nur den Genitiv von \textit{Absturzes} zeigt, nicht etwa den Dativ von \textit{Hasloh}.
Dieser Dativ spielt nur innerhalb der Phrase [\textit{des Absturzes von Hasloh}] -- genauer sogar nur innerhalb von [\textit{von Hasloh}] -- eine Rolle.
Die grammatischen Eigenschaften der gesamten Phrase werden hingegen vom Kopf, also \textit{Absturzes} bestimmt.

Aus genau diesem Grund benennen wir eine Phrase auch immer nach der Klasse des Kopfes:
Phrasen mit einem Adjektiv (A) als Kopf heißen \textit{Adjektivphrasen} (AP), Phrasen mit einem Verb (V) als Kopf heißen \textit{Verbphrasen} (VP) usw., vgl.\ Tabelle~\ref{tab:phrasenkoepfeundmerkmale082}.
Phrasen mit einem Substantiv oder Pronomen als Kopf heißen \textit{Nominalphrasen} (NP), weil sie sich gleich verhalten und es daher günstiger ist, nicht getrennt von \textit{Substantivphrasen} und \textit{Pronominalphrasen} zu sprechen.
Der Kopf einer Phrase ist innerhalb dieser Phrase typischerweise nicht weglassbar.\label{abs:phrasenkoepfeundmerkmale081}

\begin{table}[!htbp]
  \resizebox{\textwidth}{!}{
    \begin{tabular}{lll}
      \lsptoprule
      \textbf{Kopf} & \textbf{Phrase} & \textbf{Beispiel} \\
      \midrule
      Nomen (Substantiv, Pronomen) & Nominalphrase (NP) & \textit{die tolle Aufführung} \\
      Adjektiv & Adjektivphrase (AP) & \textit{sehr schön} \\
      Präposition & Präpositionalphrase (PP) & \textit{in der Uni} \\
      Adverb & Adverbphrase (AdvP) & \textit{total offensichtlich} \\
      Verb & Verbphrase (VP) & \textit{Sarah den Kuchen gebacken hat} \\
      Komplementierer & Komplementiererphrase (KP) & \textit{dass es läuft} \\
      \lspbottomrule
    \end{tabular}
  }
  \caption{Phrasenbezeichnungen nach ihren Köpfen}
  \label{tab:phrasenkoepfeundmerkmale082}
\end{table}

Wenn wir Wörter und alle anderen Einheiten (auch Phrasen) im Rahmen der Grammatik wieder als eine Menge von Merkmalen und Werten definieren, können ganz allgemeine Prinzipien des Phrasenaufbaus auch anhand von Merkmalen definiert werden.
Was Tabelle~\ref{tab:phrasenkoepfeundmerkmale082} eigentlich illustriert, ist die Merkmalsübereinstimmung zwischen der Phrase und ihrem Kopf.
Abbildung~\ref{fig:phrasenkoepfeundmerkmale083} zeigt schematisch, was gemeint ist.
In einer AP \textit{sehr schön} hat der Kopf \textit{schön} den Wert \textit{adj} für \textsc{Klasse}, weil er ein Adjektiv ist.
Beim Aufbau der Phrase muss jetzt einfach nur eine Regel oder ein Schema zum Einsatz kommen, das den Wert des \textsc{Klasse}-Merkmals der Phrase mit dem des Kopfes gleichsetzt.
Der \textsc{Klasse}-Wert des Nicht-Kopfes (hier \textit{ptkl}) ist völlig unwesentlich, wenn wir einmal auf der AP-Ebene angekommen sind.
Es zeigt sich damit auch, dass Bezeichnungen wie \textit{Wortart} oder \textit{Wortklasse} eigentlich zu kurz gegriffen sind, weil es nicht um Wörter, sondern ganz allgemein um Klassen syntaktischer Einheiten geht.
Wir bleiben aus Bequemlichkeit bei der Bezeichnung \textit{Wortklasse}.

\begin{figure}[!htbp]
  \centering
  \begin{forest}
    [AP\\{[\textsc{Klasse}: \textbf{adj}, \textsc{Segmente}: \textit{sehr schön}]}, calign=last
      [Ptkl\\{[\textsc{Klasse}: \textbf{ptkl}, \textsc{Segmente}: \textit{sehr}]}]
      [\textbf{A}\\{[\textsc{Klasse}: \textbf{adj}, \textsc{Segmente}: \textit{schön}]}]
    ]
  \end{forest}
  \caption{Merkmalsübereinstimmung zwischen Kopf und Phrase}
  \label{fig:phrasenkoepfeundmerkmale083}
\end{figure}

Es muss aber nicht nur die Wortklasse vom Kopf zur Phrase kopiert werden, sondern auch alle anderen für die weitere Strukturbildung relevanten Merkmale, \zB Kasus- und Kongruenzmerkmale.
Diese Merkmale nennt man \textit{Kopf-Merkmale}.
Das entsprechende \textit{Kopf-Merkmal-Prinzip} wird in Definition~\ref{def:hfp} angegeben.

\Definition{Kopf-Merkmal-Prinzip}{\label{def:hfp}%
Die Werte der Kopf-Merkmale des Kopfes und der Phrase, die er bildet, sind immer identisch.
\index{Kopf!Kopf-Merkmal-Prinzip}
}

Es bleibt anzumerken, dass wir hier davon ausgehen, dass einige \textit{Funktionswörter} wie Partikeln (s.\ Abschnitt~\ref{sec:adverbenadkopulasundpartikeln}) oder Artikelwörter (s.\ Definition~\ref{def:artikelwort} auf S.~\pageref{def:artikelwort}) keine eigenen Phrasen bilden können und direkt in größere Einheiten eingesetzt werden müssen.
Eine Partikel oder ein Artikelwort sind also niemals Köpfe.
Auch hierzu (besonders im Fall der Artikelwörter oder \textit{Determinierer}) haben manche Theorien andere Lösungen entwickelt, bei denen auch diese Wörter Köpfe sind und Phrasen bilden.
Das kann man im entsprechenden theoretischen Rahmen durchaus so machen, hier wird aber Definition~\ref{def:funktionswort} zugrundegelegt.

\Definition{Funktionswort}{\label{def:funktionswort}%
Ein \textit{Funktionswort} hat keinen Kopfstatus, bildet keine Phrasen und ist nicht vorfeldfähig.
Es ist damit abhängig (unselbständig) in dem Sinn, dass es nicht alleine auftreten kann und keine Rektion bzw.\ Valenz hat.
Als Funktionswörter fassen wir Artikelwörter, Konjunktionen und sonstige Partikeln auf.
\index{Funktionswort}
\index{Partikel}
\index{Artikelwort}
\index{Konjunktion}
}

Außerdem wird in Kapitel~\ref{sec:saetze} eine Analyse von unabhängigen Sätzen vertreten, bei der der Satz selber zwar einen eigenen Phrasentyp (\zB Symbol S), aber keinen Kopf hat.
Die Gründe dafür liegen in der besonderen Art, wie im Deutschen unabhängige Sätze gebildet werden.
In Kapitel~\ref{sec:phrasen} geht es jetzt aber zunächst einmal um den Aufbau der kleineren Einheiten, also im Prinzip der Phrasen, die in Tabelle~\ref{tab:phrasenkoepfeundmerkmale082} genannt sind.

\Zusammenfassung{%
Eine Phrase hat typischerweise einen Kopf, der ihre wesentlichen Merkmale bestimmt: die Kopf-Merkmale.
Innerhalb einer Phrase sind alle Konstituenten direkt oder indirekt vom Kopf abhängig (dependent).
Funktionswörter sind prinzipiell abhängige Wörter, die keine eigenen Phrasen bilden.
Syntaxbäume müssen gewissen Regeln entsprechen, \zB dass sie nur einen Wurzelknoten haben.
}

\Uebungen

\Uebung{konstituentenstruktur01} \label{exc:konstituentenstruktur01} Führen Sie für die eingeklammerten potentiellen Konstituenten je zwei Konstituententests Ihrer Wahl durch (vgl.\ Abschnitt~\ref{sec:konstituententests}, S.~\pageref{sec:konstituententests}) und entscheiden Sie auf Basis dessen, ob es sich um Konstituenten handelt.
Dass einige der Sätze vielleicht nicht ganz akzeptabel klingen, ist insofern Absicht, als das die Anwendung der Methode etwas erschwert.%
\footnote{Siglen der Belege im DeReKo: A09\slash DEZ.02319, BRZ09\slash SEP.15424, HMP08\slash FEB.00096, NON07\slash OKT.07665, A97\slash DEZ.42679, K97\slash MAI.35888, DIV\slash APS.00001, DIV\slash APS.00001, NUZ06\slash MAR.01677}

\begin{enumerate}
  \item So nimmt er sich [während den Spielen] auch zurück, denn die taktischen Anweisungen gibt es vorher.
  \item Parteichef wird [sehr wahrscheinlich Sigmar Gabriel].
  \item Ein Vermieter kann mittels eines Formularvertrags keine Betriebskosten für die Reinigung eines Öltanks [auf den Mieter umlegen].
  \item Die beste Möglichkeit vergab ein [Gäste-Stürmer, dessen Schuss knapp am Gehäuse drüber ging].
  \item Die vier Musiker lösen ihre Band nach dreieinhalb Jahren auf, [weil sich der Sänger musikalisch verändern will].
  \item In der Gemeindestube weiß man von diesen konkreten Plänen [überhaupt nichts].
  \item Wagas suchte eifrig nach einem dickeren Ast, [um zu helfen].
  \item Wagas suchte eifrig [nach einem] dickeren Ast, um zu helfen.
  \item Auch viele Beobachter sprachen von einer sterilen Debatte [ohne spannende Passagen].
\end{enumerate}

\Uebung{konstituentenstruktur02} \label{exc:konstituentenstruktur02} Die in folgenden Sätzen eingeklammerten Wörter sind Konstituenten.
Sind sie Satzglieder oder nicht?
Verwenden Sie nach dem Muster des ersten Beispiels nur den Vorfeldtest, um die Frage zu entscheiden.%
\footnote{Siglen der Belege im DeReKo: BRZ06\slash MAI.05936, A09\slash DEZ.02319, RHZ04\slash JUL.20475, HMP08\slash FEB.00096, NON07\slash OKT.07665, A97\slash DEZ.42679, K97\slash MAI.35888, K97\slash MAI.35888, DIV\slash APS.00001, RHZ98\slash AUG.01367, K98\slash SEP.69009}

\begin{enumerate}
  \item Es wird spannend sein, [den Wahlabend so direkt zu verfolgen und den direkten Kontakt mit dem Wähler zu erleben].
    \begin{itemize}
      \item \VfTest [Den Wahlabend so direkt zu verfolgen und den direkten Kontakt mit dem Wähler zu erleben], wird spannend sein.
    \end{itemize}
  \item Es wird spannend sein, den Wahlabend so direkt zu verfolgen und den direkten Kontakt [mit dem Wähler] zu erleben.
  \item So nimmt [er] sich während den Spielen auch zurück, denn die taktischen Anweisungen gibt es vorher.
  \item Dann hätten die 37 [sehr wahrscheinlich] problemlos in Deutschland Asyl erhalten.
  \item Ein Vermieter kann mittels eines Formularvertrags keine Betriebskosten für die Reinigung eines Öltanks auf [den Mieter] umlegen.
  \item Die beste Möglichkeit vergab [ein Gäste-Stürmer, dessen Schuss knapp am Gehäuse drüber ging].
  \item Die vier Musiker lösen ihre Band nach dreieinhalb Jahren auf, [weil sich der Sänger musikalisch verändern will].
  \item In [der Gemeindestube] weiß man von diesen konkreten Plänen überhaupt nichts.
  \item In der Gemeindestube weiß man [von diesen konkreten Plänen] überhaupt nichts.
  \item Wagas suchte eifrig nach einem dickeren Ast, [um zu helfen].
  \item Dort erwarteten sie [außer Kaffee und Kuchen] gekühlte Getränke und Leckeres vom Grill.
  \item Alle [bis auf den Pürierstab-Kollegen] grinsten oder kudderten.
\end{enumerate}

\Uebung[\tristar]{konstituentenstruktur03} \label{exc:konstituentenstruktur03} Diskutieren Sie (\ref{ex:phrasenkoepfeundmerkmale084}) aus \citet[1--2]{Dekuthy2002} im Kontrast zu (\ref{ex:konstituentenundsatzglieder052}) und (\ref{ex:konstituentenundsatzglieder055}) auf S.~\pageref{ex:konstituentenundsatzglieder052} als Problem für den Satzgliedbegriff (Definition~\ref{def:satzglied} auf S.~\pageref{def:satzglied}).
Gehen Sie dabei davon aus, dass der Satz akzeptabel bzw.\ grammatisch ist.

\begin{exe}
  \ex{\label{ex:phrasenkoepfeundmerkmale084} Über Syntax hat Sarah sich ein Buch ausgeliehen.}
\end{exe}
