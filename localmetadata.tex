\title{Einführung in die grammatische Beschreibung des Deutschen}
\subtitle{Dritte, überarbeitete und erweiterte Auflage}
\BackTitle{Einführung in die grammatische Beschreibung des Deutschen}
\BackBody{%\begin{sloppypar}
\textit{Einführung in die grammatische Beschreibung des Deutschen} ist eine Einführung in die deskriptive Grammatik am Beispiel des gegenwärtigen Deutschen in den Bereichen Phonetik, Phonologie, Morphologie, Syntax und Graphematik.
Das Buch ist für alle geeignet, die sich für die Grammatik des Deutschen interessieren, vor allem aber für Studierende der Germanistik bzw.\ Deutschen Philologie.
Im Vordergrund steht die Vermittlung grammatischer Erkenntnisprozesse und Argumentationsweisen auf Basis konkreten sprachlichen Materials.
Es wird kein spezieller theoretischer Rahmen angenommen, aber nach der Lektüre sollten Leser in der Lage sein, sowohl deskriptiv ausgerichtete Forschungsartikel als auch theorienahe Einführungen lesen zu können.
Das Buch enthält zahlreiche Übungsaufgaben, die im Anhang gelöst werden.

Die dritte Auflage behebt einige Tipp- und Stilfehler und bietet einige neue Vertiefungsblöcke.

\vspace{1\baselineskip}

\noindent\textbf{Roland Schäfer} studierte Sprachwissenschaft und Japanologie an der Phi\-lipps-\-Universität Marburg.
Er war wissenschaftlicher Mitarbeiter an der Georg-August Universität Göttingen und der Freien Universität Berlin.
Er promovierte 2008 an der Georg-August Universität Göttingen mit einer theoretischen Arbeit zur Syntax-Semantik-Schnittstelle.
2018 hat die Humboldt-Universität zu Berlin Roland Schäfers Habilitationsverfahren mit einer Arbeit zum Thema \textit{Probabilistic German Morphosyntax} eröffnet.
Seine aktuellen Forschungsschwerpunkte sind die korpusbasierte und kognitiv orientierte Morphosyntax und Graphematik des Deutschen sowie die Korpuserstellung.
Von 2015–2018 leitet er das DFG-finanzierte Projekt \textit{Linguistische Web-Charakterisierung und Webkorpuserstellung} an der Freien Universität Berlin.
%\end{sloppypar}
}
\dedication{%
\large Für Alma, Ariel, Block, Frau Brüggenolte, Chopin, Christina, Doro, Edgar, Elena, Elin, Emma, den ehemaligen FCR Duisburg, Frida, Gabriele, Hamlet, Helmut Schmidt, Henry, Ian Kilmister, Ingeborg, Ischariot, Jean-Pierre, Johan, Kurt, Lemmy, Liv, Marina, Martin, Mats, Mausi, Michelle, Nadezhda, Herrn Oelschlägel, Oma, Opa, Pavel, Philly, Sarah, Scully, Stig, Tania, Tante Klärchen, Tarek, Tatjana, Herrn Uhl, Ullis schreckhaften Hund, Vanessa und so.\\[\baselineskip]Wenn das schonmal klar sein würde.\\}
\typesetter{Roland Schäfer}
\proofreader{Thea Dittrich}
\author{Roland Schäfer}


% \BookDOI{}%ask coordinator for DOI
\renewcommand{\lsISBNdigital}{000-0-000000-00-0}
\renewcommand{\lsISBNhardcover}{000-0-000000-00-0}
\renewcommand{\lsISBNsoftcover}{000-0-000000-00-0}
\renewcommand{\lsISBNsoftcoverus}{000-0-000000-00-0}
\renewcommand{\lsSeries}{tbls} % use lowercase acronym, e.g. sidl, eotms, tgdi
\renewcommand{\lsSeriesNumber}{2} %will be assigned when the book enters the proofreading stage
\renewcommand{\lsURL}{http://langsci-press.org/catalog/book/000} % contact the coordinator for the right number

 
 
 
 
  
